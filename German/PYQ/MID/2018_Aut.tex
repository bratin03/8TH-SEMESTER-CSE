\documentclass{article}
\usepackage[utf8]{inputenc}
\usepackage{xcolor}

\title{
INDIAN INSTITUTE OF TECHNOLOGY KHARAGPUR \\ Mid-Autumn Semester 2017-18
}

\author{}

\date{}

\begin{document}

\maketitle

\noindent
Date of Examination: Session (FN/AN) Duration: 2 hrs \qquad Full Marks: 60/2 \\
Subject No. : HS51604 \qquad Subject : GERMAN \\
Department/Center/School : Humanities \& Social Sciences

\begin{enumerate}
    \item Read the following passage carefully and answer the questions that follow:

    Ich heiße \textcolor{red}{(My name is)} Maria Klein. (My name is Maria Klein.) \\
    Ich bin \textcolor{red}{(I am)} 25 Jahre alt \textcolor{red}{(years old)}. (I am 25 years old.) \\
    Sie kennen \textcolor{red}{(You know)} mich nicht \textcolor{red}{(me not)}. (You don't know me.) \\
    Ich arbeite \textcolor{red}{(I work)} schon 4 Jahre \textcolor{red}{(already 4 years)} in einer Bankfilliale \textcolor{red}{(in a bank branch)}. (I have been working for 4 years in a bank branch.) \\
    Die Arbeit \textcolor{red}{(The work)} macht mir hier viel Spaß \textcolor{red}{(makes me much fun here)}. (The work makes me much fun here.) \\
    Man lernt \textcolor{red}{(One learns)} bei der Arbeit viele Menschen \textcolor{red}{(many people)} kennen \textcolor{red}{(meet)}. (One learns to meet many people at work.) \\
    Ich komme \textcolor{red}{(I come)} aus Berlin. (I come from Berlin.) \\
    Meine Eltern \textcolor{red}{(My parents)} wohnen \textcolor{red}{(live)} dort \textcolor{red}{(there)}. (My parents live there.) \\
    Manchmal \textcolor{red}{(Sometimes)} fahre ich am Wochenende \textcolor{red}{(I drive on the weekend)} zu ihnen \textcolor{red}{(to them)}. (Sometimes I drive to them on the weekend.) \\
    Hier wohne \textcolor{red}{(Here I live)} ich bei meiner Tante \textcolor{red}{(I with my aunt)}. (Here I live with my aunt.) \\
    Sie ist Witwe \textcolor{red}{(She is a widow)}. (She is a widow.) \\
    Ich habe bei ihr ein Zimmer \textcolor{red}{(I have a room with her)}. (I have a room with her.) \\
    Berlin ist eine schöne Stadt \textcolor{red}{(Berlin is a beautiful city)}. (Berlin is a beautiful city.) \\
    Die Leute \textcolor{red}{(The people)} hier sind sehr nett \textcolor{red}{(are very nice)} und freundlich \textcolor{red}{(and friendly)}. (The people here are very nice and friendly.) \\
    Berlin ist die Hauptstadt \textcolor{red}{(Berlin is the capital)} von Deutschland \textcolor{red}{(of Germany)}. (Berlin is the capital of Germany.)

    Morgens \textcolor{red}{(In the mornings)} stehe ich früh auf \textcolor{red}{(I get up early)}. (In the mornings I get up early.) \\
    Meine Bank \textcolor{red}{(My bank)} hier öffnet \textcolor{red}{(opens)} um 8.30 Uhr \textcolor{red}{(at 8:30)}. (My bank here opens at 8:30.) \\
    Dann muß ich natürlich \textcolor{red}{(Then I must of course)} hier sein \textcolor{red}{(be here)}. (Then I must of course be here.) \\
    Ich fahre immer \textcolor{red}{(I always go)} mit öffentlichen Verkehrsmitteln \textcolor{red}{(with public transport)} hierher \textcolor{red}{(here)}. (I always go here with public transport.) \\
    Ich habe wohl \textcolor{red}{(I probably have)} einen kleinen Wagen \textcolor{red}{(a small car)}, aber den lasse ich \textcolor{red}{(but that I leave)} zu Hause \textcolor{red}{(at home)}. (I probably have a small car, but I leave that at home.) \\
    Sie verstehen schon \textcolor{red}{(You understand already)}, hier in der Stadt gibt es wenige Parkplätze \textcolor{red}{(here in the city there are few parking places)}. (You understand already, here in the city there are few parking places.) \\
    Das ist ein Problem \textcolor{red}{(That is a problem)} in dem ganzen Deutschland \textcolor{red}{(in the whole Germany)}. (That is a problem in the whole of Germany.) \\
    Ich fahre auch sehr oft mit dem Zug \textcolor{red}{(I also often travel by train)}. (I also often travel by train.) \\
    In Deutschland fahren die Züge sehr schnell \textcolor{red}{(In Germany the trains travel very fast)} und pünktlich \textcolor{red}{(and punctually)}. (In Germany the trains travel very fast and punctually.) \\
    Entschuldigen Sie bitte \textcolor{red}{(Excuse me please)}, aber jetzt muß ich wieder etwas tun \textcolor{red}{(but now I have to do something again)}. (Excuse me please, but now I have to do something again.) \\
    Da kommt gerade \textcolor{red}{(There comes just now)} mein Chef \textcolor{red}{(my boss)} in die Schalterhalle \textcolor{red}{(into the counter hall)}. (There comes just now my boss into the counter hall.) \\
    Vielleicht bis später einmal \textcolor{red}{(Maybe until later sometime)}. (Maybe until later sometime.) \\
    Ich wünsche Ihnen eine gute Zeit \textcolor{red}{(I wish you a good time)}. (I wish you a good time.)

    Answer the following questions based on the above text:
    \begin{enumerate}
        \item[(a)] Wo arbeitet Maria Klein? (Where does Maria Klein work?)
        \item[(b)] Woher kommt Maria Klein? (Where does Maria Klein come from?)
        \item[(c)] Wie alt ist sie? (How old is she?)
        \item[(d)] Wann öffnet die Bank? (When does the bank open?)
        \item[(e)] Was ist ein Problem in dem ganzen Deutschland? (What is a problem in the whole of Germany?)
    \end{enumerate}
    \item Form sentences with the opposites of the following words (any ten): \\
    freundlich \textcolor{red}{(friendly)}, alt \textcolor{red}{(old)}, gut \textcolor{red}{(good)}, schwer \textcolor{red}{(difficult)}, klein \textcolor{red}{(small)}, schnell \textcolor{red}{(fast)}, billig \textcolor{red}{(cheap)}, schön \textcolor{red}{(beautiful)}, immer \textcolor{red}{(always)}, früh \textcolor{red}{(early)}, hell \textcolor{red}{(bright)}, fleißig \textcolor{red}{(hardworking)}
    \item Write the following numbers in words:
    \item Use the correct forms of the verbs given in the brackets (any ten):
    \begin{enumerate}
        \item[(a)] Die Dame (trinken) Tee mit Milch und Zucker. (The lady (drink) tea with milk and sugar.)
        \item[(b)] Der Student (schreiben) das Wort in das Heft. (The student (write) the word in the notebook.)
        \item[(c)] Der Japaner (sprechen) Hindi gut. (The Japanese (speak) Hindi well.)
        \item[(d)] Der Professor (erklären) den Studenten die Sätze. (The professor (explain) the sentences to the students.)
        \item[(e)] Das Mädchen (bringen) dem Gast ein Glas Wasser. (The girl (bring) the guest a glass of water.)
        \item[(f)] Nach dem Essen (spielen) er immer das Tischtennis. (After the meal (play) he always table tennis.)
        \item[(g)] Herr Brekle (arbeiten) 10 Stunden täglich. (Mr. Brekle (work) 10 hours daily.)
        \item[(h)] Der Ausländer (fahren) heute nach Neu Delhi. (The foreigner (travel) today to New Delhi.)
        \item[(i)] Der Mann (geben) dem Kind ein neues Buch. (The man (give) the child a new book.)
        \item[(j)] Der Hafen (sein) sehr wichtig. (The harbor (be) very important.)
        \item[(k)] Die alte Dame (haben) ein schönes Haus. (The old lady (have) a beautiful house.)
    \end{enumerate}
    \item Replace the underlined portions by appropriate pronouns (any five sentences):
    \begin{enumerate}
        \item[(a)] Der Arbeiter hat ein neues Auto. (The worker has a new car.)
        \item[(b)] Der Lehrerin kauft die Studentin eine Uhr. (The student buys the teacher a watch.)
        \item[(c)] Er schenkt seinem Freund einen neuen Füller. (He gives his friend a new pen.)
        \item[(d)] Sie zeigt den Studenten die schönen Bilder. (She shows the students the beautiful pictures.)
        \item[(e)] Dieser Schüler hat ein neues Wörterbuch. (This student has a new dictionary.)
        \item[(f)] Frau Müller kauft den Kindern viele Bücher. (Mrs. Müller buys the children many books.)
    \end{enumerate}
    \item Form sentences with the following groups of words (any five)!
    \begin{enumerate}
        \item[(a)] Robert, alt, 20, Jahre, sein, und, in, Stuttgart, studieren (Robert, old, 20, years, be, and, in, Stuttgart, study)
        \item[(b)] der Lehrer, falschen, verbessern, die Antworten (the teacher, false, improve, the answers)
        \item[(c)] der Professor, die Theorie, den Studenten, erklären (the professor, the theory, the students, explain)
        \item[(d)] er, 10, arbeiten, täglich, Stunden, lesen, die Zeitung, und, abends (he, 10, work, daily, hours, read, the newspaper, and, evenings)
        \item[(e)] heute, fahren, nach, die Studenten, Kalkutta (today, travel, to, the students, Calcutta)
        \item[(f)] viele, arbeiten, deutsche, hier, Studenten, und, Geld, verdienen (many, work, German, here, students, and, money, earn)
    \end{enumerate}
    \item Frame questions to which the underlined words provide answers (any ten)!
    \begin{enumerate}
        \item[(a)] Dieses Buch kostet nur Euro 48,- (This book costs only Euro 48,-)
        \item[(b)] Mein Bruder kommt nach Kharagpur heute abend. (My brother comes to Kharagpur this evening.)
        \item[(c)] Das Zimmer ist sehr teuer aber komfortabel. (The room is very expensive but comfortable.)
        \item[(d)] Der Zug fährt um 9 Uhr von Kharagpur ab. (The train departs at 9 o'clock from Kharagpur.)
        \item[(e)] Morgen fahren wir nach Neu Delhi. (Tomorrow we travel to New Delhi.)
        \item[(f)] Mein Professor ist sehr nett und freundlich. (My professor is very nice and friendly.)
        \item[(g)] Seit 2005 studiert er in Kharagpur. (Since 2005 he studies in Kharagpur.)
        \item[(h)] Diese Studenten kommen aus Deutschland. (These students come from Germany.)
        \item[(i)] Es ist schon 10.00 Uhr. (It is already 10.00 o'clock.)
        \item[(j)] Der Züricher wohnt jetzt in Benaras. (The man from Zurich lives now in Benaras.)
        \item[(k)] Die Leute fahren nach Berlin morgen. (The people travel to Berlin tomorrow.)
    \end{enumerate}
\end{enumerate}

\end{document}
