\documentclass{article}
\usepackage[utf8]{inputenc}
\usepackage{xcolor}

\title{
INDIAN INSTITUTE OF TECHNOLOGY
}



\date{}

\begin{document}

\maketitle

\section*{Q.1. Read the following passage carefully and answer the questions that follow:}

Der Briefträger \textcolor{red}{(The postman)} klingelt \textcolor{red}{(rings)}. (The postman rings.) \\
Frau Braun öffnet \textcolor{red}{(opens)} die Tür \textcolor{red}{(door)}. (Mrs. Braun opens the door.) \\
Der Briefträger gibt Frau Braun ein Telegramm \textcolor{red}{(telegram)}. (The postman gives Mrs. Braun a telegram.) \\
Sie geht \textcolor{red}{(She goes)} in das Zimmer \textcolor{red}{(room)} zurück \textcolor{red}{(back)}. (She goes back into the room.) \\
Dort \textcolor{red}{(There)} sitzt \textcolor{red}{(sits)} ihr Mann \textcolor{red}{(her husband)} und arbeitet \textcolor{red}{(works)}. (There her husband sits and works.) \\
"Hier ist ein Telegramm, Paul!" sagt \textcolor{red}{(says)} sie. ("Here is a telegram, Paul!" she says.) \\
Herr Braun öffnet das Telegramm und liest \textcolor{red}{(reads)}. (Mr. Braun opens the telegram and reads.) \\
Dann sagt er: "Heute kommt Besuch \textcolor{red}{(visit)}. (Then he says: "Today we have a visitor.) \\
Mein Freund Walter fährt \textcolor{red}{(travels)} nach Hamburg und unterbricht \textcolor{red}{(interrupts)} seine Reise \textcolor{red}{(journey)} hier. (My friend Walter is traveling to Hamburg and interrupts his journey here.) \\
Sein Zug \textcolor{red}{(His train)} kommt schon \textcolor{red}{(already)} um 3.45 Uhr an \textcolor{red}{(arrives)!"} (His train already arrives at 3.45! ") \\
- "Oh, dann kommt er ja bald \textcolor{red}{(soon)}! ( - "Oh, then he comes soon!) \\
Ich koche schnell Kaffee \textcolor{red}{(coffee)}." (I'll quickly make coffee.") \\
Frau Braun geht in die Küche \textcolor{red}{(kitchen)}. (Mrs. Braun goes into the kitchen.) \\
Herr Braun hilft \textcolor{red}{(helps)} seiner Frau und kauft Kuchen \textcolor{red}{(cake)}. (Mr. Braun helps his wife and buys cake.)

Um vier Uhr kommt der Freund. (At four o'clock the friend comes.) \\
Herr und Frau Braun begrüßen \textcolor{red}{(greet)} ihren Gast \textcolor{red}{(guest)} herzlich \textcolor{red}{(warmly)}. (Mr. and Mrs. Braun greet their guest warmly.) \\
Dann führt \textcolor{red}{(leads)} Herr Braun seinen Freund ins Zimmer. (Then Mr. Braun leads his friend into the room.) \\
Seine Frau bietet \textcolor{red}{(offers)} ihrem Gast Kaffee und Kuchen an. (His wife offers her guest coffee and cake.) \\
"Möchtest du eine Zigarette \textcolor{red}{(cigarette)}, Walter?" fragt Herr Braun seinen Freund. ("Would you like a cigarette, Walter?" Mr. Braun asks his friend.) \\
Er aber lehnt ab \textcolor{red}{(declines)}: "Danke, nein! Zigaretten schaden \textcolor{red}{(harm)} meiner Gesundheit \textcolor{red}{(health)}." (But he declines: "Thank you, no! Cigarettes harm my health.")

Walter erzählt \textcolor{red}{(tells)} seinen Gastgebern \textcolor{red}{(hosts)} viel \textcolor{red}{(a lot)}, und die Zeit \textcolor{red}{(time)} vergeht \textcolor{red}{(passes)} schnell \textcolor{red}{(quickly)}. (Walter tells his hosts a lot, and time passes quickly.) \\
Schließlich \textcolor{red}{(Finally)} sagt Walter: "Leider \textcolor{red}{(Unfortunately)} fährt mein Zug schon um 7 Uhr. (Finally Walter says: "Unfortunately my train leaves already at 7 o'clock.) \\
Wo finde \textcolor{red}{(find)} ich hier ein Taxi \textcolor{red}{(taxi)}?" (Where can I find a taxi here?") \\
- "Du braucht \textcolor{red}{(need)} kein \textcolor{red}{(no)} Taxi", antwortet \textcolor{red}{(answers)} Herr Braun, "wir nehmen \textcolor{red}{(take)} unser Auto \textcolor{red}{(car)}. ("You don't need a taxi," answers Mr. Braun, "we'll take our car.) \\
Es gehört \textcolor{red}{(belongs to)} meiner Firma \textcolor{red}{(company)}. (It belongs to my company.) \\
Ich fahre schnell in die Stadt \textcolor{red}{(city)}, und du erreichst \textcolor{red}{(reach)} deinen Zug pünktlich \textcolor{red}{(punctually)}." (I'll drive quickly into the city, and you'll reach your train punctually.")

Frau Braun gibt ihrem Gast die Hand \textcolor{red}{(hand)} und sagt: "Auf Wiedersehen \textcolor{red}{(Goodbye)}, Walter! (Mrs. Braun shakes her guest's hand and says: "Goodbye, Walter!) \\
Hoffentlich \textcolor{red}{(Hopefully)} kommst du bald wieder \textcolor{red}{(again)}! (Hopefully you'll come again soon!)" \\
- "Ich hoffe \textcolor{red}{(hope)} es auch \textcolor{red}{(too)}. ( - "I hope so too.) \\
Auf Wiedersehen!"

\section*{Answer the following questions based on the above text:}
(a) Wer öffnet die Tür? (Who opens the door?)
(b) Wohin fährt Walter? (Where is Walter traveling to?)
(c) Was kauft Herr Braun? (What does Mr. Braun buy?)
(d) Was sagt Walter über Zigaretten? (What does Walter say about cigarettes?)
(e) Wie fährt Walter zum Bahnhof? (How does Walter travel to the train station?)
\section*{Q. 2 Use the correct verb forms (any ten) !}
(a) Morgen (fahren) der Lehrer nach Berlin.
(b) Wie lange (studieren) du Physik in Deutschland?
(c) Ein Kilo Butter (kosten) nur Euro 2, 20.
(d) Frau Schiller (haben) ein schönes Haus in Göttingen.
(e) Heute (besuchen) mein Lehrer das Sprachlabor in Kharagpur.
(f) Peter (geben) dem Gast ein Glas Wasser
(g) Der Schuler (schreiben) die Sätze in das Heft.
(h) Dieses Mädchen (sprechen) nur Englisch.
(i) (Kaufen) ihr ein neues Wörterbuch?
(j) Viele Studenten (arbeiten) nur 4 Stunden,
(k) Nach dem Essen (lesen) mein Bruder die deutsche Zeitung.
\section*{Q. 3 Form sentences with the opposites of the following words (any ten)!}
schwer \textcolor{red}{(difficult)}, richitig \textcolor{red}{(right)}, billig \textcolor{red}{(cheap)}, feundlich \textcolor{red}{(friendly)}, klein \textcolor{red}{(small)}, fleißig \textcolor{red}{(hardworking)}, schön \textcolor{red}{(beautiful)}, früh \textcolor{red}{(early)}, alt \textcolor{red}{(old)}, immer \textcolor{red}{(always)}, dunkel \textcolor{red}{(dark)}, langsam \textcolor{red}{(slow)}
\section*{Q. 4 Form sentences with the following groups of words (any ten) /}
(a) das Kind, ein Buch, kaufen, der Student
(b) Frau Müller, trinken, Kaffee, gern, am Abend
(c) Er, eine Blume, seine Freundin, schenken
(d) zusammen, wir, ein, machen, heute, Experiment
(e) Haus, nachmittags, sechs, um, nach, er, Uhr, gehen
(f) Freund, haben, zwei, mein, Brider
(g) Robert, ein Füller, seine Freundin, geben
(h) wie lange, die Zeitung, der Professor, lesen ?
(i) die Lehrerin, täglich, zehn Stunden, arbeiten
(j) der Student, Frau Kuhn, ein Bild, zeigen
(k) brauchen, Heidi, das Auto, nicht, heute

\section*{(3)}
\section*{Q. 5 Replace the underlined portions by appropriate pronouns (any five sentences)!}
(a) Das Madchen bringt dem Mann eine Tasse Kaffee.
(b) Der Schülerin kauft der Ausländer ein Fahrrad.
(c) Der Briefträger bringt Frau Schiller ein Telegram.
(d) Robert zeigt dem Gast die schnönen Bilder.
(e) Die Studenten schenken dem Professor eine Uhr.
(f) Die Lehrerin erklärt den Studenten die Relativitästheorie.
\section*{Q. 6 Frame questions to which the underlined words provide answers (any ten).}
(a) Morgen fahren wir nach Neu Delhi.
(g) Mein Professor ist sehr nett und frendlich.
(h) Seit 2000 arbeitet er bei Siemens.
(i) Dieses Buch kostet nur Euro 3,--
(j) Mein Vater kommt nach Kharagpur heute abend.
(k) Das Zimmer ist sehr teuer aber komfortabel.
(l) Der Zug fährt um 9 Uhr von Kharagpur ab,
(b) Diese Studenten kommen aus Singapore.
(c) Es ist schon 11.00 Uhr.
(d) Der Amerikaner wohnt in Benaras.
(e) Diese Studenten arbeiten 10 Stunden täglich.
\section*{Q. 7 Write twenty sentences in German on 'Mein/e Freund/Freundin' or 'Mein Institut' !}
\end{document}
