\documentclass{article}
\usepackage[utf8]{inputenc}
\usepackage{amsmath}
\usepackage{amsfonts}
\usepackage{amssymb}
\usepackage{geometry}
\usepackage{textcomp} % Sometimes needed for additional symbols
\usepackage{xcolor}
\geometry{a4paper, margin=1in}

\definecolor{darkgreen}{rgb}{0.0, 0.5, 0.0}

\title{INDIAN INSTITUTE OF TECHNOLOGY KHARAGPUR \\ Mid-Spring Semester Examination 2023-24}
\date{}

\begin{document}
\maketitle
\begin{center}
\begin{tabular}{ll}
Date of Examination: & Session (FN/AN) Duration: 2 hrs Full Marks: 60/2 \\
Subject No. : HS30048 & Department: Humanities \& Social Sciences
\end{tabular}
\end{center}

\noindent \textbf{Instructions:} Answer all questions.

\begin{enumerate}
    \item Read the following passage carefully and answer the questions that follow:

    Für das Wochenende und die Ferien (\textcolor{red}{holidays}) mache ich gern Pläne. [I like making plans for the weekend and the holidays.]

    An den freien Samstagen und Sonntagen werde ich lange schlafen. [On free Saturdays and Sundays, I will sleep in for a long time.]
    
    Dann klingelt (\textcolor{red}{ring}) der Wecker nicht. [Then the alarm clock will not ring.]
    
    Aber ich werde für die Wochenenden nicht zu viel planen, weil ich gern faul bin und nichts tue. [But I will not plan too much for the weekends because I like being lazy and doing nothing.]
    
    Aber ich werde vielleicht (maybe) zum Sport gehen. [But maybe I will go to do some sports.]
    
    Manchmal habe ich am Wochenende ein Turnier. [Sometimes I have a tournament on the weekend.]
    
    Diesen Sonntag zum Beispiel werde ich mit meinem Team in eine andere Stadt fahren. [This Sunday, for example, I will travel to another city with my team.]
    
    Wir werden dort ein Match gegen einen anderen Hockeyverein spielen. [We will play a match against another hockey club there.]
    
    Das wird bestimmt ein Spaß. [That will definitely be fun.]
    
    Wenn das Wetter schön ist, werde ich anschließend mit meinen Freunden schwimmen gehen. [If the weather is nice, I will go swimming with my friends afterward.]
    
    In der Nähe gibt es einen See, der wird schon warm genug sein. [There is a lake nearby that will be warm enough.]
    
    Wenn ich länger frei habe, mache ich gerne größere Pläne. [When I have a longer break, I like making bigger plans.]
    
    In den Sommerferien werde ich sehr oft mit meinen Freunden unterwegs sein. [During the summer holidays, I will be out and about with my friends very often.]
    
    Wir werden zum See fahren. [We will go to the lake.]
    
    Dort werden wir im Zelt übernachten und beim Lagerfeuer sitzen. [There we will sleep in a tent and sit by the campfire.]
    
    Eine oder zwei Wochen möchte ich gerne reisen. [I would like to travel for one or two weeks.]
    
    Ein Freund wird mich auf der Reise begleiten, wir werden mit dem Zug losfahren. [A friend will accompany me on the trip, we will leave by train.]
    
    Wir planen eine Route durch das ganze Land, von West bis Ost und von Süd bis Nord. [We are planning a route through the whole country, from west to east and from south to north.]
    
    Mit Rucksäcken und Wanderschuhen (\textcolor{red}{hiking boots}) werden wir auch in die Berge fahren. [With backpacks and hiking boots, we will also go to the mountains.]
    
    Am liebsten würde ich dort in einer Hütte übernachten. [I would prefer to stay overnight in a cabin there.]
    
    Wir werden sehen, ob wir das auch schaffen werden. [We will see if we can manage that.]
    
    Ein Abenteuer wird es aber ganz bestimmt. [But it will definitely be an adventure.]

    \textbf{Answer the following questions based on the above text:}
    \begin{enumerate}
        \item Was macht der Erzähler (narrater) an den freien Samstagen und Sonntagen?
        \item Was macht er wenn das Wetter schön ist?
        \item Was plant er mit den Freunden im Sommer?
        \item Wo werden sie übernachten und sitzen?
        \item Wie werden sie in die Berge fahren?
    \end{enumerate}

    \item Write the following numbers in words:

    33, 17, 789, 6750, 19868
    \begin{enumerate}
        \item \textcolor{darkgreen}{Dreiunddreißig}
        \item \textcolor{darkgreen}{Siebzehn}
        \item \textcolor{darkgreen}{Siebenhundertneunundachtzig}
        \item \textcolor{darkgreen}{Sechstausendsiebenhundertfünfzig}
        \item \textcolor{darkgreen}{Neunzehntausendachthundertachtundsechzig}
    \end{enumerate}

    \item Give the comparative and superlative forms of the following adjectives/adverbs:

    gut, fruh, dunkel, fleißig, leicht, klein, schnell, billig, schön, nie, freundlich, jung

    \item Replace the underlined portions by appropriate pronouns (any five sentences):
    \begin{enumerate}
        \item \underline{Der Student} bringt \underline{dem Kind} einen neuen Füller. \textcolor{darkgreen}{Er bringt ihm einen neuen Füller.}
        \item Die Lehrerin kauft \underline{den Studenten} \underline{viele neue Bücher}. \textcolor{darkgreen}{Die Lehrerin kauft ihnen sie}
        \item Robert schenkt \underline{seiner Freundin} \underline{eine neue Uhr}. \textcolor{darkgreen}{Robert schenkt ihr sie.}
        \item Frau Schmidt zeigt \underline{ihrem Mann} \underline{die schönen Bilder}. \textcolor{darkgreen}{Frau Schmidt zeigt ihm sie.}
        \item \underline{Die Schülerin} hat \underline{ein neues Fahrrad}. \textcolor{darkgreen}{Sie hat es.}
        \item \underline{Die alte Dame} kauft \underline{ein neues Haus} in Berlin. \textcolor{darkgreen}{Sie kauft es in Berlin.}
    \end{enumerate}

    \item Use the correct form of the verbs given in the brackets (any ten):
    \begin{enumerate}
        \item Der Kaufmann \textit{(kaufen)} dem Kind ein neues Buch. \textcolor{darkgreen}{kauft}
        \item Robert \textit{(sein)} ein Student und studiert in Frankfurt. \textcolor{darkgreen}{ist}
        \item Meine Freundin \textit{(haben)} eine Schwester. \textcolor{darkgreen}{hat}
        \item Der Amerikaner \textit{(trinken)} Tee ohne Milch und Zucker. \textcolor{darkgreen}{trinkt}
        \item Der Ausländer \textit{(lesen)} die Texte sehr schnell. \textcolor{darkgreen}{liest}
        \item Der Professor \textit{(sprechen)} Deutsch sehr langsam. \textcolor{darkgreen}{spricht}
        \item Die Hausfrau \textit{(geben)} dem Gast ein Glas Wasser. \textcolor{darkgreen}{gibt}
        \item Er \textit{(gehen)} mit Freunden ins Kino am Samstag. \textcolor{darkgreen}{geht}
        \item Die Ärtzin \textit{(arbeiten)} zehn Stunden täglich. \textcolor{darkgreen}{arbeitet}
        \item Der Lehrer \textit{(fahren)} morgen nach Frankreich. \textcolor{darkgreen}{fährt}
        \item Christina \textit{(erklären)} den Studenten die Bilder. \textcolor{darkgreen}{erklärt}
    \end{enumerate}

    \item Frame questions to which the underlined words provide answers (any five)!
    \begin{enumerate}
        \item Die Lehrerin ist \underline{sehr nett und freundlich}. \textcolor{darkgreen}{Wie ist die Lehrerin?}
        \item Diese Studenten kommen \underline{aus Deutschland}. \textcolor{darkgreen}{Woher kommen diese Studenten?}
        \item Ein Liter Milch kostet \underline{nur Euro 1,30}. \textcolor{darkgreen}{Wie viel kostet ein Liter Milch?}
        \item Der Professor fährt \underline{nach Japan} morgen. \textcolor{darkgreen}{Wohin fährt der Professor morgen?}
        \item Der Zug fährt \underline{um 10:00 Uhr} von Kharagpur ab. \textcolor{darkgreen}{Wann fährt der Zug von Kharagpur ab?}
        \item Es ist \underline{schon 13:00 Uhr}. \textcolor{darkgreen}{Wie spät ist es?}
        \item Der Inder wohnt jetzt \underline{in Berlin}. \textcolor{darkgreen}{Wo wohnt der Inder jetzt?}
    \end{enumerate}

    \item Form sentences with the following groups of words (any five)!
    \begin{enumerate}
        \item morgen, fahren, nach, der Herr, Frankfurt \textcolor{darkgreen}{Der Herr fährt nach Frankfurt morgen.}
        \item Herr Schmitt, alt, 50 Jahre, sein, und, in, Berlin, arbeiten \textcolor{darkgreen}{Herr Schmitt ist 50 Jahre alt und arbeitet in Berlin.}
        \item der Ausländer, bringen, das Mädchen, ein Glas Wasser \textcolor{darkgreen}{Der Ausländer bringt dem Mädchen ein Glas Wasser.}
        \item hier, die Leute, trinken, und, Tee, gern, ohne Milch, Zucker \textcolor{darkgreen}{Die Leute trinken hier gern Tee ohne Milch und Zucker.}
        \item der Professor, die Wörter, die Studenten, erklären \textcolor{darkgreen}{Der Professor erklärt den Studenten die Wörter.}
        \item die Studentin, sieben, arbeiten, in der Nacht, Stunden \textcolor{darkgreen}{Die Studentin arbeitet sieben Stunden in der Nacht.}
    \end{enumerate}
\end{enumerate}

\end{document}
