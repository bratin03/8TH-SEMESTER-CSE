\documentclass{article}
\usepackage[utf8]{inputenc}
\usepackage{xcolor}

\definecolor{darkgreen}{rgb}{0.0, 0.5, 0.0}

\title{}
\author{}
\date{}

\begin{document}

\maketitle

\noindent
\textbf{Date of Examination:} Session (FN/AN) \textbf{Duration:} 2 hrs \textbf{Subject No.:} HS51604 \\
\textbf{Department:} Humanities \& Social Sciences \\
\textbf{Full Marks:} 60/2 \\
\textbf{Subject Name:} German

\begin{enumerate}
    \item Read the following passages carefully and answer the questions that follow:

    Es gibt \textcolor{red}{(There are)} sehr viele verschiedene \textcolor{red}{(different)} Berufe \textcolor{red}{(professions)}. (There are very many different professions.) \\
    Ein Lehrer \textcolor{red}{(teacher)} unterrichtet \textcolor{red}{(teaches)} Schüler \textcolor{red}{(students)} und bringt \textcolor{red}{(brings)} ihnen \textcolor{red}{(them)} verschiedene \textcolor{red}{(different)} Dinge \textcolor{red}{(things)} bei \textcolor{red}{(with)}. (A teacher teaches students and teaches them different things.) \\
    In einer Volksschule \textcolor{red}{(elementary school)} lehren \textcolor{red}{(teach)} Lehrer \textcolor{red}{(teachers)} den Kindern \textcolor{red}{(the children)} das Lesen \textcolor{red}{(reading)} und Schreiben \textcolor{red}{(writing)}. (In an elementary school, teachers teach children reading and writing.) \\
    Lehrer arbeiten \textcolor{red}{(work)} auch mit älteren \textcolor{red}{(older)} Kindern und Jugendlichen \textcolor{red}{(young people)}. (Teachers also work with older children and young people.) \\
    Dort \textcolor{red}{(there)} unterrichten \textcolor{red}{(teach)} sie meistens \textcolor{red}{(mostly)} ein bestimmtes \textcolor{red}{(a specific)} Fach \textcolor{red}{(subject)}, Mathematik \textcolor{red}{(mathematics)} oder Sprachen \textcolor{red}{(languages)} zum Beispiel \textcolor{red}{(for example)}. (There they usually teach a specific subject, mathematics or languages for example.) \\
    Ein anderer \textcolor{red}{(another)} Beruf \textcolor{red}{(profession)}, bei dem \textcolor{red}{(in which)} man auf der Universität \textcolor{red}{(at the university)} studieren \textcolor{red}{(study)} muss \textcolor{red}{(must)}: Arzt \textcolor{red}{(Doctor)}. (Another profession where you have to study at the university: Doctor.) \\
    Ein Arzt behandelt \textcolor{red}{(treats)} kranke \textcolor{red}{(sick)} Leute \textcolor{red}{(people)} in einer Praxis \textcolor{red}{(practice)} oder im Krankenhaus \textcolor{red}{(in the hospital)}. (A doctor treats sick people in a practice or in a hospital.) \\
    Er untersucht \textcolor{red}{(examines)} die Menschen \textcolor{red}{(the people)} und stellt fest \textcolor{red}{(determines)}, was ihnen fehlt \textcolor{red}{(what is wrong with them)}. (He examines the people and determines what is wrong with them.) \\
    Er verschreibt \textcolor{red}{(prescribes)} Medikamente \textcolor{red}{(medication)} oder andere \textcolor{red}{(other)} Behandlungen \textcolor{red}{(treatments)}. (He prescribes medication or other treatments.) \\
    Es gibt viele verschiedene Ärzte \textcolor{red}{(doctors)}, manche \textcolor{red}{(some)} sind Chirurgen \textcolor{red}{(surgeons)}, andere \textcolor{red}{(others)} sind Ohrenärzte \textcolor{red}{(ENT doctors)} oder Zahnärzte \textcolor{red}{(dentists)}. (There are many different doctors, some are surgeons, others are ENT doctors or dentists.)

    Handwerker \textcolor{red}{(craftsmen)} lernen \textcolor{red}{(learn)} ihren Beruf \textcolor{red}{(their profession)} in einer Berufsausbildung \textcolor{red}{(vocational training)}. (Craftsmen learn their profession in vocational training.) \\
    Ein Handwerker ist zum Beispiel ein Bäcker \textcolor{red}{(Baker)}. (A craftsman is, for example, a baker.) \\
    Er stellt Brot her \textcolor{red}{(He makes bread)}. (He makes bread.) \\
    Dazu \textcolor{red}{(for this)} muss er wissen \textcolor{red}{(he must know)}, wie man Mehl \textcolor{red}{(flour)}, Salz \textcolor{red}{(salt)}, Hefe \textcolor{red}{(yeast)} und andere \textcolor{red}{(other)} Zutaten \textcolor{red}{(ingredients)} mischt \textcolor{red}{(mixes)}. (For this he must know how to mix flour, salt, yeast and other ingredients.) \\
    Bäcker können \textcolor{red}{(can)} sehr viele verschiedene Sorten \textcolor{red}{(kinds)} Brot machen, auch süßes \textcolor{red}{(sweet)} Gebäck \textcolor{red}{(pastries)}. (Bakers can make many different kinds of bread, including sweet pastries.) \\
    Ein Bauer \textcolor{red}{(Farmer)} arbeitet auch mit Lebensmitteln \textcolor{red}{(food)}. (A farmer also works with food.) \\
    Auf seinem Hof \textcolor{red}{(on his farm)} hält er Tiere \textcolor{red}{(animals)} wie Kühe \textcolor{red}{(cows)}, Hühner \textcolor{red}{(chickens)} oder Schweine \textcolor{red}{(pigs)}. (On his farm he keeps animals such as cows, chickens or pigs.) \\
    Auf den Feldern \textcolor{red}{(in the fields)} pflanzt \textcolor{red}{(plants)} er Getreidesorten \textcolor{red}{(grains)} oder Gemüse \textcolor{red}{(vegetables)} und Obst \textcolor{red}{(fruits)}. (In the fields he plants grains or vegetables and fruits.) \\
    Das liefert \textcolor{red}{(delivers)} er an Supermärkte \textcolor{red}{(supermarkets)} oder verkauft \textcolor{red}{(sells)} es selbst \textcolor{red}{(himself)} auf einem Markt \textcolor{red}{(at a market)}. (He delivers this to supermarkets or sells it himself at a market.) \\
    Ein Koch \textcolor{red}{(cook)} arbeitet auch mit Lebensmitteln \textcolor{red}{(food)}. (A cook also works with food.) \\
    In einem Restaurant \textcolor{red}{(restaurant)} bereitet \textcolor{red}{(prepares)} er die Speisen \textcolor{red}{(dishes)} zu \textcolor{red}{(to)}. (In a restaurant, he prepares the dishes.) \\
    Seine Ausbildung \textcolor{red}{(his training)} macht \textcolor{red}{(does)} er entweder \textcolor{red}{(either)} in einer Schule \textcolor{red}{(school)} oder in einer Lehre \textcolor{red}{(apprenticeship)}. (He does his training either in a school or in an apprenticeship.) \\
    Ein Verkäufer \textcolor{red}{(Salesman)} arbeitet in einem Laden \textcolor{red}{(shop)}. (A salesman works in a shop.) \\
    Dort \textcolor{red}{(there)} verkauft \textcolor{red}{(sells)} er an die Kunden \textcolor{red}{(to the customers)}, was der Laden bietet \textcolor{red}{(what the shop offers)}: Das können Lebensmittel sein, aber auch Kleidung \textcolor{red}{(clothing)} oder Autos \textcolor{red}{(cars)}. (There he sells to the customers what the shop offers: These can be food, but also clothing or cars.)

    Answer the following questions based on the above texts:
    \begin{enumerate}
        \item Was lehren die Lehrer in einer Volksschule?
        \item Wo behandelt ein Arzt kranke Lente?
        \item Wo lernen die Handwerker ihren Beruf?
        \item Was pflanzt ein Bauer auf den Feldern?
        \item Wo arbeitet ein Verkäufer?
    \end{enumerate}

    \item Form sentences with the opposites of the following words (any ten):
    schwer, klein, schnell, billig, schön, gut, spät, hell, fleißig, immer, freundlich, alt
    \begin{enumerate}
        \item schwer - \textcolor{darkgreen}{Das Koffer ist leicht.}
        \item klein - \textcolor{darkgreen}{Mein Zimmer ist groß.}
        \item schnell - \textcolor{darkgreen}{Ich spreche langsam.}
        \item billig - \textcolor{darkgreen}{Das Auto ist teuer.}
        \item schön - \textcolor{darkgreen}{Sie ist hässlich.}
        \item gut - \textcolor{darkgreen}{Der Lehrer ist schlecht.}
        \item spät - \textcolor{darkgreen}{Ich komme früh.}
        \item hell - \textcolor{darkgreen}{Das Zimmer ist dunkel.}
        \item fleißig - \textcolor{darkgreen}{Ich bin faul.}
        \item immer - \textcolor{darkgreen}{Ich komme nie spät.}
        \item freundlich - \textcolor{darkgreen}{Er ist unfreundlich.}
        \item alt - \textcolor{darkgreen}{Das Auto ist neu.}
    \end{enumerate}


    \item Write the following numbers in words:
    \(17,36,95,817,6960\)
    \begin{enumerate}
        \item 17 - \textcolor{darkgreen}{Siebzehn}
        \item 36 - \textcolor{darkgreen}{Sechsunddreißig}
        \item 95 - \textcolor{darkgreen}{Fünfundneunzig}
        \item 817 - \textcolor{darkgreen}{Achthundertsiebzehn}
        \item 6960 - \textcolor{darkgreen}{Sechstausendneunhundertsechzig}
    \end{enumerate}

    \item Use the correct forms of the verbs given in the brackets (any ten):
    \begin{enumerate}
        \item Der Mann (kaufen) der Frau eine neue Uhr. \textcolor{darkgreen}{kauft}
        \item Maria (sein) eine Studentin und studiert in Paris. \textcolor{darkgreen}{ist}
        \item Mein Freund (haben) einen Bruder und eine Schwester. \textcolor{darkgreen}{hat}
        \item Der Kaufmann (trinken) Tee ohne Milch und Zucker. \textcolor{darkgreen}{trinkt}
        \item Der Ausläder (lesen) die Wörter sehr Jangsam. \textcolor{darkgreen}{liest}
        \item Der Amerikaner (sprechen) Hindi langsam. \textcolor{darkgreen}{spricht}
        \item Das Mädchen (geben) ihm ein Glas Wasser. \textcolor{darkgreen}{gibt}
        \item Nach dem Unterricht (gehen) die Studenten zum Theater. \textcolor{darkgreen}{gehen}
        \item Der Professor (arbeiten) acht Stunden täglich. \textcolor{darkgreen}{arbeitet}
        \item Der Amerikaner (fahren) morgen nach Indien. \textcolor{darkgreen}{fährt}
        \item Die Lehrerin (erklären) den Studenten die Sätze. \textcolor{darkgreen}{erklärt}
    \end{enumerate}
    \item Replace the underlined portions by appropriate pronouns (any five sentences):
    \begin{enumerate}
        \item \underline{Mein Bruder} hat \underline{ein neues Fahrrad}. \textcolor{darkgreen}{Er hat es.}
        \item Der Student schenkt \underline{der Lehrerin} \underline{einen neuen Füller}. \textcolor{darkgreen}{Der Student schenkt ihr ihn.}
        \item \underline{Robert} zeigt seinem Freund \underline{sein neues Haus}. \textcolor{darkgreen}{Er zeigt seinem Freund es.}
        \item \underline{Die Dame} gibt \underline{den Studenten} die neue Kugelschreiber. \textcolor{darkgreen}{Sie gibt ihnen die neuen Kugelschreiber.}
        \item Die Lehrerin kauft \underline{ihrem Mann} \underline{viele neue Bücher}. \textcolor{darkgreen}{Die Lehrerin kauft ihm sie.}
        \item Frau Schmid bringt \underline{ihrer Freundin} \underline{eine schöne Blume}. \textcolor{darkgreen}{Frau Schmid bringt ihr sie.}
    \end{enumerate}
    \item Form sentences with the following groups of words (any five)!
    \begin{enumerate}
        \item heute, fahren, nach, der Professor, Berlin \textcolor{darkgreen}{Der Professor fährt nach Berlin heute.}
        \item Herr Robert, alt, 50, Jahre, sein, und. in, Paris, arbeiten \textcolor{darkgreen}{Herr Robert ist 50 Jahre alt und arbeitet in Paris.}
        \item die Lehrerin, die Wörter, die Studenten, erklären \textcolor{darkgreen}{Die Lehrerin erklärt den Studenten die Wörter.}
        \item die Arbeiter, zehn, arbeiten, täglich, Stunden \textcolor{darkgreen}{Die Arbeiter arbeiten zehn Stunden täglich.}
        \item der Ausländer, bringen, das Mädchen, ein Glas Wasser \textcolor{darkgreen}{Der Ausländer bringt dem Mädchen ein Glas Wasser.}
        \item Hier, die Leute, trinken, und, Tee, gern, mit Milch, Zucker \textcolor{darkgreen}{Die Leute trinken hier gern Tee mit Milch und Zucker.}
    \end{enumerate}
    \item Frame questions to which the underlined words provide answers (any five)!
    \begin{enumerate} 
        \item Der Inder wohnt jetzt \underline{in London}. \textcolor{darkgreen}{Wo wohnt der Inder jetzt?}
        \item Die Lehrerin ist sehr \underline{nett und freundlich}. \textcolor{darkgreen}{Wie ist die Lehrerin?}
        \item Diese Studenten Kommen \underline{aus Deutschland}. \textcolor{darkgreen}{Woher kommen diese Studenten?}
        \item Der Zug fährt \underline{um 10 Uhr} von Howrah ab. \textcolor{darkgreen}{Wann fährt der Zug von Howrah ab?}
        \item Es ist schon \underline{13.00 Uhr}. \textcolor{darkgreen}{Wie spät ist es?}
        \item Das Zimmer kostet nur \underline{Euro i5.30}. \textcolor{darkgreen}{Wie viel kostet das Zimmer?}
        \item Die Studenten fahren \underline{nach Hamburg} morgen. \textcolor{darkgreen}{Wohin fahren die Studenten morgen?}
    \end{enumerate}
\end{enumerate}

\end{document}
