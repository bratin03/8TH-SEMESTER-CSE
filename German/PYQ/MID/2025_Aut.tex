\documentclass{article}
\usepackage[T1]{fontenc}
\usepackage[utf8]{inputenc}
\usepackage{amsmath}
\usepackage{geometry}
\usepackage{textcomp} % Sometimes needed for additional symbols
\usepackage{xcolor}
\geometry{a4paper, margin=1in}

\definecolor{darkgreen}{rgb}{0.0, 0.5, 0.0}

\title{
INDIAN INSTITUTE OF TECHNOLOGY KHARAGPUR \\
Mid-Autumn Semester Examination 2024-25
}
\author{}
\date{}

\begin{document}
\maketitle
\begin{center}
Date of Examination: \hspace{0.5cm} Session (FN/AN) \hspace{0.5cm} Duration: 2 hrs \hspace{0.5cm} Subject No.: HS 51604 \\
Department: Humanities \& Social Sciences
\end{center}

\vspace{0.5cm}
\begin{center}
Full Marks: 60/2 \\
Subject: German
\end{center}

\section*{1. Read the following passages carefully and answer the questions that follow:}

Silvia ist eine Frau, die gerne wandert (\textcolor{red}{hiking}) (Silvia is a woman who likes to hike). \\
Sie lebt im Harz in Niedersachsen. (She lives in the Harz in Lower Saxony.) \\
Das ist eine Region in Norddeutschland. (That is a region in Northern Germany.) \\
Hier sind die Temperaturen mild und es regnet oft. (Here, the temperatures are mild, and it rains often.) \\
Die Sommer sind nicht sehr hei\ss. (The summers are not very hot.) \\

Fast (\textcolor{red}{almost}) jedes (\textcolor{red}{every}) Wochenende nimmt Silvia ihren Rucksack (\textcolor{red}{backpack}) und ihre Wasserflasche und läuft zum (\textcolor{red}{for the}) Wolfskopf. (Almost every weekend, Silvia takes her backpack and water bottle and walks to Wolfskopf.) \\
Der Wolfskopf ist ein bekannter (\textcolor{red}{famous}) Berg (mountain) im Harz. (The Wolfskopf is a well-known mountain in the Harz.) \\
Er ist fast 700 Meter hoch. (It is almost 700 meters high.) \\

Der Wolfskopf ist bekannt (\textcolor{red}{famous}) für unterschiedliche (\textcolor{red}{different}) Freizeitaktivitäten (\textcolor{red}{Leisure activities}). (The Wolfskopf is famous for various leisure activities.) \\
Wandern, Joggen und Mountainbiken sind besonders (\textcolor{red}{particularly}) beliebt (\textcolor{red}{popular}). (Hiking, jogging, and mountain biking are especially popular.) \\

Auch am letzte (\textcolor{red}{last}) Samstag wollte (\textcolor{red}{wanted}) Silvia wieder (\textcolor{red}{again}) wandern. (Last Saturday, Silvia wanted to hike again.) \\
Aber diesmal (\textcolor{red}{this time}) war es anders (\textcolor{red}{different}) als (\textcolor{red}{than}) sonst (\textcolor{red}{usual}). (But this time, it was different than usual.) \\
Dabei fing alles ganz normal an. (However, everything started quite normally.) \\
Silvia traf (\textcolor{red}{met}) ihren Freund Jochen am Anfang (\textcolor{red}{beginning}) des Wanderweges (\textcolor{red}{hiking trail}). (Silvia met her friend Jochen at the beginning of the hiking trail.) \\
Da Jochen ebenfalls (\textcolor{red}{also}) gern wandert, kam er mit auf den Ausflug (excursion). (Since Jochen also likes to hike, he joined the excursion.) \\

„Silvia, schön, dich (\textcolor{red}{you}) wieder (\textcolor{red}{again}) zu sehen!" ("Silvia, nice to see you again!") \\
„Hallo Jochen, ich freue (\textcolor{red}{happy}) mich (\textcolor{red}{myself}) auch!" ("Hello Jochen, I am happy too!") \\
Sie machten (\textcolor{red}{made}) sich (\textcolor{red}{themselves})sofort (\textcolor{red}{immediately}) auf den Weg. (They immediately set off on their way.) \\

„Jochen, geh nicht so schnell. ("Jochen, don't walk so fast.") \\
Sonst geht uns gleich (\textcolor{red}{otherwise}) die Puste (steam) aus (\textcolor{red}{out of})." ("Otherwise, we'll run out of breath soon.") \\

„Keine Sorge (worries), ich habe doch ein Energiegetränk dabei." ("No worries, I have an energy drink with me.") \\

„Jochen, welchen Weg nehmen wir? Den rechten oder den linken?" ("Jochen, which path should we take? The right one or the left one?") \\
„Hm, lass uns den linken Weg nehmen." ("Hm, let's take the left path.") \\
„Aber ich finde den rechten Weg besser." ("But I think the right path is better.") \\
„Warum denn, Silvia?" ("Why, Silvia?") \\

„Weil man sagt, dass dort oft ein großes Wesen (creature) gesehen wurde." ("Because they say a large creature has often been seen there.") \\
„Glaubst du, dass da etwas dran ist?" ("Do you believe there's something to it?") \\
„Das können wir herausfinden, wenn wir den Weg nehmen." ("We can find out if we take that path.") \\

„Na gut, Silvia. Dann also da entlang." ("Alright, Silvia. Then let's go that way.") \\
Eine Stunde später gingen sie immer noch (\textcolor{red}{still}) auf dem Weg. (An hour later, they were still on the path.) \\
Es war bereits (\textcolor{red}{already}) Nachmittag (\textcolor{red}{afternoon}). (It was already afternoon.) \\
Silvia fragte Jochen: (Silvia asked Jochen:) \\

„Meinst du, dass es ungewöhnliche (strange) Wesen in den Wäldern (\textcolor{red}{forests}) gibt?" ("Do you think there are strange creatures in the forests?") \\
„Nein, das glaube ich nicht." ("No, I don't think so.") \\
„Warum nicht?" ("Why not?") \\
„Ich habe noch nie solche Wesen gesehen. Du vielleicht?" ("I've never seen such creatures. Have you?") \\
„Nicht in diesem Wald (forest)." ("Not in this forest.") \\

„Was Silvia wohl damit meint?" dachte Jochen. ("What does Silvia mean by that?" thought Jochen.) \\

Answer the following questions based on the above text:
\begin{enumerate}
    \item[(i)] Wo lebt Silvia?
    \item[(ii)] Wohin läuft Silvia fast jedes Wochenende?
    \item[(iii)] Wofür ist der Wolfskopf bekannt?
    \item[(iv)] Wer ist Jochen?
    \item[(v)] Warum findet Silvia den rechten Weg besser?
\end{enumerate}

\section*{2. Write the following numbers in words:}
16, 39, 770, 6855, 17835
\begin{enumerate}
    \item[(i)] \textcolor{darkgreen}{sechzehn}
    \item[(ii)] \textcolor{darkgreen}{neununddreißig}
    \item[(iii)] \textcolor{darkgreen}{siebenhundertsiebzig}
    \item[(iv)] \textcolor{darkgreen}{sechstausendachthundertfünfundfünfzig}
    \item[(v)] \textcolor{darkgreen}{siebzehntausendachthundertfünfunddreißig}
\end{enumerate}

\section*{3. Form sentences with the opposites of the following words (any ten):}
langsam, billig, schön, immer, freundlich, alt, gut, früh, dunkel, fleißig, leicht, groß
\begin{enumerate}
    \item[(i)] \textcolor{darkgreen}{Ich spreche schnell.}
    \item[(ii)] \textcolor{darkgreen}{Das Auto ist teuer.}
    \item[(iii)] \textcolor{darkgreen}{Sie ist hässlich.}
    \item[(iv)] \textcolor{darkgreen}{Ich komme nie spät.}
    \item[(v)] \textcolor{darkgreen}{Der Lehrer ist unfreundlich.}
    \item[(vi)] \textcolor{darkgreen}{Das Auto ist neu.}
    \item[(vii)] \textcolor{darkgreen}{Der Lehrer ist sehr schlecht.}
    \item[(viii)] \textcolor{darkgreen}{Ich komme nie spät.}
    \item[(ix)] \textcolor{darkgreen}{Das Zimmer ist hell.}
    \item[(x)] \textcolor{darkgreen}{Ich bin faul.}
    \item[(xi)] \textcolor{darkgreen}{Der Koffer ist schwer.}
    \item[(xii)] \textcolor{darkgreen}{Mein Zimmer ist klein.}
\end{enumerate}

\section*{4. Replace the underlined portions by appropriate pronouns (any five sentences):}
\begin{enumerate}
    \item[(a)] Frau Silvia zeigt \underline{ihrem Freund} \underline{ihre neue Uhr}. \textcolor{darkgreen}{Frau Silvia zeigt \underline{ihm} \underline{sie}.}
    \item[(b)] \underline{Die Lehrerin} kauft \underline{ein neues Auto}. \textcolor{darkgreen}{\underline{Sie} kauft \underline{es}.}
    \item[(c)] \underline{Die schöne Dame} schenkt \underline{Johannes} ein neues Fahrrad. \textcolor{darkgreen}{\underline{Sie} schenkt \underline{ihm} \underline{es}.}
    \item[(d)] \underline{Der Student} bringt \underline{der Lehrerin} viele neue Bücher. \textcolor{darkgreen}{\underline{Er} bringt \underline{ihr} \underline{sie}.}
    \item[(e)] Der Professor erklärt \underline{den Studenten} \underline{die neue Theorie}. \textcolor{darkgreen}{\underline{Er} erklärt \underline{ihnen} \underline{sie}.}
    \item[(f)] \underline{Mein Vater} hat \underline{ein großes Haus} in Neu-Delhi. \textcolor{darkgreen}{\underline{Er} hat \underline{es} in Neu-Delhi.}
\end{enumerate}

\section*{5. Use the correct form of the verbs given in the brackets (any ten):}
\begin{enumerate}
    \item[(a)] Der Gast (schreiben) die Texte sehr schnell. \textcolor{darkgreen}{schreibt}
    \item[(b)] Die Studentin (sprechen) Englisch sehr schnell. \textcolor{darkgreen}{spricht}
    \item[(c)] Das Mädchen (geben) dem Gast die Zeitung. \textcolor{darkgreen}{gibt}
    \item[(d)] Peter (gehen) mit Freunden ins Theater am Sonntag. \textcolor{darkgreen}{geht}
    \item[(e)] Die Dame (schenken) den Kindern viele neue Bücher. \textcolor{darkgreen}{schenkt}
    \item[(f)] Herr Schmid (sein) ein Student und studiert in Stuttgart. \textcolor{darkgreen}{ist}
    \item[(g)] Mein Freund (haben) einen Bruder. \textcolor{darkgreen}{hat}
    \item[(h)] Der Inder (trinken) Tee mit Milch und Zucker. \textcolor{darkgreen}{trinkt}
    \item[(i)] Die Ärztin (arbeiten) acht Stunden täglich. \textcolor{darkgreen}{arbeitet}
    \item[(j)] Der Professor (fahren) morgen nach Deutschland. \textcolor{darkgreen}{fährt}
    \item[(k)] Martina (besuchen) Familie Kunz am Wochenende. \textcolor{darkgreen}{besucht}
\end{enumerate}

\section*{6. Frame questions to which the underlined words provide answers (any five)!}
\begin{enumerate}
    \item[(a)] Ein Kilo Apfel kostet nur \underline{Euro 3.30}. \textcolor{darkgreen}{Wie viel kostet ein Kilo Apfel?}
    \item[(b)] Der Amerikaner fährt nach Berlin \underline{morgen}. \textcolor{darkgreen}{Wann fährt der Amerikaner nach Berlin?}
    \item[(c)] Der Bus kommt \underline{um 13:00 Uhr} von Kharagpur. \textcolor{darkgreen}{Wann kommt der Bus von Kharagpur?}
    \item[(d)] Die Sekretärin ist \underline{sehr nett und freundlich}. \textcolor{darkgreen}{Wie ist die Sekretärin?}
    \item[(e)] Diese Studenten kommen aus \underline{Japan}. \textcolor{darkgreen}{Woher kommen diese Studenten?}
    \item[(f)] Es ist schon \underline{14:00 Uhr}. \textcolor{darkgreen}{Wie spät ist es?}
    \item[(g)] Diese Studenten wohnen jetzt in \underline{Frankfurt}. \textcolor{darkgreen}{Wo wohnen diese Studenten jetzt?}
\end{enumerate}

\section*{7. Form sentences with the following groups of words (any five)!}
\begin{enumerate}
    \item[(a)] der Gast, bringen, das Mädchen, eine Tasse Kaffee \textcolor{darkgreen}{Das Mädchen bringt dem Gast eine Tasse Kaffee.}
    \item[(b)] hier, die Leute, singen, tanzen, am Abend, und -> \textcolor{red}{Die Leute \textbf{singen und tanzen} hier am Abend.}
    \item[(c)] der Professor, die Sätze, die Studenten, erklären -> \textcolor{darkgreen}{Der Professor erklärt den Studenten die Sätze.}
    \item[(d)] morgen, fahren, nach, der Gast, Kalkutta -> \textcolor{red}{Der Gast fährt nach Kalkutta morgen}
    \item[(e)] die Frau, der Mann, eine neue Uhr, schenken, am Sonntag \textcolor{darkgreen}{Der Mann schenkt der Frau eine neue Uhr am Sonntag.}
    \item[(f)] die Ärztin, acht, arbeiten, in der Nacht, Stunden \textcolor{darkgreen}{Die Ärztin arbeitet acht Stunden in der Nacht.}
\end{enumerate}

\end{document}
