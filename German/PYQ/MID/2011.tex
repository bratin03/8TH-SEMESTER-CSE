\documentclass{article}
\usepackage[utf8]{inputenc}
\usepackage{xcolor}

\title{
INDIAN INSTITUTE OF TECHNOLOGY KHARAGPUR
}

\date{}

\begin{document}

\maketitle

\section*{Instructions: Attempt all questions.}
Q. 1 Robert studiert \textcolor{red}{(studies)} seit einem Monat \textcolor{red}{(for a month)} in München \textcolor{red}{(Munich)}. (Robert has been studying in Munich for a month.)\\
Er wohnt \textcolor{red}{(He lives)} mit seinem Freund \textcolor{red}{(with his friend)} Hans beim Kaufmann \textcolor{red}{(at the merchant)} Krüger, Elisabethplatz \textcolor{red}{(Elisabeth Square)} 30. (He lives with his friend Hans at the merchant Krüger's, Elisabeth Square 30.)\\
Frau Krüger ist ihre Hausfrau \textcolor{red}{(housewife)}. (Mrs. Krüger is their housewife.)\\
Die Wohnung \textcolor{red}{(The apartment)} ist nicht weit \textcolor{red}{(not far)} von der Universität \textcolor{red}{(university)}. (The apartment is not far from the university.)\\
Sie liegt \textcolor{red}{(It lies)} der Post \textcolor{red}{(post office)} gegenüber \textcolor{red}{(opposite)}. (It lies opposite the post office.)

Morgens \textcolor{red}{(In the morning)} um 8.00 Uhr geht Robert aus dem Haus \textcolor{red}{(Robert leaves the house)} und fährt \textcolor{red}{(travels)} mit seinem Fahrrad \textcolor{red}{(with his bicycle)} zur Universität. (In the morning at 8.00 Robert leaves the house and travels to the university with his bicycle.)\\
Hans geht immer zu Fuß \textcolor{red}{(Hans always goes on foot)}, denn er hat kein Fahrrad \textcolor{red}{(because he has no bicycle)}. (Hans always goes on foot, because he has no bicycle.)\\
Der Weg \textcolor{red}{(The way)} ist nicht weit: vom Elisabethplatz zur Universität braucht er nur 10 Minuten \textcolor{red}{(he only needs 10 minutes)}. (The way is not far: from Elisabeth Square to the university he only needs 10 minutes.)

Mittags \textcolor{red}{(At noon)} geht Robert mit seinem Freund zum Essen \textcolor{red}{(Robert goes with his friend to eat)}. (At noon Robert goes with his friend to eat.)\\
Sie gehen die Ludwigstraße \textcolor{red}{(Ludwig Street)} entlang \textcolor{red}{(along)} und dann links um die Ecke \textcolor{red}{(around the corner)} zu einem Gasthaus \textcolor{red}{(inn)}. (They go along Ludwig Street and then around the corner to an inn.)\\
Dort ißt \textcolor{red}{(eats)} man sehr gut \textcolor{red}{(very well)}. (There one eats very well.)\\
Gewöhnlich \textcolor{red}{(Usually)} bestellen \textcolor{red}{(order)} sie das Menü \textcolor{red}{(menu)}, das ist nicht so teuer \textcolor{red}{(not so expensive)}. (Usually they order the menu, which is not so expensive.)\\
Nach dem Essen lesen sie manchmal noch die Zeitungen \textcolor{red}{(newspapers)} oder die Illustrierten \textcolor{red}{(magazines)} und trinken \textcolor{red}{(drink)} ein Glas Bier \textcolor{red}{(a glass of beer)} oder eine Tasse Kaffee \textcolor{red}{(a cup of coffee)}. (After eating they sometimes read the newspapers or magazines and drink a glass of beer or a cup of coffee.)

Nachmittags \textcolor{red}{(In the afternoon)} geht Robert ohne seinen Freund zur Universität, denn Hans arbeitet zu Haus \textcolor{red}{(at home)} für seine Prüfung \textcolor{red}{(for his exam)}. (In the afternoon Robert goes to the university without his friend, because Hans works at home for his exam.)\\
Nach der Vorlesung \textcolor{red}{(After the lecture)} fährt er nach Haus \textcolor{red}{(he goes home)}. (After the lecture he goes home.)\\
Manchmal macht er auch noch einen Spaziergang \textcolor{red}{(a walk)} durch den Park \textcolor{red}{(through the park)}. (Sometimes he also takes a walk through the park.)\\
Nach dem Abendessen \textcolor{red}{(After dinner)} gehen die Freunde zusammen spazieren \textcolor{red}{(the friends go for a walk together)}. (After dinner the friends go for a walk together.)\\
Manchmal besuchen sie ein Kino \textcolor{red}{(cinema)} oder ein Theater \textcolor{red}{(theater)}, oder sie arbeiten zu Haus. (Sometimes they visit a cinema or a theater, or they work at home.)\\
Meistens \textcolor{red}{(Mostly)} gehen sie aber früh \textcolor{red}{(early)} zu Bett \textcolor{red}{(to bed)}, denn sie sind abends \textcolor{red}{(in the evenings)} immer sehr müde \textcolor{red}{(tired)}. (Mostly they go to bed early, because they are always very tired in the evenings.)

Answer the following questions based on the above text !
(a) Was ist Robert von Beruf und wo wohnt er?
(b) Wie fährt er zur Universität?
(c) Was macht er nach dem Essen ?
(d) Warum geht Robert ohne seinen Freund zur Universität nachmittags?
(e) Was machen die Freunde nach dem Abendessen?
Q. 2 Form sentences with the opposites of the following words (any ten)!
teuer \textcolor{red}{(expensive)}, feundlich \textcolor{red}{(friendly)}, groß \textcolor{red}{(big)}, fleißig \textcolor{red}{(hardworking)}, dunkel \textcolor{red}{(dark)}, langsam \textcolor{red}{(slow)}, schwer \textcolor{red}{(difficult)}, richtig \textcolor{red}{(right)}, schön \textcolor{red}{(beautiful)}, früh \textcolor{red}{(early)}, alt \textcolor{red}{(old)}, immer \textcolor{red}{(always)}
Q. 3 Use the correct verb forms (any ten)!
(a) Das Mädchen (sprechen) nur Englisch.
(b) Ich (kaufen) ein neues Wörterbuch.
(c) Jeder Student an der Universität (arbeiten) 8 Stunden.
(d) Wie lange (studieren) er Mathematik?
(e) Ein Kilo Butter (kosten) nur \(€ 3,00\).
(f) Die Lehrerin (geben) den Studenten die Hausaufgabe.
(g) Nach dem Essen (lesen) er die deutsche Zeitung.
(h) Morgen (fahren) der Professor nach Berlin.
(i) Die Studentin (schreiben) die Sätze in das Heft.
(j) Fräulein Klein (haben) ein schönes Haus in Stuttgart.
(k) Heute (besuchen) mein Freund das Sprachlabor.
Q. 4 Frame questions to which the underlined words provide answers (any ten) !
(a) Diese Studenten kommen aus Japan.
(b) Der Unterricht beginnt um 11.00 Uhr.
(c) Das Seminar dauert eine Stunde.
(d) Morgen fahern wir nach Kalkutta.
(e) Seit 2010 studiert er in IIT, Kharagpur.
(f) Herr Brekle ist sehr nett und freundlich.
(g) Eine Postkarte kostet \(€ 1,30\).
(h) Meine Eltern wohnen in Deutschland.
(i) Sein Vater ist Autofahrer.
(j) Ungefähr 30 Studenten lernen Deutsch.
(k) Der Student versteht den Satz nicht.
Q. 5 Replace the underlined portions by appropriate pronouns (any five sentences)!
(a) Herr Hoffmann kauft seiner Frau einen neuen Wagen.
(b) Der Ausländer zeigt dem Kind viele Bücher.
(c) Herr Robert schenkt dem Gast eine Uhr.
(d) Monika zeigt den Freunden ein schönes Bild.
(e) Das Mädchen bringt dem Mann die Zeitung.
(f) Der Lehrerin gibt der Briefträger ein Telegramm.
Q. 6 Form sentences with the following groups of words (any ten) !
(a) er, sein Bruder, ein Fahrrad,kaufen,morgen
(b) trinken, noch,ich,eine Tasse,möchte,Kaffee
(c) derAusländer,gehen,das Institut,in,täglich
(d) trinken, noch, wir, eine Tasse, möchten, Kaffee
(e) er, sein Freund, schreiben, ein Brief, immer
(f) viele, fahren, Leute, nach, heute, Delhi
(g) er, seine Freundin, eine Uhr, schenken, morgen
(h) die Mutter, der Sohn, kaufen, ein Fahrrad
(i) ich,mein Freund, schreiben, ein Brief,immer
(j) die Leute, Wein, hier, trinken, und, singen
(k) brauchen, Anna, das Auto, nicht, heute

\end{document}
