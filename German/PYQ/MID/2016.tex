\documentclass{article}
\usepackage[utf8]{inputenc}
\usepackage{xcolor}

\title{}


\date{Mid-Spring Semester 2015-2016}

\begin{document}

\maketitle

\noindent
Sub Name: German \\


\section*{Q.1. Read the following passage carefully and answer the questions that follow:}

Morgens \textcolor{red}{(In the morning)} um 10.00 Uhr geht Martin aus dem Haus \textcolor{red}{(leaves the house)} und fährt \textcolor{red}{(travels)} mit seinem Auto \textcolor{red}{(with his car)} zur Universität \textcolor{red}{(to the university)}. (In the morning at 10.00 Martin leaves the house and travels to the university with his car.)\\
Peter geht immer zu Fuß \textcolor{red}{(always goes on foot)}, denn er hat kein Fahrrad \textcolor{red}{(because he has no bicycle)}. (Peter always goes on foot, because he has no bicycle.)\\
Der Weg \textcolor{red}{(The way)} ist nicht weit \textcolor{red}{(is not far)}: vom Elisabethplatz \textcolor{red}{(from Elisabeth Square)} zur Universität braucht er nur 10 Minuten \textcolor{red}{(he only needs 10 minutes)}. (The way is not far: from Elisabeth Square to the university he only needs 10 minutes.)

Martin studiert \textcolor{red}{(studies)} seit einem Jahr \textcolor{red}{(for a year)} in Hamburg \textcolor{red}{(Hamburg)}. (Martin has been studying in Hamburg for a year.)\\
Er wohnt \textcolor{red}{(He lives)} mit seinem Freund \textcolor{red}{(with his friend)} Peter beim Kaufmann \textcolor{red}{(at the merchant)} Krüger, Elisabethplatz 40. (He lives with his friend Peter at the merchant Krüger's, Elisabeth Square 40.)\\
Frau Krüger ist ihre Hausfrau \textcolor{red}{(housewife)}. (Mrs. Krüger is their housewife.)\\
Die Wohnung \textcolor{red}{(The apartment)} ist nicht weit von der Universität. (The apartment is not far from the university.)\\
Sie liegt \textcolor{red}{(It lies)} der Post \textcolor{red}{(the post office)} gegenüber \textcolor{red}{(opposite)}. (It lies opposite the post office.)

Mittags \textcolor{red}{(At noon)} geht Martin mit seinem Freund \textcolor{red}{(with his friend)} zum Essen \textcolor{red}{(to eat)}. (At noon Martin goes with his friend to eat.)\\
Sie gehen die Ludwigstraße \textcolor{red}{(Ludwig Street)} entlang \textcolor{red}{(along)} und dann links \textcolor{red}{(left)} um die Ecke \textcolor{red}{(around the corner)} zu einem Gasthaus \textcolor{red}{(to an inn)}. (They go along Ludwig Street and then left around the corner to an inn.)\\
Dort ißt man sehr gut \textcolor{red}{(One eats very well)}. (One eats very well there.)\\
Gewöhnlich \textcolor{red}{(Usually)} bestellen \textcolor{red}{(order)} sie das Menü \textcolor{red}{(the menu)}, das ist nicht so teuer \textcolor{red}{(which is not so expensive)}. (Usually they order the menu, which is not so expensive.)\\
Nach dem Essen \textcolor{red}{(After the meal)} lesen sie manchmal noch die Zeitungen \textcolor{red}{(the newspapers)} oder die Illustrierten \textcolor{red}{(the magazines)} und trinken \textcolor{red}{(drink)} ein Glas Bier \textcolor{red}{(a glass of beer)} oder eine Tasse Kaffee \textcolor{red}{(a cup of coffee)}. (After the meal they sometimes read the newspapers or the magazines and drink a glass of beer or a cup of coffee.)

Nachmittags \textcolor{red}{(In the afternoon)} geht Martin ohne seinen Freund \textcolor{red}{(without his friend)} zur Universität, denn Peter arbeitet zu Haus \textcolor{red}{(works at home)} für seine Prüfung \textcolor{red}{(for his exam)}. (In the afternoon Martin goes to the university without his friend, because Peter works at home for his exam.)\\
Nach der Vorlesung \textcolor{red}{(After the lecture)} fährt er nach Haus \textcolor{red}{(he goes home)}. (After the lecture he goes home.)\\
Manchmal macht er auch noch einen Spaziergang \textcolor{red}{(he also takes a walk)} durch den Park \textcolor{red}{(through the park)}. (Sometimes he also takes a walk through the park.)\\
Nach dem Abendessen \textcolor{red}{(After the dinner)} gehen die Freunde zusammen spazieren \textcolor{red}{(the friends go for a walk)}. (After the dinner the friends go for a walk together.)\\
Manchmal besuchen sie ein Kino \textcolor{red}{(a cinema)} oder ein Theater \textcolor{red}{(a theater)}, oder sie arbeiten zu Haus \textcolor{red}{(or they work at home)}. (Sometimes they visit a cinema or a theater, or they work at home.)\\
Meistens \textcolor{red}{(Mostly)} gehen sie aber früh zu Bett \textcolor{red}{(they go to bed early)}, denn sie sind abends immer sehr müde \textcolor{red}{(because they are always very tired in the evenings)}. (Mostly they go to bed early, because they are always very tired in the evenings.)

\section*{Answer the following questions based on the above text !}
(a) Wie fährt Martin zur Universität ? (How does Martin travel to the university?)
(b) Seit wann studiert Martin in Hamburg ? (Since when has Martin been studying in Hamburg?)
(c) Wohin gehen die Freunde zum Essen ? (Where do the friends go to eat?)
(d) Was machen sie nach dem essen? (What do they do after eating?)
(e) Warum gehen sie früh zum Bett? (Why do they go to bed early?)
. \(V\) N \(G(R)\)
Name in Capital Letter

\section*{Q.2. Form sentences with the opposites of the following words (any ten)!}
billig \textcolor{red}{(cheap)}, gut \textcolor{red}{(good)}, schwer \textcolor{red}{(difficult)}, klein \textcolor{red}{(small)}, schnell \textcolor{red}{(fast)}, schön \textcolor{red}{(beautiful)}, immer \textcolor{red}{(always)}, früh \textcolor{red}{(early)}, hell \textcolor{red}{(bright)}, fleißig \textcolor{red}{(hardworking)}, freundlich \textcolor{red}{(friendly)}, alt \textcolor{red}{(old)}
\section*{Q.3. Use the correct verb forms (any ten)!}
(a) Morgen (fahren) der Professor nach Berlin. (Tomorrow (travel) the professor to Berlin.)
(b) Wie lange (studieren) er Mathematik in Erfurt? (How long (study) he mathematics in Erfurt?)
(c) Ein Kilo Butter (kosten) nur Euro 2,20-. (One kilo butter (cost) only Euro 2.20-.)
(d) Fräulein Fenchel (haben) ein schönes Haus in Stuttgart. (Miss Fenchel (have) a beautiful house in Stuttgart.)
(e) Heute (besuchen) mein Freund das Sprachlabor in München. (Today (visit) my friend the language lab in Munich.)
(f) Jeder Student im Institut (arbeiten) 11 Stunden. (Every student in the institute (work) 11 hours.)
(g) Nach dem Essen (lesen) Robert die englische Zeitung. (After the meal (read) Robert the English newspaper.)
(h) Das Mädchen (geben) dem Gast ein Glas Wasser. (The girl (give) the guest a glass of water.)
(i) Die Studentin (schreiben) die Sätze in das Heft. (The student (write) the sentences in the notebook.)
(j) Der Amerikaner (sprechen) nur Englisch. (The American (speak) only English.)
(k) (Kaufen) du ein neues deutsches Wörterbuch? ((Buy) you a new German dictionary?)
\section*{Q.4. Replace the underlined portions by appropriate pronouns (any five sentences)!}
(a) Herr Hoffmann kauft seiner Frau einen neuen Wagen. (Mr. Hoffmann buys his wife a new car.)
(b) Der Briefträger gibt der Lehrerin ein Telegramm. (The postman gives the teacher a telegram.)
(c) Herr Robert schenkt dem Gast viele Bücher. (Mr. Robert gives the guest many books.)
(d) Monika zeigt den Freunden ein schönes Bild. (Monika shows the friends a beautiful picture.)
(e) Der Ausländer kauft dem Kind ein Fahrrad. (The foreigner buys the child a bicycle.)
(f) Das Mädchen bringt dem Mann eine englische Zeitung. (The girl brings the man an English newspaper.)
\section*{Q.5. Form sentences with the following groups of words (any five)!}
(a) er, sein Freund, schreiben, ein Brief, immer (he, his friend, write, a letter, always)
(b) der Ausländer, gehen, das Institut, in, täglich (the foreigner, go, the institute, in, daily)
(c) die Mutter, der Sohn, kaufen, ein Auto (the mother, the son, buy, a car)
(d) viele, fahren, Leute, nach, heute, Banglore (many, travel, people, to, today, Banglore)
(e) er, seine Freundin, eine Uhr, schenken, morgen (he, his girlfriend, a watch, give, tomorrow)
(f) trinken, noch, Sie, eine Tasse, möchten, Kaffee (drink, still, you, a cup, would like, coffee)
\section*{Q.6. Frame questions to which the underlined words provide answers (any ten)!}
(a) Morgen fahren die Studenten nach Kalkutta. (Tomorrow travel the students to Calcutta.)
(b) Unsere Lehrerin ist sehr nett und freundlich. (Our teacher is very nice and friendly.)
(c) Ungefähr 70 Studenten lernen Deutsch als Fremdsprache. (Approximately 70 students learn German as a foreign language.)
(d) Dieser Student versteht das Wort nicht. (This student doesn't understand the word.)
(e) Das Buch liegt auf dem Tisch in dem Zimmer. (The book lies on the table in the room.)
(f) Eine Postkarte kostet Euro 1,30. (A postcard costs Euro 1.30.)
(g) Mein Sohn studiert in Deutschland. (My son studies in Germany.)
(h) Seit 2006 studiert er in IIT Kharagpur. (Since 2006 he studies in IIT Kharagpur.)
(i) Diese Studenten kommen aus Berlin. (These students come from Berlin.)
(j) Das Seminar beginnt um 11.00 Uhr. (The seminar begins at 11.00 o'clock.)
(k) Die Fahrt dauert drei Stunden. (The journey takes three hours.)

\end{document}
