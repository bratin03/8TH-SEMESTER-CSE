\documentclass{article}
\usepackage[utf8]{inputenc}
\usepackage{xcolor}

\definecolor{darkgreen}{rgb}{0.0, 0.5, 0.0}

\title{
INDIAN INSTITUTE OF TECHNOLOGY KHARAGPUR
}

\title{
Mid-Spring Semester 2022-23
}

\date{}

\begin{document}

\maketitle

\noindent
Date of Examination: Session (FN/AN) Duration: 2 hrs \\
Subject No. : HS30048 \\
Department: Humanities \& Social Sciences

\noindent
Full Marks: 60/2 \\
Subject: German

\section*{}
\begin{enumerate}
    \item Read the following passages carefully and answer the questions that follow:

    Die Stadt \textcolor{red}{(The city)} Paris ist für viele ein Traumziel \textcolor{red}{(dream destination)}, das sie wenigstens \textcolor{red}{(at least)} einmal im Leben \textcolor{red}{(in life)} besuchen \textcolor{red}{(visit)} möchten \textcolor{red}{(would like)}. (The city of Paris is a dream destination for many, that they would like to visit at least once in their life.) \\
    Wer möchte nicht einmal unter dem Eiffelturm \textcolor{red}{(Eiffel Tower)} stehen \textcolor{red}{(stand)} und in den Pariser Himmel \textcolor{red}{(Parisian sky)} hoch \textcolor{red}{(up)} schauen \textcolor{red}{(look)}? (Who wouldn't want to stand under the Eiffel Tower and look up at the Parisian sky?) \\
    Und dann zu Fuß \textcolor{red}{(on foot)} und mit dem Aufzug \textcolor{red}{(elevator)} wenigstens \textcolor{red}{(at least)} auf halbe Höhe \textcolor{red}{(half height)} gelangen \textcolor{red}{(reach)} und den Blick \textcolor{red}{(view)} über die ganze \textcolor{red}{(whole)} französische Hauptstadt \textcolor{red}{(French capital)} genießen \textcolor{red}{(enjoy)}? (And then walk and take the elevator to at least half height and enjoy the view over the entire French capital?) \\
    Die Stadt hat noch viele weitere \textcolor{red}{(more)} Höhepunkte \textcolor{red}{(highlights)} zu bieten \textcolor{red}{(to offer)}. (The city has many more highlights to offer.) \\
    Typisch \textcolor{red}{(Typical)} ist das zum zweiten Mal \textcolor{red}{(for the second time)} umgebaute \textcolor{red}{(rebuilt)} Hallenviertel \textcolor{red}{(market district)}, wo der ehemalige \textcolor{red}{(former)} Großmarkt \textcolor{red}{(wholesale market)} im Herzen \textcolor{red}{(heart)} der Stadt lag \textcolor{red}{(was located)}. (Typical is the market district, rebuilt for the second time, where the former wholesale market was located in the heart of the city.) \\
    Hier findet man heute viele junge \textcolor{red}{(young)} Köche \textcolor{red}{(cooks)}, die ihr kleines Restaurant \textcolor{red}{(small restaurant)} eröffnet \textcolor{red}{(opened)} haben \textcolor{red}{(have)}, auf dem Weg \textcolor{red}{(on the way)} zum vielleicht \textcolor{red}{(perhaps)} nächsten Starkoch \textcolor{red}{(star chef)}. (Here you can find many young cooks today, who have opened their small restaurant, on the way to becoming perhaps the next star chef.) \\
    Die Besucher \textcolor{red}{(visitors)} freut \textcolor{red}{(pleases)} es, so können sie sich an innovativer \textcolor{red}{(innovative)}, französischer Küche \textcolor{red}{(French cuisine)} zu vernünftigen \textcolor{red}{(reasonable)} Preisen \textcolor{red}{(prices)} erfreuen \textcolor{red}{(enjoy)}. (The visitors are pleased, so they can enjoy innovative French cuisine at reasonable prices.)

    Heute ist der erste Schultag \textcolor{red}{(first day of school)}. (Today is the first day of school.) \\
    Lena steht mit ihrer Schultüte \textcolor{red}{(school cone)} vor der Schule \textcolor{red}{(school)}. (Lena stands with her school cone in front of the school.) \\
    Sandra, Susanne und Paul sind auch da \textcolor{red}{(there)}. (Sandra, Susanne and Paul are also there.) \\
    Die Kinder kennen \textcolor{red}{(know)} sich aus dem Kindergarten \textcolor{red}{(kindergarten)}. (The children know each other from kindergarten.) \\
    Jetzt gehen sie in die gleiche Klasse \textcolor{red}{(same class)}. (Now they go to the same class.) \\
    Sie freuen sich schon auf den Unterricht \textcolor{red}{(class)}. (They are already looking forward to the class.) \\
    Lena freut sich besonders \textcolor{red}{(especially)} auf das Rechnen \textcolor{red}{(arithmetic)}. (Lena is particularly looking forward to arithmetic.) \\
    Sandra und Susanne aufs Schreiben \textcolor{red}{(writing)}. (Sandra and Susanne to writing.) \\
    Und Paul? (And Paul?) \\
    Paul sagt \textcolor{red}{(says)}, er freut sich besonders auf die Pausen \textcolor{red}{(breaks)}. (Paul says he is especially looking forward to the breaks.) \\
    In der Klasse lernen sie ihren Lehrer \textcolor{red}{(teacher)}, Herrn Mayer, kennen \textcolor{red}{(meet)}. (In the class they meet their teacher, Mr. Mayer.) \\
    Herr Mayer ist noch sehr jung \textcolor{red}{(young)} und lustig \textcolor{red}{(funny)}. (Mr. Mayer is still very young and funny.) \\
    In der ersten Stunde \textcolor{red}{(hour)} lernen die Kinder das ABC-Lied \textcolor{red}{(ABC song)}. (In the first hour the children learn the ABC song.) \\
    Alle singen \textcolor{red}{(sing)} begeistert \textcolor{red}{(enthusiastically)} mit \textcolor{red}{(along)}. (Everyone sings along enthusiastically.) \\
    Danach schreibt der Lehrer die ersten Buchstaben \textcolor{red}{(letters)} an die Tafel \textcolor{red}{(blackboard)}: A wie Affe \textcolor{red}{(monkey)}, B wie Banane \textcolor{red}{(banana)}. (Then the teacher writes the first letters on the blackboard: A for monkey, B for banana.) \\
    Herr Mayer zeichnet \textcolor{red}{(draws)} einen Affen dazu \textcolor{red}{(with it)}, der einen Banane frisst \textcolor{red}{(eats)}. (Mr. Mayer draws a monkey with it, who is eating a banana.) \\
    Die Kinder lachen \textcolor{red}{(laugh)} laut \textcolor{red}{(loudly)}. (The children laugh loudly.) \\
    Dann läutet \textcolor{red}{(rings)} auch schon die Schulglocke \textcolor{red}{(school bell)}. (Then the school bell rings.) \\
    Der erste Tag in der Schule ist vorbei \textcolor{red}{(over)}. (The first day at school is over.) \\
    Vor der Schule warten die Eltern \textcolor{red}{(parents)} auf die Kinder. (In front of the school the parents are waiting for the children.) \\
    Die Kinder erzählen \textcolor{red}{(tell)} vom ersten Tag. (The children tell about the first day.) \\
    Sie freuen sich schon auf morgen \textcolor{red}{(tomorrow)}. (They are already looking forward to tomorrow.)

    Answer the following questions based on the above texts:
    \begin{enumerate}
        \item Welche Stadt ist für viele ein Traumziel? (Which city is a dream destination for many?)
        \item Wo findet man heute viele junge Köche? (Where can you find many young cooks today?)
        \item Wie heißen Lenas Freunde? (What are Lena's friends' names?)
        \item Was machen die Kinder in ihrer ersten Stunde? (What do the children do in their first hour?)
        \item Wer warten auf die Kinder vor der Schule? (Who is waiting for the children in front of the school?)
    \end{enumerate}
    \item Form sentences with the opposites of the following words (any ten):
    schlecht, spät, hell, fleißig, leicht, klein, schnell, billig, schön, nie, freundlich, alt
    \begin{enumerate}
        \item[(a)] schlecht - \textcolor{darkgreen}{gut}
        \item[(b)] spät - \textcolor{darkgreen}{früh}
        \item[(c)] hell - \textcolor{darkgreen}{dunkel}
        \item[(d)] fleißig - \textcolor{darkgreen}{faul}
        \item[(e)] leicht - \textcolor{darkgreen}{schwer}
        \item[(f)] klein - \textcolor{darkgreen}{groß}
        \item[(g)] schnell - \textcolor{darkgreen}{langsam}
        \item[(h)] billig - \textcolor{darkgreen}{teuer}
        \item[(i)] schön - \textcolor{darkgreen}{hässlich}
        \item[(j)] nie - \textcolor{darkgreen}{immer}
        \item[(k)] freundlich - \textcolor{darkgreen}{unfreundlich}
        \item[(l)] alt - \textcolor{darkgreen}{neu}
    \end{enumerate}
    \item Write the following numbers in words:
    16,38,98,317,7964
    \begin{enumerate}
        \item \textcolor{darkgreen}{Sechzehn}
        \item \textcolor{darkgreen}{Achtunddreißig}
        \item \textcolor{darkgreen}{Achtundneunzig}
        \item \textcolor{darkgreen}{Dreihundertsiebzehn}
        \item \textcolor{darkgreen}{Siebentausendneunhundertvierundsechzig}
    \end{enumerate}


\end{enumerate}
\section*{}
    Use the correct form of the verbs given in the brackets (any ten):
    \begin{enumerate}
        \item[(a)] Nach dem Unterricht (gehen) die Studenten zum Studentenheim. \textcolor{darkgreen}{gehen}
        \item[(b)] Die Studentin (arbeiten) zehn Stunden täglich. \textcolor{darkgreen}{arbeitet}
        \item[(c)] Der Professor (fahren) morgen nach Deutschland. \textcolor{darkgreen}{fährt}
        \item[(d)] Die Dame (erklären) den Studenten das Bild. \textcolor{darkgreen}{erklärt}
        \item[(e)] Der Student (kaufen) dem Kind ein neues Buch. \textcolor{darkgreen}{kauft}
        \item[(f)] Peter (sein) ein Student und studiert in Berlin. \textcolor{darkgreen}{ist}
        \item[(g)] Meine Freundin (haben) einen Bruder und eine Schwester. \textcolor{darkgreen}{hat}
        \item[(h)] Der Gast (trinken) Kaffee ohne Milch und Zucker. \textcolor{darkgreen}{trinkt}
        \item[(i)] Der Ausläder (lesen) die Wörter sehr schnell. \textcolor{darkgreen}{liest}
        \item[(j)] Die Studentin (sprechen) Deutsch langsam. \textcolor{darkgreen}{spricht}
        \item[(k)] Das Mädchen (geben) dem Gast ein Glas Wasser. \textcolor{darkgreen}{gibt}
\end{enumerate}
\begin{enumerate}
    \setcounter{enumi}{4}
    \item Replace the underlined portions by appropriate pronouns (any five sentences):
    \begin{enumerate}
        \item[(a)] Robert schenkt \underline{seinem Freund} \underline{ein neues Buch}. \textcolor{darkgreen}{Robert schenkt ihm es.}
        \item[(b)] Die Frau zeigt \underline{den Studenten} \underline{die schönen Bilder}. \textcolor{darkgreen}{Die Frau zeigt ihnen sie.}
        \item[(c)] \underline{Der Schüler} hat \underline{ein neues Fahrrad}. \textcolor{darkgreen}{Er hat es.}
        \item[(d)] \underline{Der Mann} kauft \underline{der Frau} eine neue Uhr. \textcolor{darkgreen}{Er kauft ihr eine neue Uhr.}
        \item[(e)] \underline{Die Lehrerin} bringen \underline{den Kindern} die neuen Bücher. \textcolor{darkgreen}{Sie bringen ihnen die neuen Bücher.}
        \item[(f)] \underline{Frau Schmid} bringt \underline{ihrer Freundin} viele Bücher.  \textcolor{darkgreen}{Sie bringt ihr viele Bücher.}
    \end{enumerate}
    \item Form sentences with the following groups of words (any five)!
    \begin{enumerate}
        \item[(a)] die Lehrerin, die Sätze, die Studenten, eklären \textcolor{darkgreen}{Die Lehrerin erklärt den Studenten die Sätze.}
        \item[(b)] die Arbeiter, acht, arbeiten, täglich, Stunden \textcolor{darkgreen}{Die Arbeiter arbeiten acht Stunden täglich.}
        \item[(c)] heute, fahren, nach, die Dame, Stuttgart \textcolor{darkgreen}{Die Dame fährt nach Stuttgart heute.} 
        \item[(d)] Herr Schmitt, alt, 40, Jahre, sein, und, in, Paris, arbeiten \textcolor{darkgreen}{Herr Schmitt ist 40 Jahre alt und arbeitet in Paris.}
        \item[(e)] der Ausländer, bringen, das Mädchen, die Zeitung \textcolor{darkgreen}{Der Ausländer bringt dem Mädchen die Zeitung.}
        \item[(f)] die Leute, trinken, und, Tee, gern, mit Milch, Zucker \textcolor{darkgreen}{Die Leute trinken Tee gern mit Milch und Zucker.}
    \end{enumerate}
    \item Frame questions to which the underlined words provide answers (any five)!
    \begin{enumerate}
        \item[(a)] Der Bus fährt \underline{um 9 Uhr} von Tatanagar ab. \textcolor{darkgreen}{Wann fährt der Bus von Tatanagar ab?}
        \item[(b)] Es ist schon \underline{10.00 Uhr}. \textcolor{darkgreen}{Wie spät ist es?}
        \item[(c)] Der Amerikaner wohnt jetzt \underline{in Bonn}. \textcolor{darkgreen}{Wo wohnt der Amerikaner jetzt?}
        \item[(d)] Die Sekretärin ist sehr \underline{nett und freundlich}. \textcolor{darkgreen}{Wie ist die Sekretärin?}
        \item[(e)] Diese Studenten kommen \underline{aus Japan}. \textcolor{darkgreen}{Woher kommen diese Studenten?}
        \item[(f)] Das Buch kostet nur \underline{Euro 3,30}. \textcolor{darkgreen}{Wie viel kostet das Buch?}
        \item[(g)] Der Direktor fährt \underline{nach Berlin} morgen. \textcolor{darkgreen}{Wohin fährt der Direktor morgen?}
    \end{enumerate}
\end{enumerate}

\end{document}
