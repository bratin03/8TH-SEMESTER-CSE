\documentclass{article}
\usepackage[utf8]{inputenc}
\usepackage{xcolor}

\title{
INDIAN INSTITUTE OF TECHNOLOGY KHARAGPUR Mid-Spring Semester 2017-18
}

\author{
Date of Examination: \\ Subject No. HS30048 \\ Department: Humanities \& Social Sciences \\ Full Marks: 60/2 \\ Subject: German
}

\date{}

\begin{document}

\maketitle

\section*{Q.1. Read the following passage carefully and answer the questions that follow:}

Maria studiert \textcolor{red}{(studies)} seit einem Monat \textcolor{red}{(for a month)} in Berlin \textcolor{red}{(Berlin)}. (Maria has been studying in Berlin for a month.) \\
Sie wohnt \textcolor{red}{(She lives)} mit ihrem Freund \textcolor{red}{(with her friend)} Robert beim Kaufmann \textcolor{red}{(at the merchant)} Brekle, Elizabethplatz \textcolor{red}{(Elizabeth Square)}, 12. (She lives with her friend Robert at the merchant Brekle's, Elizabeth Square, 12.) \\
Frau Schmit \textcolor{red}{(Mrs. Schmit)} ist ihre Hausfrau \textcolor{red}{(housewife)}. (Mrs. Schmit is their housewife.) \\
Die Wohnung \textcolor{red}{(The apartment)} ist nicht weit \textcolor{red}{(not far)} von der Universität \textcolor{red}{(from the university)}. (The apartment is not far from the university.) \\
Die liegt \textcolor{red}{(It lies)} der Post \textcolor{red}{(the post office)} gegenüber \textcolor{red}{(opposite)}. (It lies opposite the post office.) \\
Morgens \textcolor{red}{(In the mornings)} um 11.00 Uhr geht Maria aus dem Haus \textcolor{red}{(Maria leaves the house)} und fährt \textcolor{red}{(travels)} mit ihrem Auto \textcolor{red}{(with her car)} zur Universität \textcolor{red}{(to the university)}. (In the mornings at 11.00 Maria leaves the house and travels with her car to the university.) \\
Robert geht immer zu Fuß \textcolor{red}{(Robert always goes on foot)}, denn er hat kein Auto \textcolor{red}{(because he has no car)}. (Robert always goes on foot, because he has no car.) \\
Der Weg \textcolor{red}{(The way)} ist nicht weit \textcolor{red}{(is not far)}: vom Elizabethplatz \textcolor{red}{(from Elizabeth Square)} zur Universität braucht man nur 30 Minuten \textcolor{red}{(it takes only 30 minutes)}. (The way is not far: from Elizabeth Square to the university it takes only 30 minutes.)

Mittags \textcolor{red}{(At noon)} geht Maria mit ihrem Freund \textcolor{red}{(Maria goes with her friend)} zum Essen \textcolor{red}{(to eat)}. (At noon Maria goes with her friend to eat.) \\
Sie gehen die Ludwigstraße \textcolor{red}{(They go along Ludwig Street)} entlang \textcolor{red}{(along)} und dann links \textcolor{red}{(then left)} um die Ecke \textcolor{red}{(around the corner)} zu einem Gasthaus \textcolor{red}{(to an inn)}. (They go along Ludwig Street and then left around the corner to an inn.) \\
Dort ißt man sehr gut \textcolor{red}{(One eats very well there)}. (One eats very well there.) \\
Gewöhnlich \textcolor{red}{(Usually)} bestellen sie das Menü \textcolor{red}{(they order the menu)}, das ist nicht so teuer \textcolor{red}{(which is not so expensive)}. (Usually they order the menu, which is not so expensive.) \\
Nach dem essen \textcolor{red}{(After eating)} lesen sie manchmal noch die Zeitungen \textcolor{red}{(they sometimes read the newspapers)} oder die Illustrierten \textcolor{red}{(or the magazines)} und trinken ein Glas Tee \textcolor{red}{(and drink a glass of tea)} oder eine Tasse Kaffee \textcolor{red}{(or a cup of coffee)}. (After eating they sometimes read the newspapers or the magazines and drink a glass of tea or a cup of coffee.) \\
Nachmittags \textcolor{red}{(In the afternoon)} geht Maria ohne ihren Freund \textcolor{red}{(Maria goes without her friend)} zur Universität \textcolor{red}{(to the university)}, denn Robert arbeitet zu Haus \textcolor{red}{(because Robert works at home)} für seine Prüfung \textcolor{red}{(for his exam)}. (In the afternoon Maria goes to the university without her friend, because Robert works at home for his exam.) \\
Nach der Vorlesung \textcolor{red}{(After the lecture)} fährt Robert nach Haus \textcolor{red}{(Robert goes home)}. (After the lecture Robert goes home.) \\
Abendessen \textcolor{red}{(For dinner)} gehen die Freunde zusammen spazieren \textcolor{red}{(the friends go for a walk together)}. (For dinner the friends go for a walk together.) \\
Manchmal besuchen sie ein Kino \textcolor{red}{(Sometimes they visit a cinema)} oder ein Theater \textcolor{red}{(or a theater)}, oder sie arbeiten zu Haus \textcolor{red}{(or they work at home)}. (Sometimes they visit a cinema or a theater, or they work at home.) \\
Meistens \textcolor{red}{(Mostly)} gehen sie aber früh zu Bett \textcolor{red}{(they go to bed early)}, denn sie sind abends immer sehr müde \textcolor{red}{(because they are always very tired in the evenings)}. (Mostly they go to bed early, because they are always very tired in the evenings.)

\section*{Answer the following questions based on the above text !}
(a) Wo studiert Maria und wo wohnt sie? (Where does Maria study and where does she live?)
(b) Wie fährt Maria zur Universität? (How does Maria travel to the university?)
(c) Wo essen Maria und Robert? (Where do Maria and Robert eat?)
(d) Was machen Maria und Robert nach dem Essen? (What do Maria and Robert do after eating?)
(e) Wohin fährt Robert nach der Vorlesung? (Where does Robert travel after the lecture?)
Q.2. Form sentences with the opposites of the following words (any ten)!
schnell \textcolor{red}{(fast)}, billig \textcolor{red}{(cheap)}, fleißig \textcolor{red}{(hardworking)}, freundlich \textcolor{red}{(friendly)}, alt \textcolor{red}{(old)}, kalt \textcolor{red}{(cold)}, gut \textcolor{red}{(good)}, schwer \textcolor{red}{(difficult)}, klein \textcolor{red}{(small)}, schön \textcolor{red}{(beautiful)}, immer \textcolor{red}{(always)}, früh \textcolor{red}{(early)}, richtig \textcolor{red}{(correct)}
Q.3. Frame questions to which the underlined words provide answers (any ten)!
(a) Der Lehrer erklärt den Studenten die Theorie. (The teacher explains the theory to the students.)
(b) Das Wörterbuch kostet ein hundert Euro. (The dictionary costs one hundred Euro.)
(c) Diese Studenten kommen aus Deutschland. (These students come from Germany.)
(d) Es ist schon 10.00 Uhr. (It is already 10.00 o'clock.)
(e) Der Amerikaner wohnt in Varanasi. (The American lives in Varanasi.)
(f) Robert studiert in München. (Robert studies in Munich.)
(g) Morgen fahren die Studenten nach Neu Delhi. (Tomorrow the students travel to New Delhi.)
(h) Meine Studenten sind sehr nett. (My students are very nice.)
(i) Seit 2016 arbeitet er bei Siemens. (Since 2016 he works at Siemens.)
(j) Der Zug fährt um 10.00 Uhr von Kharagpur ab. (The train departs at 10.00 o'clock from Kharagpur.)
(k) Es gibt dreißig Studenten in dem Zimmer. (There are thirty students in the room.)
Q. 4 . Replace the underlined portions by appropriate pronouns (any five sentences)! 10
(a) Das Mädchen bringt dem Gast eine Tasse Kaffee. (The girl brings the guest a cup of coffee.)
(b) Die Schülerin kauft dem Ausländer ein Fahrrad. (The student buys the foreigner a bicycle.)
(c) Der Briefträger bringt der Frau ein Telegram. (The postman brings the woman a telegram.)
(d) Robert zeigt Frau Kühn die schnönen Bilder. (Robert shows Mrs. Kühn the beautiful pictures.)
(e) Die Studenten schenken der Lehrerin ein neues Buch. (The students give the teacher a new book.)
(f) Die Lehrerin erklärt den Studenten die Relativitästheorie. (The teacher explains the theory of relativity to the students.)
Q.5. Use the correct verb forms (any ten)!
(a) Mein Bruder (arbeiten) bei Firma Siemens. (My brother (work) at Siemens.)
(b) Frau Müller (sprechen) Hindi sehr schnell. (Mrs. Müller (speak) Hindi very fast.)
(c) Prof. Brekle (haben) ein großes Haus in Bremen. (Prof. Brekle (have) a big house in Bremen.)
(d) Der Schüler (geben) der Lehrerin ein neues Buch. (The student (give) the teacher a new book.)
(e) Wieviel (kosten) ein Bleistift? (How much (cost) a pencil?)
(f) (Kaufen) du ein neues Fahrrad? ((Buy) you a new bicycle?)
(g) Die Studentin (schreiben) die Regeln in die Hefte. (The student (write) the rules in the notebooks.)
(h) Der Professor (erklären) den Studenten die Theorie. (The professor (explain) the students the theory.)
(i) Der Gast (fahren) morgen früh nach Berlin. (The guest (travel) to Berlin early tomorrow.)
(j) Das Mädchen (antworten) immer richtig. (The girl (answer) always correctly.)
(k) (Lesen) du die Zeitung täglich? (Do (read) you the newspaper daily?)

\section*{Q.6. Form sentences with the following groups of words (any five)!}
(a) brauchen, sie, das Auto, nicht, heute (need, she, the car, not, today)
(b) sie, das Kind, ein Buch, kaufen (she, the child, a book, buy)
(c) der Student, Frau Kühn, ein Bild, zeigen (the student, Mrs. Kühn, a picture, show)
(d) Robert, ein Füller, sein Freund, geben (Robert, a pen, his friend, give)
(e) wie lange, die Zeitung, er, lesen? (how long, the newspaper, he, read?)
(f) die Lehrerin, täglich, zehn Stunden, arbeiten (the teacher, daily, ten hours, work)

\section*{Q.7. Write ten sentences in German about "Mein/e Lehrer/Lehrerin" or 'Mein/e Freund/in!}

\end{document}
