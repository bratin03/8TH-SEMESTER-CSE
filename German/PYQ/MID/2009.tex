\documentclass{article}
\usepackage[utf8]{inputenc}
\usepackage{xcolor}

\title{
INDIAN INSTITUTE OF TECHNOLOGY KHARAGPUR
}


\date{}

\begin{document}

\maketitle

\section*{Q.1. Read the following passage carefully and answer the questions that follow:}

Ravi ist Student \textcolor{red}{(student)} und kommt \textcolor{red}{(comes)} aus Indien \textcolor{red}{(India)}. (Ravi is a student and comes from India.)\\
Zwei Wochen \textcolor{red}{(weeks)} ist er schon \textcolor{red}{(already)} in Deutschland \textcolor{red}{(Germany)} und lernt \textcolor{red}{(learns)} Deutsch \textcolor{red}{(German)}. (He has been in Germany for two weeks and is learning German.)\\
Sein Zimmer \textcolor{red}{(His room)} ist groß \textcolor{red}{(big)}, und geräumig \textcolor{red}{(spacious)}. (His room is big and spacious.)\\
Es hat \textcolor{red}{(It has)} einen Tisch \textcolor{red}{(table)}, drei Stühle \textcolor{red}{(chairs)}, zwei Lampen \textcolor{red}{(lamps)}, ein Bett \textcolor{red}{(bed)} und einen Schrank \textcolor{red}{(wardrobe)}. (It has a table, three chairs, two lamps, a bed and a wardrobe.)

Seine Frau \textcolor{red}{(His wife)} ist auch Studentin \textcolor{red}{(student)} an der Universität \textcolor{red}{(university)} in Ludwigsburg. (His wife is also a student at the university in Ludwigsburg.)\\
Sie studiert \textcolor{red}{(She studies)} Pädagogik \textcolor{red}{(pedagogy)} und schreibt \textcolor{red}{(writes)} jetzt \textcolor{red}{(now)} ihre Abschlußprüfung \textcolor{red}{(final exam)}. (She studies pedagogy and is now writing her final exam.)\\
Sie spricht \textcolor{red}{(She speaks)} fließend \textcolor{red}{(fluently)} Deutsch. (She speaks fluent German.)\\
Sie hat viele Freunde \textcolor{red}{(friends)}. (She has many friends.)\\
Ravi besucht \textcolor{red}{(visits)} auch ihre Freunde oft \textcolor{red}{(often)}. (Ravi also visits her friends often.)

Ravi studiert Medizin \textcolor{red}{(medicine)} an der Universität Stuttgart. (Ravi studies medicine at the University of Stuttgart.)\\
Er und seine Frau sprechen \textcolor{red}{(speak)} und üben \textcolor{red}{(practice)} zusammen \textcolor{red}{(together)} Deutsch zu Hause \textcolor{red}{(at home)}. (He and his wife speak and practice German together at home.)\\
Ravi hat auch einen Freund \textcolor{red}{(friend)}. (Ravi also has a friend.)\\
Er heißt \textcolor{red}{(He is called)} Peter Bremen. (He is called Peter Bremen.)\\
Er kommt aus Dänmark \textcolor{red}{(Denmark)}. (He comes from Denmark.)\\
Ravi und Peter arbeiten zusammen und lernen Deutsch. (Ravi and Peter work together and learn German.)\\
Peter fragt \textcolor{red}{(asks)} und Ravi antwortet \textcolor{red}{(answers)}. (Peter asks and Ravi answers.)\\
Er sagt \textcolor{red}{(He says)} die Wörter \textcolor{red}{(words)}, die Nomen \textcolor{red}{(nouns)}, und Ravi sagt die Bedeutung \textcolor{red}{(meaning)}, die Artikel \textcolor{red}{(articles)} und die Pluralformen \textcolor{red}{(plural forms)}. (He says the words, the nouns, and Ravi says the meaning, the articles and the plural forms.)\\
Sie machen viele Fehler \textcolor{red}{(mistakes)} aber sie lernen langsam \textcolor{red}{(slowly)} und werden immer \textcolor{red}{(always)} sicherer \textcolor{red}{(more confident)}. (They make many mistakes but they learn slowly and become more confident.)\\
Sein Freund lacht \textcolor{red}{(laughs)} und verbessert \textcolor{red}{(corrects)} die Fehler. (His friend laughs and corrects the mistakes.)\\
Ravi nimmt \textcolor{red}{(takes)} das Buch \textcolor{red}{(book)} später \textcolor{red}{(later)}. (Ravi takes the book later.)\\
Jetzt ist Ravi der Lehrer \textcolor{red}{(teacher)}, und Peter beantwortet \textcolor{red}{(answers)} seine Fragen \textcolor{red}{(questions)}. (Now Ravi is the teacher, and Peter answers his questions.)\\
Er macht auch viele Fehler. (He also makes many mistakes.)\\
Nun \textcolor{red}{(Now)} lacht Ravi. (Now Ravi laughs.)\\
So lernen sie die Sprache \textcolor{red}{(language)} - leicht \textcolor{red}{(easily)} und schnell \textcolor{red}{(quickly)}. (That's how they learn the language - easily and quickly.)

\section*{Answer the following questions based on the above text:}
(a) Was ist Ravi von Beruf und woher kommt er ? (What is Ravi's profession and where does he come from?)
(b) Wo und was studiert seine Frau? (Where and what does his wife study?)
(c) Was studiert Ravi? (What does Ravi study?)
(d) Hat Ravi einen Freund? Wenn, ja wie heißt sein Freund? (Does Ravi have a friend? If so, what is his friend's name?)
(e) Woher kommt Peter Bremen? (Where does Peter Bremen come from?)

\footnotetext{Q.2. Form sentences with the opposites of the following words (any ten)!}
groß \textcolor{red}{(big)}, schnell \textcolor{red}{(fast)}, teuer \textcolor{red}{(expensive)}, schön \textcolor{red}{(beautiful)}, nie \textcolor{red}{(never)}, spät \textcolor{red}{(late)}, fleißig \textcolor{red}{(hardworking)}, freundlich \textcolor{red}{(friendly)}, neu \textcolor{red}{(new)}, schlecht \textcolor{red}{(bad)}, schwer \textcolor{red}{(difficult)}

\section*{Q. 3. Use the correct forms of the verbs given in the brackets:}
(a) Der Professor (erklären) den Studenten die Sätze. (The professor (explain) the sentences to the students.)
(b) Die Studentin (fahren) morgen früh nach Berlin. (The student (travel) to Berlin early tomorrow.)
(c) Mein Freund (antworten) immer richtig. (My friend (answer) always correctly.)
(d) (Lesen) sie die Zeitung täglich? (Do you (read) the newspaper daily?)
(e) Peter Bremen (arbeiten) bei Firma Siemens. (Peter Bremen (work) at Siemens company.)
(f) Frau Schmid (sprechen) Deutsch sehr schnell. (Mrs. Schmid (speak) German very quickly.)
(g) Prof. Schmid (haben) ein großes Auto in Berlin. (Prof. Schmid (have) a big car in Berlin.)
(h) Frau Maier (geben) dem Kind ein neues Buch. (Mrs. Maier (give) the child a new book.)
(i) Wieviel (kosten) ein Liter Milch? (How much (cost) a liter of milk?)
(j) (Kaufen) ihr ein neues Fahrrad? (Are you (buy) a new bicycle?)
(k) Die Studentinnen (schreiben) die Regeln in die Hefte. (The students (write) the rules in the notebooks.)

\section*{Q.4. Replace the underlined portions by appropriate pronouns (any five sentences)!}
(a) Sie zeigt ihren Freunden die schönen Bilder. (She shows her friends the beautiful pictures.)
(b) Dieser Schüler hat ein neues Wörterbuch. (This student has a new dictionary.)
(c) Der Lehrerin kauft er eine Uhr. (He buys the teacher a watch.)
(d) Der Arbeiter hat einen alten Wagen. (The worker has an old car.)
(e) Er schenkt seinem Freund einen neuen Füller. (He gives his friend a new pen.)
(f) Frau Müller kauft den Kindern viele Bücher. (Mrs. Müller buys the children many books.)

\section*{Q.5. Form sentences with the following groups of words (any ten)!}
(a) Ravi, ein Füller, der Freund, kaufen (Ravi, a pen, the friend, buy)
(b) wie lange, die Zeitung, Sie, lesen ? (how long, the newspaper, you, read?)
(c) die Lehrerin, täglich, zehn Stunden, arbeiten (the teacher, daily, ten hours, work)
(d) brauchen, Sie, das Auto, nicht, heute (need, you, the car, not, today)
(e) der Ausländer, das Kind, ein Buch, schenken (the foreigner, the child, a book, give)
(f) der Student, Frau Schmit, ein Bild, zeigen (the student, Mrs. Schmit, a picture, show)
(g) ich, immer, der indische Tee, trinken (I, always, the Indian tea, drink)
(h) er, der Gast, eine Blume, schenken (he, the guest, a flower, give)
(i) Herr Robert, rauchen, Zigaretten, nie (Mr. Robert, smoke, cigarettes, never)
(j) Herr Pfeil, in, Stuttgart, wohnen (Mr. Pfeil, in, Stuttgart, live)
(k) der Professor, schnell, Deutsch, sprechen (the professor, quickly, German, speak)

\section*{Q.6. Frame questions to which the underlined words provide answers (any ten)!}
(a) Diese Studenten arbeiten 10 Stunden täglich. (These students work 10 hours daily.)
(b) Morgen fahren wir nach Neu Delhi. (Tomorrow we travel to New Delhi.)
(c) Mein Professor ist sehr nett und freundlich. (My professor is very nice and friendly.)
(d) Seit 2005 studiert er in Kharagpur. (He has been studying in Kharagpur since 2005.)
(e) Dieses Buch kostet nur Euro 36. (This book costs only Euro 36.)
(f) Mein Vater kommt nach Kharagpur heute abend. (My father comes to Kharagpur this evening.)
(g) Das Zimmer ist sehr teuer aber komfortabel. (The room is very expensive but comfortable.)
(h) Der Zug fährt um 9 Uhr von Kharagpur ab. (The train departs from Kharagpur at 9 o'clock.)
(i) Diese Studenten kommen aus Singapore. (These students come from Singapore.)
(j) Es ist schon 11.00 Uhr. (It is already 11.00 o'clock.)
(k) Der Amerikaner wohnt in Benaras. (The American lives in Benaras.)
(l) Die Leute fahren nach Berlin morgen. (The people travel to Berlin tomorrow.)

\section*{Q. 7 Write definite article of the following nouns (any ten)):}

Tisch \textcolor{red}{(table)}, Füller \textcolor{red}{(pen)}, Mädchen \textcolor{red}{(girl)}, Haus \textcolor{red}{(house)}, Bleistift \textcolor{red}{(pencil)}, Schule \textcolor{red}{(school)}, Institut \textcolor{red}{(institute)}, Uhr \textcolor{red}{(clock)}, Jahr \textcolor{red}{(year)}, Stdentenheim \textcolor{red}{(student dormitory)}, Glas \textcolor{red}{(glass)}, Apfel \textcolor{red}{(apple)}, Frühstück \textcolor{red}{(breakfast)}, Tag \textcolor{red}{(day)}, Arbeiter \textcolor{red}{(worker)}
\end{document}
