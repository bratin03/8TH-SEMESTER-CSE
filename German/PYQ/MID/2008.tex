\documentclass{article}
\usepackage[utf8]{inputenc}
\usepackage{xcolor}

\title{
INDIAN INSTITUTE OF TECHNOLOGY
}

\date{}

\begin{document}

\maketitle

\section*{Q.1. Read the following passage carefully and answer the questions that follow:}

Ismet Tosun kommt aus der Türkei \textcolor{red}{(Turkey)}, aus Ankara. (Ismet Tosun comes from Turkey, from Ankara.)\\
Er arbeitet \textcolor{red}{(He works)} in der Bundesrepublik \textcolor{red}{(Federal Republic)} Deutschland \textcolor{red}{(Germany)}. (He works in the Federal Republic of Germany.)\\
Er wohnt \textcolor{red}{(He lives)} im Haus \textcolor{red}{(house)} von Dino unten \textcolor{red}{(down)} im Erdgeschoß \textcolor{red}{(ground floor)}. (He lives in Dino's house downstairs on the ground floor.)\\
Dino hat oben \textcolor{red}{(upstairs)} im dritten Stock \textcolor{red}{(third floor)} ein Zimmer \textcolor{red}{(a room)}. (Dino has a room upstairs on the third floor.)\\
Manchmal \textcolor{red}{(Sometimes)} geht Ismet zu Dino rauf \textcolor{red}{(up)}, oder Dino kommt runter \textcolor{red}{(down)} zu ihm. (Sometimes Ismet goes up to Dino, or Dino comes down to him.)

In jedem \textcolor{red}{(every)} Stockwerk \textcolor{red}{(floor)} sind drei Wohnungen \textcolor{red}{(apartments)}: eine Wohnung links \textcolor{red}{(left)}, eine in der Mitte \textcolor{red}{(middle)} und eine rechts \textcolor{red}{(right)}. (On every floor there are three apartments: one apartment on the left, one in the middle and one on the right.)\\
Links neben \textcolor{red}{(next to)} ihm wohnt eine Familie aus der Schweiz \textcolor{red}{(Switzerland)}. (Next to him lives a family from Switzerland.)\\
Er geht oft rüber \textcolor{red}{(over)} zu ihnen. (He often goes over to them.)\\
Über ihm wohnt eine Jugoslawin \textcolor{red}{(Yugoslavian woman)}. (Above him lives a Yugoslavian woman.)

Ismet kommt von der Arbeit \textcolor{red}{(work)} nach Haus \textcolor{red}{(home)}. (Ismet comes home from work.)\\
Vor seiner Tür \textcolor{red}{(door)} steht Dino. (Dino is standing in front of his door.)\\
Dino möchte \textcolor{red}{(would like)} ihn einladen \textcolor{red}{(invite)}, denn morgen ist Silvester \textcolor{red}{(New Year's Eve)}. (Dino would like to invite him, because tomorrow is New Year's Eve.)\\
Er hat Freunde in der Nähe \textcolor{red}{(near)} von Köln \textcolor{red}{(Cologne)}. (He has friends near Cologne.)\\
Sie machen morgen abend \textcolor{red}{(tomorrow evening)} ein Fest \textcolor{red}{(a party)}. (They are having a party tomorrow evening.)

Am Silvesterabend \textcolor{red}{(On New Year's Eve)} fährt \textcolor{red}{(travels)} Ismet zu dem Fest. (On New Year's Eve Ismet travels to the party.)\\
Zuerst \textcolor{red}{(First)} geht er über einen Platz \textcolor{red}{(square)} und durch einen Park \textcolor{red}{(park)}, dann an einem Restaurant vorbei \textcolor{red}{(past)} und die Straße \textcolor{red}{(street)} entlang \textcolor{red}{(along)}. (First he goes over a square and through a park, then past a restaurant and along the street.)\\
Er wartet \textcolor{red}{(He waits)} an der Haltestelle \textcolor{red}{(bus stop)}. (He waits at the bus stop.)\\
Hier steigt er in den Bus Linie \textcolor{red}{(line)} 10 ein \textcolor{red}{(gets on)} und fährt bis zur post \textcolor{red}{(post office)}. (Here he gets on the bus line 10 and goes to the post office.)\\
Von dort geht er zum Bahnhof \textcolor{red}{(train station)}. (From there he goes to the train station.)
Answer the following questions based on the above text! ..... 10
\begin{enumerate}
    \item[(a)] Wo arbeitet Ismet Tosun.
    \item[(b)] Wohin geht Ismet manchmal?
    \item[(c)] Wieviele Wohnungen sind in jedem Stockwerk?
    \item[(d)] Was machen die Freunde morgen abend?
    \item[(e)] Wohin fäht Ismet mit dem Bus?
\end{enumerate}
Q. 2. Form sentences with the opposites of the following words (any ten)! 10 \\
schwer \textcolor{red}{(difficult)}, falsch \textcolor{red}{(wrong)}, billig \textcolor{red}{(cheap)}, schön \textcolor{red}{(beautiful)}, spät \textcolor{red}{(late)}, alt \textcolor{red}{(old)}, nie \textcolor{red}{(never)}, feundlich \textcolor{red}{(friendly)}, groß \textcolor{red}{(big)}, fleißig \textcolor{red}{(hardworking)}, dunkel \textcolor{red}{(dark)}, schnell \textcolor{red}{(fast)}

\section*{Q. 3. Use the correct verb forms (any ten) !}
\begin{enumerate}
    \item[(a)] Ravi (lesen) die Zeitung nach dem Mittagessen.
    \item[(b)] Die Studentin (antworten) immer richtig.
    \item[(c)] Ist sie Lehrerin? Nein, sie (sein) Studentin.
    \item[(d)] Die Frau (geben) dem Mann eine neue Uhr.
    \item[(e)] Die Dame (haben) ein schönes Auto.
    \item[(f)] Der Ausländer (sprechen) Deutsch sehr schnell.
    \item[(g)] Prof. Schmid (fahren) mit einem großen Auto nach Berlin.
    \item[(h)] Was trinkt sie gern? Sie (trinken) gern Bier.
    \item[(i)] Dieser Student (kommen) aus Berlin.
    \item[(j)] Die Dame (arbeiten) 10 Stunden im Büro.
    \item[(k)] (Kaufen) du ein neues Auto morgen?
    \item[(l)] Der Professor (rauchen) Zigaretten manchmal.
\end{enumerate}

\section*{Q. 4. Complete the following sentences by using appropriate words!.}
\begin{enumerate}
    \item[(a)] Frau -------möchte \(\qquad\)
    \item[(b)] Danke, ich \(\qquad\)
    \item[(c)] Die -------- trinkt Tee.
    \item[(d)] Die ------------- kommen nicht---- Japan.
    \item[(e)] Herr Rose ------ Bier, ----------.
    \item[(f)] Der Herr ------Kaffee ----..---.
    \item[(g)] Das ----------- trinkt Milch.
    \item[(h)] ------------ Sie Orangensaft ?
    \item[(i)] Nehmen Sie \(\qquad\) oder Milch ?
    \item[(j)] Der alte Mann \(\qquad\)
\end{enumerate}
Q. 6. Make sentences by combining the words in each group, changing verb forms when necessary.
\begin{enumerate}
    \item[(a)] der Lehrer, schnell, Deutsch, sprechen
    \item[(b)] Wie lange, die Zeitung, die Studentin, lesen ?
    \item[(c)] ich, bei, immer, kaufen, Frau Luft
    \item[(d)] Die Tomaten, kosten, 5 Rupien, das Kilo, nur
    \item[(e)] Tee, du, trinken, Milch, ohne
    \item[(f)] es, mir, heute, nicht, gehen, gut, so
    \item[(g)] Sie, lernen, oder, schnell, langsam
    \item[(h)] du, lernen, Deutsch, fleißig, sehr
    \item[(i)] die Frau, kaufen, und, ein Pfund, Brot, Milch, ein Liter
    \item[(j)] ich, trinken, jetzt, Tee, eine Tasse
    \item[(k)] kommen, der Herr, aus, Deutschland, nicht
\end{enumerate}
Q. 7. Frame questions to which the underlined portions provide answers (any ten)!
\begin{enumerate}
    \item[(a)] Mein Vater kommt nach Kharagpur heute abend.
    \item[(b)] Das Zimmer ist sehr teuer aber komfortabel.
    \item[(c)] Der Zug fährt um 9 Uhr von Kharagpur ab.
    \item[(d)] Diese Studenten kommen aus Singapore.
    \item[(e)] Es ist schon 11.00 Uhr .
    \item[(f)] Der Amerikaner wohnt in Benaras.
    \item[(g)] Die Leute fahren nach Berlin morgen.
    \item[(h)] Diese Studenten arbeiten 10 Stunden täglich.
    \item[(i)] Morgen fahren wir nach Neu Delhi.
    \item[(j)] Mein Professor ist sehr nett und freundlich.
    \item[(k)] Seit 2005 studiert er in Kharagpur.
    \item[(l)] Dieses Buch kostet nur Euro 3,:
\end{enumerate}

\end{document}
