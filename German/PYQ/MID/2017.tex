\documentclass{article}
\usepackage[utf8]{inputenc}
\usepackage{xcolor}

\title{
INDIAN INSTITUTE OF TECHNOLOGY KHARAGPUR \\
Department of Humanities \& Social Sciences
}


\date{}

\begin{document}

\maketitle

\section*{Q.1. Read the following passage carefully and answer the questions that follow:}

Deutschland ist ein großes Land in der Mitte Europas \textcolor{red}{(Germany is a big country in the middle of Europe)}. (Germany is a big country in the middle of Europe.)\\
Es ist eines der reichsten Länder der Welt \textcolor{red}{(It is one of the richest countries in the world)} und ein wichtiges Mitglied \textcolor{red}{(important member)} (member) der Europäischen Union \textcolor{red}{(European Union)} (EU). (It is one of the richest countries in the world and an important member of the European Union (EU).) \\
Deutschland ist berühmt \textcolor{red}{(famous)} (famous) für hochwertige Industrieprodukte \textcolor{red}{(high-quality industrial products)}, besonders \textcolor{red}{(especially)} aus den Bereichen \textcolor{red}{(fields)} (field) Elektrizität \textcolor{red}{(electricity)}, Elektronik \textcolor{red}{(electronics)}, Chemie \textcolor{red}{(chemistry)} und Autos \textcolor{red}{(cars)}. (Germany is famous for high-quality industrial products, especially from the fields of electricity, electronics, chemistry and cars.) \\
Norddeutschland ist ziemlich flach \textcolor{red}{(quite flat)} mit kleinen Hügeln \textcolor{red}{(small hills)} (hills) und weiten Ebenen \textcolor{red}{(wide plains)}. (Northern Germany is quite flat with small hills and wide plains.) \\
Im Tiefland \textcolor{red}{(In the lowlands)} an der Nordseeküste \textcolor{red}{(at the North Sea coast)} befinden sich \textcolor{red}{(are located)} (are situated) die Städte Bremen und Hamburg \textcolor{red}{(the cities Bremen and Hamburg)}. (In the lowlands at the North Sea coast are located the cities Bremen and Hamburg.)\\
Auch die Hauptstadt \textcolor{red}{(Also the capital)} Berlin befindet sich im Norden Deutschlands \textcolor{red}{(is located in the north of Germany)}. (Also the capital Berlin is located in the north of Germany.)

Im Süden liegt der Schwarzwald \textcolor{red}{(In the south lies the Black Forest)}, ein Mittelgebirge \textcolor{red}{(a low mountain range)}. (In the south lies the Black Forest, a low mountain range.)\\
Von dort sieht man die weißen Gipfel \textcolor{red}{(From there one sees the white peaks)} der Schweizer Alpen \textcolor{red}{(of the Swiss Alps)}. (From there one sees the white peaks of the Swiss Alps.) \\
Das Klima \textcolor{red}{(The climate)} im Rheinland \textcolor{red}{(in the Rhineland)} ist freundlich \textcolor{red}{(is friendly)} und es wachsen \textcolor{red}{(grow)} (grow) nicht nur Mais \textcolor{red}{(corn)}, Kartoffeln \textcolor{red}{(potatoes)} und Weizen \textcolor{red}{(wheat)}, sondern auch Spargeln \textcolor{red}{(asparagus)} und Trauben \textcolor{red}{(grapes)}. (The climate in the Rhineland is friendly and not only corn, potatoes and wheat grow, but also asparagus and grapes.)\\
Deutschland hat 80 Millionen Einwohner \textcolor{red}{(Germany has 80 million inhabitants)} (inhabitants). (Germany has 80 million inhabitants.)\\
Die meisten sprechen Deutsch als Muttersprache \textcolor{red}{(Most speak German as their mother tongue)}. (Most speak German as their mother tongue.)\\
In der Schule lernen alle Englisch als erste Fremdsprache \textcolor{red}{(In school everyone learns English as their first foreign language)}. (In school everyone learns English as their first foreign language.) \\
In Deutschland wohnen auch viele Leute mit Migrationshintergrund \textcolor{red}{(In Germany also live many people with a migration background)}. (In Germany also live many people with a migration background.) \\
Vor allem \textcolor{red}{(Especially)} in den 1960er-Jahren kamen viele Ausländer nach Deutschland \textcolor{red}{(in the 1960s many foreigners came to Germany)}. (Especially in the 1960s many foreigners came to Germany.)\\
Damals waren die deutschen Fabriken knapp an Arbeitskräften \textcolor{red}{(At that time the German factories were short of workers)}. (At that time the German factories were short of workers.)\\
Heute arbeitet ein großer Teil der Deutschen in der Industrie oder in Dienstleistungsbetrieben wie Banken und Versicherungen oder Hotels \textcolor{red}{(Today a large part of Germans works in industry or in service companies like banks and insurances or hotels)}. (Today a large part of Germans works in industry or in service companies like banks and insurances or hotels.)

\section*{Answer the following questions based on the above text !}
(a) Wo liegt Deutschland? (Where is Germany located?)
(b) Was für ist Deutschland berühmt? (What is Germany famous for?)
(c) Wo befinden sich die Städte Bremen und Hamburg? (Where are the cities Bremen and Hamburg located?)
(d) Wie ist das Klima im Rheinland? (What is the climate in the Rhineland like?)
(e) Wie viele Leute wohnen in Deutschland? (How many people live in Germany?)

\hrulefill

\section*{Q.2. Frame questions to which the underlined words provide answers (any ten)!}
(a) Es gibt vierzig Studenten in dem Seminar. (There are forty students in the seminar.)
(b) Der Professor erklärt den Studenten die Sätze. (The professor explains the sentences to the students.)
(c) Das Wörterbuch kostet zwei hundert Euro. (The dictionary costs two hundred Euro.)
(d) Diese Studenten kommen aus Japan. (These students come from Japan.)
(e) Heute fahren die Studenten nach Berlin. (Today the students travel to Berlin.)
(f) Der Ausländer ist sehr nett und freundlich. (The foreigner is very nice and friendly.)
(g) Mein Bruder arbeitet bei Microsoft seit 2008. (My brother has been working at Microsoft since 2008.)
(h) Der Bus fährt um 13.00 Uhr von Howrah ab. (The bus departs from Howrah at 13.00 o'clock.)
(i) Es ist schon 13.00 Uhr. (It is already 13.00 o'clock.)
(j) Die Japanerin wohnt in Benaras und lernt Musik. (The Japanese woman lives in Benaras and learns music.)
(k) Robert studiert in Stuttgart seit vier Jahren? (Robert has been studying in Stuttgart for four years?)

\section*{Q.3. Replace the underlined portions by appropriate pronouns (any five sentences)!}
(a) Der Lehrer erklärt den Studenten die Relativitästheorie. (The teacher explains the theory of relativity to the students.)
(b) Das Mädchen bringt dem Gast eine Tasse Kaffee. (The girl brings the guest a cup of coffee.)
(c) Die Schülerin kauft dem Ausländer ein Fahrrad. (The student buys the foreigner a bicycle.)
(d) Peter zeigt Frau Schmid die schönen Bilder. (Peter shows Mrs. Schmid the beautiful pictures.)
(e) Die Studentinnen schenken der Lehrerin ein neues Buch. (The students give the teacher a new book.)
(f) Der Briefträger bringt Frau Kühn ein Telegram. (The postman brings Mrs. Kühn a telegram.)

\section*{Q.4. Use the correct verb forms (any ten)!}
(a) Herr Löffelholz (sprechen) Hindi sehr schnell. (Mr. Löffelholz (speak) Hindi very fast.)
(b) Prof. Raman (haben) ein großes Haus in Frankfurt. (Prof. Raman (have) a big house in Frankfurt.)
(c) Der Student (geben) der Lehrerin ein neues Buch. (The student (give) the teacher a new book.)
(d) Wieviel (kosten) ein neues Fahrrad? (How much (cost) a new bicycle?)
(e) Die Studentin (schreiben) die Regeln in die Hefte. (The student (write) the rules in the notebooks.)
(f) Der Professor (erklären) den Studenten die Theorie. (The professor (explain) the theory to the students.)
(g) Der Gast (fahren) morgen früh nach Berlin. (The guest (travel) to Berlin early tomorrow.)
(h) (Kaufen) du ein neues Fahrrad? (Are you (buy) a new bicycle?)
(i) Das Mädchen (antworten) immer richtig. (The girl (answer) always correctly.)
(j) (Lesen) er die deutsche Zeitung täglich? (Does he (read) the German newspaper daily?)
(k) Mein Freund (arbeiten) bei Siemens seit sieben Jahren. (My friend (work) at Siemens for seven years.)

\section*{Q.5. Form sentences with the opposites of the following words (any ten)!}
schnell \textcolor{red}{(fast)}, schön \textcolor{red}{(beautiful)}, immer \textcolor{red}{(always)}, früh \textcolor{red}{(early)}, hell \textcolor{red}{(bright)}, billig \textcolor{red}{(cheap)}, gut \textcolor{red}{(good)}, schwer \textcolor{red}{(difficult)}, klein \textcolor{red}{(small)}, fleißig \textcolor{red}{(hardworking)}, freundlich \textcolor{red}{(friendly)}, alt \textcolor{red}{(old)}

\section*{Q.6. Form sentences with the following groups of words (any ten)!}
(a) brauchen, mein Freund, das Auto, nicht, heute (need, my friend, the car, not, today)
(b) sie, das Kind, ein Bleistift, kaufen (she, the child, a pencil, buy)
(c) der Student, Frau Inge, ein Bild, zeigen (the student, Mrs. Inge, a picture, show)
(d) Freund, haben, zwei, mein, Brüder (friend, have, two, my, brothers)
(e) arbeiten, ich, Technogerma, bei, Japan, in (work, I, Technogerma, at, Japan, in)
(f) Peter, ein Füller, sein Freund, geben (Peter, a pen, his friend, give)
(g) wie lange, die Zeitung, Maria, lesen ? (how long, the newspaper, Maria, read?)
(h) die Lehrerin, täglich, zehn Stunden, arbeiten (the teacher, daily, ten hours, work)
(i) zusammen, wir, ein, machen, heute, Experiment (together, we, an, make, today, experiment)
(j) der Briefträger, ein Telegram, der Ausländer, bringen (the postman, a telegram, the foreigner, bring)
(k) trinken, hier, Bier, die Leute, heute (drink, here, beer, the people, today)

\section*{Q.7.Write ten sentences in German about "Mein/e Lehrer/Lehrein" or 'Meine Familie'! }
\end{document}
