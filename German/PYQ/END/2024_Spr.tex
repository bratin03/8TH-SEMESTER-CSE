\documentclass[a4paper,12pt]{article}
\usepackage[utf8]{inputenc}
\usepackage{geometry}
\geometry{margin=1in}
\usepackage{amsmath}
\usepackage{amssymb}
\usepackage{titlesec}
\usepackage{xcolor}
\definecolor{darkgreen}{rgb}{0.0, 0.5, 0.0}

% Formatting for sections
\titleformat{\section}{\normalfont\Large\bfseries}{\thesection}{1em}{}
\titleformat{\subsection}{\normalfont\large\bfseries}{\thesubsection}{1em}{}

\begin{document}

% Title
\begin{center}
    \textbf{INDIAN INSTITUTE OF TECHNOLOGY KHARAGPUR}\\
    \textbf{End-Spring Semester Examination 2023-24}\\
    \vspace{0.5cm}
    \begin{tabular}{l l}
        Date of Examination: & \hspace{2cm} \\
        Session: & FN/AN \\
        Duration: & 3 hrs \\
        Subject No.: & HS30048 \\
        Subject Name: & German \\
        Department/Center/School: & Humanities \& Social Sciences \\
        Full Marks: & 100/2 \\
    \end{tabular}
\end{center}

\vspace{1cm}

% Question 1
\section*{Q.1 Übersetzen Sie ins Englische! (10 marks)}

Familie Huber wohnt in der Nähe von Wien. Frank und Monika haben den Kindern einen Tag in Wien versprochen. Am Samstag fahren sie gemeinsam \textcolor{red}{together} nach Wien, in die Hauptstadt von Österreich. Das Auto stellen sie in einem Parkhaus ab und fahren mit der U-Bahn \textcolor{red}{subway} weiter in die Innenstadt \textcolor{red}{City Center}. Als \textcolor{red}{as} Erstes \textcolor{red}{first} besichtigen sie den Stephansdom \textcolor{red}{St. Stephen's Cathedral}, das Wahrzeichen \textcolor{red}{Landmark} von Wien. Diese alte gotische \textcolor{red}{Gothic} Kirche wollen alle Besucher sehen. Wer Zeit hat, kann den Turm \textcolor{red}{tower} besteigen \textcolor{red}{climb} oder die Katakomben \textcolor{red}{catacombs} unter der Kirche besichtigen. Die "Pummerin" ist die größte Glocke \textcolor{red}{bell} Österreichs. Sie hängt im Nordturm \textcolor{red}{Northern Tower} und kann mit einem Aufzug \textcolor{red}{elevator} erreicht werden. Familie Huber ist beeindruckt \textcolor{red}{impressed}. Sie zünden \textcolor{red}{light} in der Kirche eine Kerze \textcolor{red}{Candle} an und gehen weiter.

Inzwischen haben alle Hunger und essen bei einem Würstelstand eine Burenwurst. Das ist eine Wiener Spezialität. Es schmeckt ihnen und sie spazieren \textcolor{red}{troll} quer durch die Altstadt. Als Nächstes machen sie eine Besichtigungsfahrt \textcolor{red}{Sight Seeing trip} mit der Straßenbahn \textcolor{red}{Tram}. Die Ringstraße führt rund um das Zentrum der Stadt. Sie nutzen \textcolor{red}{use} die ganz normale Straßenbahn und können beim Vorbeifahren die prächtigen \textcolor{red}{magnificent} Bauten \textcolor{red}{buildings} bewundern \textcolor{red}{admire}. Sie sehen die Votivkirche, das Wiener Rathaus, das Burgtheater, das Parlament, zwei große Museen und die Wiener Staatsoper \textcolor{red}{Street Opera}

Die Kinder wollen endlich in den Prater. Sie wollen keine Häuser mehr anschauen \textcolor{red}{look at}. Der Prater ist ein Vergnügungspark \textcolor{red}{amusement Park}. Hier fahren sie mit Ringelspiel \textcolor{red}{ring game}, Autodrom \textcolor{red}{bumper cars} und dem Riesenrad \textcolor{red}{Ferris wheel}. Das ist auch eines der Wahrzeichen der Stadt. Frank möchte ins Schweizer Haus, einen großen Biergarten. Dort machen sie Rast \textcolor{red}{red} und essen Wiener Schnitzel.

Sie wollten noch zum wunderschönen Schloss \textcolor{red}{Castle} Schönbrunn und in den Tiergarten. Dafür reicht die Zeit aber nicht. Das machen sie beim nächsten \textcolor{red}{next} Besuch in Wien.

\vspace{0.5cm}

Vocabulary: 
versprechen = to promise; besichtigen = to visit; Wahrzeichen = landmark; besteigen = to climb; beeindrucken = to get impressed; zünden = to light; Kerze = candle; Burenwurst = boer sausage; quer = across; nutzen = to use; prächtigen = magnificent; bewundern = to admire; anschauen = to look at; Vergnügungspark = amusement park; Ringelspiel = ring game; Riesenrad = Ferris wheel; Schloss = castle; reichen = to be sufficient.

\subsection*{Q.1 Translate into English! (10 marks)}

The Huber family lives near Vienna. Frank and Monika promised the children a day in Vienna. On Saturday, they travel together to Vienna, the capital of Austria. They park the car in a parking garage and take the subway to the city center. First, they visit St. Stephen's Cathedral, the landmark of Vienna. All visitors want to see this old Gothic church. Those who have time can climb the tower or visit the catacombs beneath the church. The "Pummerin" is the largest bell in Austria. It hangs in the north tower and can be reached by an elevator. The Huber family is impressed. They light a candle in the church and continue on.

Meanwhile, everyone is hungry and they eat a Burenwurst at a sausage stand. That is a Viennese specialty. They enjoy it and walk through the old town. Next, they take a sightseeing tour by tram. The Ringstrasse circles the center of the city. They use the regular tram and can admire the magnificent buildings as they pass by. They see the Votive Church, Vienna City Hall, the Burgtheater, the Parliament, two large museums, and the Vienna State Opera.

The children finally want to go to the Prater. They don’t want to look at buildings anymore. The Prater is an amusement park. Here, they ride the merry-go-round, bumper cars, and the giant Ferris wheel. That is also one of the city’s landmarks. Frank wants to go to the Schweizerhaus, a large beer garden. There, they take a break and eat Wiener Schnitzel.

They also wanted to go to the beautiful Schönbrunn Palace and to the zoo. But there isn’t enough time for that. They will do it on their next visit to Vienna.




\vspace{1cm}

% Question 2
\section*{Q.2 Bilden Sie das Präteritum und das Perfekt (nur 10)! (10 marks)}

(a) Das Mädchen bringt dem Gast eine Tasse Kaffee.\\
(b) Der Professor erklärt den Studenten die Sätze.\\
(c) Die Studentin spricht Deutsch sehr langsam.\\
(d) Monika schreibt ihrer Mutter einen Brief am Wochenende.\\
(e) Der Kaufmann kauft dem Kind ein Buch in Berlin.\\
(f) Diese Studenten kommen aus Japan.\\
(g) Die Lehrerin fährt mit ihren Studenten nach Deutschland.\\
(h) Der Amerikaner wohnt in Kalkutta ungefähr drei Jahre.\\
(i) Die Dame trinkt Tee ohne Milch und Zucker.\\
(j) Die Schülerin schenkt dem Lehrer einen Füller.\\
(k) Heidi und Robert arbeiten acht Stunden täglich bei Siemens.

\subsection*{Answer:}
\textcolor{darkgreen}{(a) Das Mädchen \textbf{brachte} dem Gast eine Tasse Kaffee. | Das Mädchen \textbf{hat} dem Gast eine Tasse Kaffee \textbf{gebracht}.}\\ \textcolor{red}{(Brignen - brachte - gebracht)}\\
\textcolor{darkgreen}{(b) Der Professor \textbf{erklärte} den Studenten die Sätze. | Der Professor \textbf{hat} den Studenten die Sätze \textbf{erklärt}.}\\ \textcolor{red}{(Erklären - erklärte - erklärt)}\\
\textcolor{darkgreen}{(c) Die Studentin \textbf{sprach} Deutsch sehr langsam. | Die Studentin \textbf{hat} Deutsch sehr langsam \textbf{gesprochen}.}\\ \textcolor{red}{(Sprechen - sprach - gesprochen)}\\
\textcolor{darkgreen}{(d) Monika \textbf{schrieb} ihrer Mutter einen Brief am Wochenende. | Monika \textbf{hat} ihrer Mutter einen Brief am Wochenende \textbf{geschrieben}.}\\ \textcolor{red}{(Schreiben - schrieb - geschrieben)}\\
\textcolor{darkgreen}{(e) Der Kaufmann \textbf{kaufte} dem Kind ein Buch in Berlin. | Der Kaufmann \textbf{hat} dem Kind ein Buch in Berlin \textbf{gekauft}.}\\ \textcolor{red}{(Kaufen - kaufte - gekauft)}\\
\textcolor{darkgreen}{(f) Diese Studenten \textbf{kamen} aus Japan. | Diese Studenten \textbf{sind} aus Japan \textbf{gekommen}.}\\ \textcolor{red}{(Kommen - kam - gekommen) (bin is used)}\\
\textcolor{darkgreen}{(g) Die Lehrerin \textbf{fuhr} mit ihren Studenten nach Deutschland. | Die Lehrerin \textbf{ist} mit ihren Studenten nach Deutschland \textbf{gefahren}.}\\ \textcolor{red}{(Fahren - fuhr - gefahren) (bin is used)}\\
\textcolor{darkgreen}{(h) Der Amerikaner \textbf{wohnte} in Kalkutta ungefähr drei Jahre. | Der Amerikaner \textbf{hat} in Kalkutta ungefähr drei Jahre \textbf{gewohnt}.}\\ \textcolor{red}{(Wohnen - wohnte - gewohnt)}\\
\textcolor{darkgreen}{(i) Die Dame \textbf{trank} Tee ohne Milch und Zucker. | Die Dame \textbf{hat} Tee ohne Milch und Zucker \textbf{getrunken}.}\\ \textcolor{red}{(Trinken - trank - getrunken)}\\
\textcolor{darkgreen}{(j) Die Schülerin \textbf{schenkte} dem Lehrer einen Füller. | Die Schülerin \textbf{hat} dem Lehrer einen Füller \textbf{geschenkt}.}\\ \textcolor{red}{(Schenken - schenkte - geschenkt)}\\
\textcolor{darkgreen}{(k) Heidi und Robert \textbf{arbeiteten} acht Stunden täglich bei Siemens. | Heidi und Robert \textbf{haben} acht Stunden täglich bei Siemens \textbf{gearbeitet}.}\\ \textcolor{red}{(Arbeiten - arbeitete - gearbeitet)}



\vspace{1cm}

% Question 3
\section*{Q.3 Setzen Sie richtige Präpositionen ein (nur 10)! (10 marks)}

(a) Diese Studenten kommen ----- Amerika.\\
(b) Der Hund sitzt ----- dem Tisch in dem Wohnzimmer.\\
(c) ----- 13:00 Uhr gehen sie zum Markt.\\
(d) Diese Taschen sind nur ----- die kleinen Kinder.\\
(e) Diese Studenten fahren ----- Frankreich.\\
(f) Das Schwimmbad liegt ----- der Schule.\\
(g) Die Bücher liegen ----- dem Stuhl.\\
(h) Viele Leute in Deutschland trinken Tee ----- Milch und Zucker.\\
(i) ----- drei Jahren arbeiteten diese Leute in Hamburg.\\
(j) ----- dem Frühstück hört die Frau immer die klassische Musik.\\
(k) Ein Gast steht ----- der Tür und fragt Frau Herbert.

\subsection*{Answer:}
\textcolor{darkgreen}{(a) Diese Studenten kommen \textbf{aus} Amerika.}\\
\textcolor{darkgreen}{(b) Der Hund sitzt \textbf{unter} dem Tisch in dem Wohnzimmer | Der Hund sitzt \textbf{auf} dem Tisch in dem Wohnzimmer.}\\
\textcolor{darkgreen}{(c) \textbf{Um} 13:00 Uhr gehen sie zum Markt.}\\
\textcolor{darkgreen}{(d) Diese Taschen sind nur \textbf{für} die kleinen Kinder.}\\
\textcolor{darkgreen}{(e) Diese Studenten fahren \textbf{nach} Frankreich.}\\
\textcolor{darkgreen}{(f) Das Schwimmbad liegt der Schule \textbf{gegenüber}.}\\
\textcolor{darkgreen}{(g) Die Bücher liegen \textbf{auf} dem Stuhl.}\\
\textcolor{darkgreen}{(h) Viele Leute in Deutschland trinken Tee \textbf{mit} Milch und Zucker. | Viele Leute in Deutschland trinken Tee \textbf{ohne} Milch und Zucker.}\\
\textcolor{darkgreen}{(i) \textbf{Vor} drei Jahren arbeiteten diese Leute in Hamburg. | \textbf{Seit} drei Jahren arbeiteten diese Leute in Hamburg.}\\
\textcolor{darkgreen}{(j) \textbf{Nach} dem Frühstück hört die Frau immer die klassische Musik.}\\
\textcolor{darkgreen}{(k) Ein Gast steht \textbf{an} der Tür und fragt Frau Herbert. | Ein Gast steht \textbf{vor} der Tür und fragt Frau Herbert.}


\vspace{1cm}

% Question 4
\section*{Q.4 Gebrauchen Sie die Modalverben! (10 marks)}

(a) Die Studenten spielen nur mit ihren Freunden.\\
(b) Die Dame verkauft ein großes Haus in Neu Delhi.\\
(c) Viele Ärzte arbeiten auch am Wochenende.\\
(d) Der Student schenkt der Lehrerin eine neue Uhr.\\
(e) Diese Studenten verstehen die Wörter nicht.\\
(f) Die Amerikanerin trinkt jetzt Tee mit Milch und Zucker.\\
(g) Versteht das Kind auch Deutsch?\\
(h) Seine Freundin kommt heute nach Frankfurt.\\
(i) Die Leute trinken viel Bier hier.\\
(j) Herbert fährt nach Haus mit dem Auto.

\subsection*{Answer:}
\textcolor{darkgreen}{(a) Die Studenten \textbf{wollen} nur mit ihren Freunden spielen.}\\
\textcolor{darkgreen}{(b) Die Dame \textbf{will} ein großes Haus in Neu Delhi verkaufen.}\\
\textcolor{darkgreen}{(c) Viele Ärzte \textbf{müssen} auch am Wochenende arbeiten.}\\
\textcolor{darkgreen}{(d) Der Student \textbf{will} der Lehrerin eine neue Uhr schenken.}\\
\textcolor{darkgreen}{(e) Diese Studenten \textbf{können} die Wörter nicht verstehen.}\\
\textcolor{darkgreen}{(f) Die Amerikanerin \textbf{will} jetzt Tee mit Milch und Zucker trinken.}\\
\textcolor{darkgreen}{(g) \textbf{Kann} das Kind auch Deutsch verstehen?}\\
\textcolor{darkgreen}{(h) Seine Freundin \textbf{will} heute nach Frankfurt kommen.}\\
\textcolor{darkgreen}{(i) Die Leute \textbf{wollen} hier viel Bier trinken.}\\
\textcolor{darkgreen}{(j) Herbert \textbf{soll} nach Haus mit dem Auto fahren.}
\vspace{1cm}

% Question 5
\section*{Q.5 Gebrauchen Sie passende Wörter! (10 marks)}

(a) Wohin fährt der Professor? Er fährt ----- Haus.\\
(b) Ohne Wasser kann ----- nicht leben.\\
(c) Er kommt nicht ----- . Vielleicht ist er krank.\\
(d) Sie fahren nach Hamburg mit dem ----- .\\
(e) Er hat ----- nicht mehr Zeit.\\
(f) Es gibt zehn Studenten in dem ----- .\\
(g) Wie lange ----- die Fahrt von Kharagpur bis Tatanagar?\\
(h) Die Studenten ----- die Wörter nicht.\\
(i) Viele ----- sprechen Deutsch gut.\\
(j) Ist sie Lehrerin? Nein, sie ist ----- .

\subsection*{Answer:}
\textcolor{darkgreen}{(a) Wohin fährt der Professor? Er fährt \textbf{nach} Haus.}\\
\textcolor{darkgreen}{(b) Ohne Wasser kann \textbf{man} nicht leben.}\\
\textcolor{darkgreen}{(c) Er kommt nicht \textbf{heute} . Vielleicht ist er krank.}\\
\textcolor{darkgreen}{(d) Sie fahren nach Hamburg mit dem \textbf{Zug}.}\\
\textcolor{darkgreen}{(e) Er hat \textbf{heute} nicht mehr Zeit.}\\
\textcolor{darkgreen}{(f) Es gibt zehn Studenten in dem \textbf{Kurs}.}\\
\textcolor{darkgreen}{(g) Wie lange \textbf{dauert} die Fahrt von Kharagpur bis Tatanagar?}\\
\textcolor{darkgreen}{(h) Die Studenten \textbf{verstehen} die Wörter nicht.}\\
\textcolor{darkgreen}{(i) Viele \textbf{Ausländer} sprechen Deutsch gut.}\\
\textcolor{darkgreen}{(j) Ist sie Lehrerin? Nein, sie ist \textbf{Kellnerin}.}\\

\vspace{1cm}

% Question 6
\section*{Q.6 Ergänzen Sie Adjektivendungen (nur 5 Sätze)! (10 marks)}

(a) Die amerikanisch--- Studentin wohnt in einer teuer--- Wohnung.\\
(b) Das schön--- Mädchen fährt in einem rote--- Auto.\\
(c) Alle klein--- Kinder haben ein neu--- Fahrrad.\\
(d) Die gut--- Studenten lesen immer die interessant--- Bücher.\\
(e) Viele neu--- Tische liegen in dem alt--- Haus.\\
(f) Eine deutsch--- Dame kauft den arm--- Kindern die Hefte.

\subsection*{Answer:}
\textcolor{darkgreen}{(a) Die \textbf{amerikanische} Studentin wohnt in einer teuren Wohnung.}\\
\textcolor{darkgreen}{(b) Das \textbf{schöne} Mädchen fährt in einem roten Auto.}\\
\textcolor{darkgreen}{(c) Alle \textbf{kleinen} Kinder haben ein neues Fahrrad.}\\
\textcolor{darkgreen}{(d) Die \textbf{guten} Studenten lesen immer die \textbf{interessanten} Bücher.}\\
\textcolor{darkgreen}{(e) Viele \textbf{neue} Tische liegen in dem \textbf{alten} Haus.}\\
\textcolor{darkgreen}{(f) Eine \textbf{deutsche} Dame kauft den \textbf{armen} Kindern die Hefte.}
\vspace{1cm}

% Question 7
\section*{Q.7 Übersetzen Sie ins Deutsche (nur 10 Sätze)! (10 marks)}

(a) He writes a letter to his mother at the weekend.\\
(b) I am sorry, I cannot help you.\\
(c) School children play football in the afternoon.\\
(d) Professor purchased lots of new books in Kolkata.\\
(e) Student will not come to the class today. He is sick.\\
(f) Some Germans speak Sanskrit very well.\\
(g) Have you already talked to your parents?\\
(h) Can you please explain this sentence once again?\\
(i) The film will start on Sunday at 4.00 pm.\\
(j) Children don't want to drink milk. They want chocolates.\\
(k) The teacher has explained the theory to him many times.\\
(l) They are very tired today and want to sleep now.

\subsection*{Answers}

\textcolor{darkgreen}{(a) Er schreibt am Wochenende einen Brief an seine Mutter.}\\
\textcolor{darkgreen}{(b) Es tut mir leid, ich kann dir nicht helfen.}\\
\textcolor{darkgreen}{(c) Schulkinder spielen am Nachmittag Fußball.}\\
\textcolor{darkgreen}{(d) Der Professor kaufte viele neue Bücher in Kalkutta.}\\
\textcolor{darkgreen}{(e) Der Student wird heute nicht zum Unterricht kommen. Er ist krank.}\\
\textcolor{darkgreen}{(f) Einige Deutsche sprechen sehr gut Sanskrit.}\\
\textcolor{darkgreen}{(g) Hast du schon mit deinen Eltern gesprochen?}\\
\textcolor{darkgreen}{(h) Kannst du bitte diesen Satz noch einmal erklären?}\\
\textcolor{darkgreen}{(i) Der Film wird am Sonntag um 16.00 Uhr beginnen.}\\
\textcolor{darkgreen}{(j) Kinder wollen keine Milch trinken. Sie wollen Schokolade.}\\
\textcolor{darkgreen}{(k) Der Lehrer hat ihm die Theorie viele Male erklärt.}\\
\textcolor{darkgreen}{(l) Sie sind heute sehr müde und wollen jetzt schlafen.}\\


\vspace{1cm}

% Question 8
\section*{Q.8 Schreiben Sie 10 Sätze über Ihre Familie oder über Ihr Institut! (10 marks)}

This section requires you to write ten sentences in German about your family or your institute. Here is an example of how you might structure this:

- Meine Familie wohnt in Kharagpur.
- Mein Vater arbeitet an der IIT Kharagpur.
- Meine Mutter ist Lehrerin.
- Ich habe zwei Geschwister.
- Wir feiern immer Diwali zusammen.
- Mein Institut ist sehr groß und schön.
- Es gibt viele Studenten aus verschiedenen Ländern hier.
- Die Professoren sind sehr freundlich und hilfsbereit.
- Wir haben viele Clubs und Aktivitäten im Institut.
- Ich liebe es, hier zu studieren.

\end{document}
