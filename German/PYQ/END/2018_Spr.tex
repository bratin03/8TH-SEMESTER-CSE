\documentclass[a4paper,12pt]{article}
\usepackage[utf8]{inputenc}
\usepackage[T1]{fontenc}
\usepackage{geometry}
\geometry{left=2.5cm, right=2.5cm, top=2.5cm, bottom=2.5cm}
\usepackage{amsmath}
\usepackage{enumitem}
\usepackage{setspace}
\usepackage{graphicx}
\usepackage{xcolor}
\definecolor{darkgreen}{rgb}{0.0, 0.5, 0.0}
\setstretch{1.2}

\begin{document}

\begin{center}
    {\LARGE \textbf{INDIAN INSTITUTE OF TECHNOLOGY KHARAGPUR}}\\[0.3cm]
    {\Large End-Spring Semester 2017-18}\\[0.5cm]
    \textbf{Date of Examination : \underline{\hspace{4cm}} \quad Session (FN/AN) \underline{\hspace{2cm}}}\\[0.3cm]
    Duration: 3 hrs \\
    Subject No.: HS30048 \quad Subject Name: German \\
    Full mark: 50 \\
    Department/Center/School: Humanities \& Social Sciences
\end{center}

\vspace{1cm}

\section*{1. Übersetzen Sie ins Englische!}

\begin{enumerate}[label=(\Alph*)]
    \item Es gibt vierzig Studenten in unserem Deutschunterricht. Da sind Männer und Frauen. Sie kommen aus allen Teilen Indiens und studieren Mathematik. Alle Studenten und Studentinnen sprechen gut Englisch. Englisch ist auch die Unterrichtssprache im IIT Kharagpur. Wir lernen Deutsch seit einem Semester. Wir können jetzt ein wenig Deutsch sprechen. Wir sehen einmal in der Woche einen deutschen Videofilm. Er heißt 'Extra'. Der Film ist sehr interessant. Wir lernen viele neue Wörter aus dem Film.
    \item Nishant ist Student und kommt aus Indien. Zwei Wochen ist er schon in Deutschland und lernt Deutsch. Sein Zimmer ist groß, und geräumig (spacious). Es hat einen Tisch, drei Stühle, zwei Lampen, ein Bett und einen Schrank. Seine Frau ist auch Studentin an der Universität in Ludwigsburg. Sie studiert Pädagogik und schreibt jetzt ihre Abschlußprüfung (final exam). Sie spricht fließend Deutsch. Sie hat viele Freunde. Nishant besucht auch ihre Freunde oft. Er studiert Medizin an der Universität Stuttgart. Er und seine Frau sprechen und üben zusammen Deutsch zu Hause. Ravi hat auch einen Freund. Er heißt Peter Bremen. Er kommt aus Dänemark. Nishant und Peter arbeiten zusammen und lernen Deutsch.
    \item Saiba Tosun kommt aus der Türkei, aus Ankara. Er arbeitet in der Bundesrepublik Deutschland. Er wohnt im Haus von Dino unten im Erdgeschoß (ground floor). Dino hat oben im dritten Stock ein Zimmer. Manchmal geht er zu Dino rauf, oder Dino kommt runter zu ihm. In jedem Stockwerk sind drei Wohnungen: eine Wohnung links, eine in der Mitte und eine rechts. Links neben ihm wohnt eine Familie aus der Schweiz. Er geht oft rüber zu ihnen. Über ihm wohnt eine Jugoslawin.
\end{enumerate}

\subsection*{Answer:}
\begin{enumerate}[label=(\Alph*)]
    \item There are forty students in our German class. There are men and women. They come from all parts of India and study mathematics. All the male and female students speak good English. English is also the language of instruction at IIT Kharagpur. We have been learning German for one semester. We can now speak a little German. Once a week we watch a German video film. It is called 'Extra'. The film is very interesting. We learn many new words from the film.

    \item Nishant is a student and comes from India. He has already been in Germany for two weeks and is learning German. His room is big and spacious. It has a table, three chairs, two lamps, a bed, and a wardrobe. His wife is also a student at the university in Ludwigsburg. She studies education and is currently taking her final exam. She speaks fluent German. She has many friends. Nishant also often visits her friends. He studies medicine at the University of Stuttgart. He and his wife speak and practice German together at home. Ravi also has a friend. His name is Peter Bremen. He comes from Denmark. Nishant and Peter work together and learn German.

    \item Saiba Tosun comes from Turkey, from Ankara. He works in the Federal Republic of Germany. He lives in Dino’s house, downstairs on the ground floor. Dino has a room upstairs on the third floor. Sometimes he goes up to Dino’s place, or Dino comes down to his. On each floor, there are three apartments: one apartment on the left, one in the middle, and one on the right. A family from Switzerland lives to the left of him. He often goes over to visit them. A Yugoslav woman lives above him.
\end{enumerate}


\section*{2. Bilden Sie das Präteritum und das Perfekt (nur 10)!}

\begin{enumerate}[label=(\alph*)]
    \item Der Lehrer erklärt den Studenten die Wörter.
    \item Der Student spricht Deutsch sehr langsam aber richtig.
    \item Peter Krein schreibt seinem Vater einen Brief.
    \item Diese Studenten kommen aus Japan.
    \item Der Professor fährt nach Deutschland mit seinen Studenten.
    \item Der Japener bleibt in Mumbai ungefähr ein Jahr.
    \item Viele Ausländer trinken Tee ohne Milch und Zucker.
    \item Das Mädchen bringt dem Gast eine Tasse Kaffee.
    \item Die Studenten finden das Buch in Neu Delhi.
    \item Die Schülerin schenkt der Lehrerin eine schöne Blume.
    \item Sam und Heidi arbeiten acht Stunden täglich.
\end{enumerate}

\section*{3. Setzen Sie richtige Präpositionen ein (nur 10)!}

\begin{enumerate}[label=(\alph*)]
    \item ------ 13.00 Uhr gehen wir zum Bahnhof.
    \item Diese Bücher sind nur --- die armen Leute.
    \item Diese Studenten kommen ---- Japan und studieren in Indien.
    \item ----- drei Jahren arbeitet er in Tokyo.
    \item Das Krankenhaus liegt der Schule \underline{\hspace{1cm}}
    \item Die Bücher liegen ---- dem Tisch.
    \item Viele Leute trinken Tee ----- Zucker.
    \item Die Studenten gehen jetzt ---- das Kino.
    \item Die Katze sitzt ----- dem Tisch in dem Wohnzimmer.
    \item ------ dem Essen liest er die Zeitung.
    \item Ein Herr steht ---- der Tür und fragt Herrn Kühn.
\end{enumerate}

\section*{4. Gebrauchen Sie die Modalverben!}

\begin{enumerate}[label=(\alph*)]
    \item Die Kinder spielen nur mit dem Computer.
    \item Viele Studenten arbeiten am Wochenende.
    \item Die Frau kauft dem Mann eine neue Uhr.
    \item Sie trinkt jetzt Tee mit Milch und Zucker.
    \item Spricht der Student schon gut Deutsch?
    \item Sein Vater kommt heute nach Kharagpur.
    \item Die Kinder spielen nur mit ihren Eltern.
    \item Viele Studenten verstehen die Theorie nicht.
    \item Die Leute trinken nur Tee hier.
    \item Peter fährt nach Haus mit dem Bus.
\end{enumerate}

\section*{5. Gebrauchen Sie passende Wörter!}

\begin{enumerate}[label=(\alph*)]
    \item Ich verstehe nicht. Können Sie bitte noch \underline{\hspace{3cm}} erklären?
    \item --------- fahren Sie nach Hamburg?
    \item Dieser Schüler versteht das Wort------.
    \item ------- habe ich nicht mehr Zeit.
    \item Es---- zehn Studenten in dem Zimmer.
    \item Wie---- dauert die Fahrt von Howrah bis Tatanagar?
    \item Wo ist der Professor? Er ist --- Haus.
    \item Ohne ----- kann man nicht leben.
    \item Viele Ausländer sprechen ----- gut.
    \item Sind Sie \underline{\hspace{1cm}}? Nein, ich bin Inder.
\end{enumerate}

\section*{6. Ergänzen Sie Adjektivendungen (nur 5 Sätze)!}

\begin{enumerate}[label=(\alph*)]
    \item Die alt\underline{\hspace{1cm}} Dame wohnt in einem groß\underline{\hspace{1cm}} Haus in Varanasi.
    \item Viele neu\underline{\hspace{1cm}} Bücher liegen in dem alt\underline{\hspace{1cm}} Schrank.
    \item Alle jung\underline{\hspace{1cm}} Arbeiter haben ein neu\underline{\hspace{1cm}} Auto.
    \item Die klein\underline{\hspace{1cm}} Kinder sollen den schrecklich\underline{\hspace{1cm}} Film sehen.
    \item Die ausländisch\underline{\hspace{1cm}} Studentin fährt mit einem schön\underline{\hspace{1cm}} Auto.
    \item Eine indisch\underline{\hspace{1cm}} Frau schenkt den klein\underline{\hspace{1cm}} Kindern Schokolade.
\end{enumerate}

\vspace{0.5cm}
\section*{7. Setzen Sie die Sätze ins Deutsche (nur 10 Sätze)!}

\noindent (a) Some students read a German newspaper in the evening.\\
(b) She has purchased new German books in Kolkata.\\
(c) Our examination will start from Thursday.\\
(d) Children don't want to play football in the morning.\\
(e) He is very late today; he is not well.\\
(f) Some foreigners speak Hindi very good.\\
(g) Have you already visited your friends?\\
(h) Can you please explain this word to me once again?\\
(i) I write a letter to my friend in the weekend.\\
(j) I am sorry, I cannot come with you.\\
(k) Professor has explained the theory to students many times.\\
(l) Today we are very much tired. We shall start our journey tomorrow.

\vspace{0.5cm}

\subsection*{Answer:}
\textcolor{darkgreen}
{(a) Einige Studenten lesen am Abend eine deutsche Zeitung.}\\
\textcolor{darkgreen}
{(b) Sie hat neue deutsche Bücher in Kalkutta gekauft.}\\
\textcolor{darkgreen}
{(c) Unsere Prüfung wird am Donnerstag beginnen.}\\
\textcolor{darkgreen}
{(d) Die Kinder wollen am Morgen nicht Fußball spielen.}\\
\textcolor{darkgreen}
{(e) Er ist heute sehr spät; er ist nicht wohl.}\\
\textcolor{darkgreen}
{(f) Einige Ausländer sprechen sehr gut Hindi.}\\
\textcolor{darkgreen}
{(g) Hast du deine Freunde schon besucht?}\\
\textcolor{darkgreen}
{(h) Können Sie mir dieses Wort bitte noch einmal erklären?}\\
\textcolor{darkgreen}
{(i) Ich schreibe am Wochenende einen Brief an meinen Freund.}\\
\textcolor{darkgreen}
{(j) Es tut mir leid, ich kann nicht mit dir kommen.}\\
\textcolor{darkgreen}
{(k) Der Professor hat den Studenten die Theorie viele Male erklärt.}\\
\textcolor{darkgreen}
{(l) Heute sind wir sehr müde. Wir werden morgen unsere Reise beginnen.}\\

\end{document}
