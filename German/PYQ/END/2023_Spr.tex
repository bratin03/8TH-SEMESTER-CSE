\documentclass[a4paper,12pt]{article}
\usepackage[utf8]{inputenc}
\usepackage{geometry}
\geometry{margin=1in}
\usepackage{amsmath}
\usepackage{amssymb}
\usepackage{titlesec}
\usepackage{xcolor}
\definecolor{darkgreen}{rgb}{0.0, 0.5, 0.0}

% Formatting for sections
\titleformat{\section}{\normalfont\Large\bfseries}{\thesection}{1em}{}
\titleformat{\subsection}{\normalfont\large\bfseries}{\thesubsection}{1em}{}

\begin{document}

% Title
\begin{center}
    \textbf{INDIAN INSTITUTE OF TECHNOLOGY KHARAGPUR}\\
    \textbf{End-Spring Semester Examination 2023-24}\\
    \vspace{0.5cm}
    \begin{tabular}{l l}
        Date of Examination: & \hspace{2cm} \\
        Session: & FN/AN \\
        Duration: & 3 hrs \\
        Subject No.: & HS30048 \\
        Subject Name: & German \\
        Department/Center/School: & Humanities \& Social Sciences \\
        Full Marks: & 100/2 \\
    \end{tabular}
\end{center}

\vspace{1cm}

% Question 1
\section*{1. Übersetzen Sie ins Englische!}

(A) Oscar geht einkaufen in einen großen Supermarkt. Seine Einkaufsliste ist lang, er kauft für das ganze \textcolor{red}{entire} Wochenende ein. Außerdem kommen Gäste, für die er kochen wird. Beim Obstregal \textcolor{red}{fruit shelf} kauft er verschiedene Früchte: Äpfel, Bananen, Erdbeeren \textcolor{red}{strawberries} und Kirschen \textcolor{red}{cherries} wird er für den Nachtisch \textcolor{red}{dessert} verwenden, es gibt Obstsalat. Die Trauben verwendet er für die Vorspeise \textcolor{red}{starter}. Er möchte gerne kleine Spieße \textcolor{red}{skewers} mit Käse und Trauben \textcolor{red}{grapes} anbieten.

Mit dem Gemüse kocht er eine Suppe. Dafür braucht er ein Kilo Karotten, einige große Kartoffeln, ein halbes Kilo Zwiebeln \textcolor{red}{onions} und verschiedene Pilze \textcolor{red}{mushrooms}. Er findet   und getrocknete \textcolor{red}{dried} Steinpilze \textcolor{red}{porcini mushrooms}. Diese eignen sich sehr gut für eine Suppe. Außerdem nimmt er grünen \textcolor{red}{green} Salat und Tomaten mit für die zweite Vorspeise.

Im ersten Kühlregal \textcolor{red}{refrigerated section} gibt es eine große Auswahl \textcolor{red}{selection} an Fleisch und Fisch. Oscar entscheidet sich für ein Huhn. Er kauft zusätzlich noch eine Packung Reis als Beilage zum Fleisch. Damit hat er bereits alles, was er für das Essen braucht. Er nimmt aber einiges mit, das ihm zu Hause fehlt: einen großen Laib Brot, ein halbes Kilo Salz, ein Kilo Mehl und zwei Kilo Zucker findet er neben dem Kühlregal. Dort nimmt er auch eine Flasche Milch mit. Was ihm jetzt noch fehlt: Käse und zehn Eier. Die findet er auch im Kühlregal um die Ecke.

(B) Ich habe sehr viele Verwandte \textcolor{red}{relatives}, die ich auch fast alle kenne. Meine Familie ist groß, weil meine Eltern beide \textcolor{red}{both} viele Geschwister haben. Meine Mutter hat vier Schwestern. Sie ist die jüngste. Die Kinder meiner Tanten sind meine Cousinen und Cousins. Sie leben nicht alle in meiner Umgebung \textcolor{red}{vicinity}, manche \textcolor{red}{some} wohnen sehr weit entfernt. Aber zu Familienfeiern kommen meistens alle angereist.

Auch mein Vater hat nicht nur einen Bruder oder eine Schwester, sondern insgesamt 4 Geschwister. Mein ältester Onkel ist aber schon  \textcolor{red}{died}. Er ist auch der einzige \textcolor{red}{only}, der keine Kinder hatte. Ich habe auch von der Seite meines Vaters viele Cousins und Cousinen. Aber meine liebste Cousine ist die Tochter meiner ältesten Tante, der Schwester meiner Mutter.

Ich mag alle Kinder meiner Onkel und Tanten, ob Söhne oder Töchter. Wir verstehen uns sehr gut, auch wenn wir sehr unterschiedlich sind. Meine Onkel und Tanten kümmern \textcolor{red}{take care} sich sehr um die anderen in der Familie. Auch um mich und meine Geschwister als ihre Nichten und Neffen. Ich habe eine Schwester und einen Bruder. Ich bin die mittlere. Meine Großeltern sind nicht mehr alle am Leben. Meine Großmutter ist schon 90 Jahre alt. Ihr Ehemann, also mein Großvater, ist vor zwei Jahren gestorben. Er war schon 93. Sie haben sehr viele Enkel \textcolor{red}{grandchildren} und Enkelinnen.

\subsection*{Answer}

\textbf{(A)} Oscar goes shopping in a large supermarket. His shopping list is long; he is shopping for the entire weekend. He is also expecting guests, for whom he will cook. In the fruit section, he buys various fruits: apples, bananas, strawberries, and cherries, which he will use for dessert—he’s making a fruit salad. He’ll use the grapes for the starter. He would like to offer small skewers with cheese and grapes.

With the vegetables, he is going to make a soup. For that, he needs a kilo of carrots, some large potatoes, half a kilo of onions, and various mushrooms. He finds champignons and dried porcini mushrooms. These are very good for soup. He also takes green salad and tomatoes for the second starter.

In the first refrigerated section, there is a wide selection of meat and fish. Oscar chooses a chicken. He also buys a packet of rice to go with the meat. With that, he already has everything he needs for the meal. But he also picks up a few things he is missing at home: a large loaf of bread, half a kilo of salt, a kilo of flour, and two kilos of sugar, which he finds next to the refrigerated section. There, he also grabs a bottle of milk. What he’s still missing: cheese and ten eggs. He finds those as well in the refrigerated section around the corner.

\vspace{1em}

\textbf{(B)} I have a lot of relatives, and I know almost all of them. My family is large because both of my parents have many siblings. My mother has four sisters. She is the youngest. The children of my aunts are my cousins. They don’t all live nearby—some live very far away. But for family celebrations, most of them travel to attend.

My father also has more than one sibling—he has four in total. My oldest uncle has already passed away. He was also the only one who didn’t have children. I also have many cousins on my father’s side. But my favorite cousin is the daughter of my oldest aunt, my mother’s sister.

I like all the children of my uncles and aunts, whether sons or daughters. We get along very well, even if we are very different. My uncles and aunts take great care of everyone in the family—including me and my siblings, their nieces and nephews. I have a sister and a brother. I am the middle child. Not all my grandparents are still alive. My grandmother is already 90 years old. Her husband—my grandfather—died two years ago. He was already 93. They have many grandchildren.


\vspace{1cm}

% Question 2
\section*{2. Bilden Sie das Präteritum und das Perfekt (nur 10)!}

(a) Viele Studenten kommen aus Deutschland.\\
(b) Der Professor fährt nach Amerika mit seinen Studenten.\\
(c) Der Japaner bleibt in Mumbai ungefähr ein Jahr.\\
(d) Viele Ausländer trinken Tee mit Milch und Zucker.\\
(e) Die Frau bringt dem Gast eine Tasse Kaffee.\\
(f) Die Lehrerin erklärt den Studenten die Wörter.\\
(g) Der Student spricht Deutsch sehr langsam aber richtig.\\
(h) Peter Klein schreibt seiner Mutter einen Brief.\\
(i) Die Studenten finden das Buch in Neu Delhi.\\
(j) Die Schülerin schenkt dem Lehrer eine schöne Blume.\\
(k) Johan und Heidi arbeiten sieben Stunden täglich.

\subsection*{Answer:}
\textcolor{darkgreen}{(a) Viele Studenten \textbf{kamen} aus Deutschland. | Viele Studenten \textbf{sind} aus Deutschland \textbf{gekommen}.}\\ \textcolor{red}{( Kommen - kam - gekommen) (bin is used)}\\
\textcolor{darkgreen}{(b) Der Professor \textbf{fuhr} nach Amerika mit seinen Studenten. | Der Professor \textbf{ist} nach Amerika mit seinen Studenten \textbf{gefahren}.}\\ \textcolor{red}{(Fahren - fuhr - gefahren) (bin is used)}\\
\textcolor{darkgreen}{(c) Der Japaner \textbf{blieb} in Mumbai ungefähr ein Jahr. | Der Japaner \textbf{ist} in Mumbai ungefähr ein Jahr \textbf{geblieben}.}\\ \textcolor{red}{(Bleiben - blieb - geblieben) (bin is used)}\\
\textcolor{darkgreen}{(d) Viele Ausländer \textbf{tranken} Tee mit Milch und Zucker. | Viele Ausländer \textbf{haben} Tee mit Milch und Zucker \textbf{getrunken}.}\\ \textcolor{red}{(Trinken - trank - getrunken)}\\
\textcolor{darkgreen}{(e) Die Frau \textbf{brachte} dem Gast eine Tasse Kaffee. | Die Frau \textbf{hat} dem Gast eine Tasse Kaffee \textbf{gebracht}.}\\ \textcolor{red}{(Bringen - brachte - gebracht)}\\
\textcolor{darkgreen}{(f) Die Lehrerin \textbf{erklärte} den Studenten die Wörter. | Die Lehrerin \textbf{hat} den Studenten die Wörter \textbf{erklärt}.}\\ \textcolor{red}{(Erklären - erklärte - erklärt)}\\
\textcolor{darkgreen}{(g) Der Student \textbf{sprach} Deutsch sehr langsam aber richtig. | Der Student \textbf{hat} Deutsch sehr langsam aber richtig \textbf{gesprochen}.}\\ \textcolor{red}{(Sprechen - sprach - gesprochen)}\\
\textcolor{darkgreen}{(h) Peter Klein \textbf{schrieb} seiner Mutter einen Brief. | Peter Klein \textbf{hat} seiner Mutter einen Brief \textbf{geschrieben}.}\\ \textcolor{red}{(Schreiben - schrieb - geschrieben)}\\
\textcolor{darkgreen}{(i) Die Studenten \textbf{fanden} das Buch in Neu Delhi. | Die Studenten \textbf{haben} das Buch in Neu Delhi \textbf{gefunden}.}\\ \textcolor{red}{(Finden - fand - gefunden)}\\
\textcolor{darkgreen}{(j) Die Schülerin \textbf{schenkte} dem Lehrer eine schöne Blume. | Die Schülerin \textbf{hat} dem Lehrer eine schöne Blume \textbf{geschenkt}.}\\ \textcolor{red}{(Schenken - schenkte - geschenkt)}\\
\textcolor{darkgreen}{(k) Johan und Heidi \textbf{arbeiteten} sieben Stunden täglich. | Johan und Heidi \textbf{haben} sieben Stunden täglich \textbf{gearbeitet}.}\\ \textcolor{red}{(Arbeiten - arbeitete - gearbeitet)}\\



\vspace{1cm}

% Question 3
\section*{3. Setzen Sie richtige Präpositionen ein (nur 10)!}

(a) Das Krankenhaus liegt der Schule ----- \\
(b) Die Bücher liegen ----- dem Tisch.\\
(c) Viele Leute trinken Tee ----- Zucker.\\
(d) Die Studenten gehen jetzt ----- das Kino.\\
(e) Die Katze sitzt ----- dem Tisch in dem Wohnzimmer.\\
(f) ----- 13.00 Uhr gehen wir zum Bahnhof.\\
(g) Diese Bücher sind nur ----- die armen Kinder.\\
(h) Diese Studenten kommen ----- Frankreich und studieren in Indien.\\
(i) ----- drei Jahren arbeitet er in Tokyo.\\
(j) ----- dem Essen liest er immer die Zeitung.\\
(k) Ein Herr steht ----- der Tür und fragt Herrn Schmid.

\subsection*{Answer:}
\textcolor{darkgreen}{(a) Das Krankenhaus liegt der Schule \textbf{gegenüber}.}\\
\textcolor{darkgreen}{(b) Die Bücher liegen \textbf{auf} dem Tisch.}\\
\textcolor{darkgreen}{(c) Viele Leute trinken Tee \textbf{mit} Zucker. | Viele Leute trinken Tee \textbf{ohne} Zucker.}\\
\textcolor{darkgreen}{(d) Die Studenten gehen jetzt \textbf{in} das Kino.}\\
\textcolor{darkgreen}{(e) Die Katze sitzt \textbf{unter} dem Tisch in dem Wohnzimmer.}\\
\textcolor{darkgreen}{(f) \textbf{Um} 13.00 Uhr gehen wir zum Bahnhof.}\\
\textcolor{darkgreen}{(g) Diese Bücher sind nur \textbf{für} die armen Kinder.}\\
\textcolor{darkgreen}{(h) Diese Studenten kommen \textbf{aus} Frankreich und studieren in Indien.}\\
\textcolor{darkgreen}{(i) \textbf{Vor} drei Jahren arbeitet er in Tokyo. | \textbf{Seit} drei Jahren arbeitet er in Tokyo.}\\
\textcolor{darkgreen}{(j) \textbf{Nach} dem Essen liest er immer die Zeitung.}\\
\textcolor{darkgreen}{(k) Ein Herr steht \textbf{an} der Tür und fragt Herrn Schmid. | Ein Herr steht \textbf{vor} der Tür und fragt Herrn Schmid.}\\

\vspace{1cm}

% Question 4
\section*{4. Gebrauchen Sie die Modalverben!}

(a) Sie trinkt jetzt Tee mit Milch und Zucker.\\
(b) Spricht die Studentin schon gut Deutsch?\\
(c) Mein Bruder kommt heute nach Kharagpur.\\
(d) Die Kinder spielen nur mit ihren Eltern.\\
(e) Die Dame kauft ein großes Haus in Berlin.\\
(f) Viele Studenten arbeiten am Wochenende.\\
(g) Die Frau schenkt dem Mann eine neue Uhr.\\
(h) Viele Studenten verstehen die Theorie nicht.\\
(i) Die Leute trinken nur Kaffee hier.\\
(j) Peter fährt nach Haus mit dem Zug \textcolor{magenta}{(Train)}.

\subsection*{Answer:}
\textcolor{darkgreen}{(a) Sie \textbf{will} jetzt Tee mit Milch und Zucker trinken.}\\
\textcolor{darkgreen}{(b) \textbf{Kann} die Studentin schon gut Deutsch sprechen?}\\
\textcolor{darkgreen}{(c) Mein Bruder \textbf{will} heute nach Kharagpur kommen.}\\
\textcolor{darkgreen}{(d) Die Kinder \textbf{dürfen} nur mit ihren Eltern spielen.}\\
\textcolor{darkgreen}{(e) Die Dame \textbf{will} ein großes Haus in Berlin kaufen.}\\
\textcolor{darkgreen}{(f) Viele Studenten \textbf{müssen} am Wochenende arbeiten.}\\
\textcolor{darkgreen}{(g) Die Frau \textbf{will} dem Mann eine neue Uhr schenken.}\\
\textcolor{darkgreen}{(h) Viele Studenten \textbf{können} die Theorie nicht verstehen.}\\
\textcolor{darkgreen}{(i) Die Leute \textbf{mögen} nur Kaffee hier trinken.}\\
\textcolor{darkgreen}{(j) Peter \textbf{kann} nach Haus mit dem Zug fahren.}\\
\

\vspace{1cm}

% Question 5
\section*{5. Gebrauchen Sie passende Wörter!}

(a) ------- habe ich nicht mehr Zeit.\\
(b) Es---- zehn Studenten in dem Zimmer.\\
(c) Wie---- dauert die Fahrt von Howrah bis Tatanagar?\\
(d) Wo ist der Professor? Er ist --- Haus.\\
(e) Ohne ----- kann man nicht leben.\\
(f) Ich verstehe nicht. Können Sie bitte noch ------- erklären?\\
(g) --------- fahren Sie nach Hamburg?\\
(h) Dieser Schüler versteht das Wort------.\\
(i) Viele Ausländer sprechen ----- gut.\\ 
(j) Sind Sie ---------. Nein, ich bin Inder.

\subsection*{Answer:}
\textcolor{darkgreen}{(a) \textbf{Jetzt} habe ich nicht mehr Zeit.}\\
\textcolor{darkgreen}{(b) Es \textbf{gibt} zehn Studenten in dem Zimmer.}\\
\textcolor{darkgreen}{(c) Wie \textbf{lange} dauert die Fahrt von Howrah bis Tatanagar?}\\
\textcolor{darkgreen}{(d) Wo ist der Professor? Er ist \textbf{im} Haus.}\\
\textcolor{darkgreen}{(e) Ohne \textbf{Wasser} kann man nicht leben.}\\
\textcolor{darkgreen}{(f) Ich verstehe nicht. Können Sie bitte noch \textbf{einmal} erklären?}\\
\textcolor{darkgreen}{(g) \textbf{Wann} fahren Sie nach Hamburg?}\\
\textcolor{darkgreen}{(h) Dieser Schüler versteht das Wort \textbf{nicht}.}\\
\textcolor{darkgreen}{(i) Viele Ausländer sprechen \textbf{Deutsch} gut.}\\
\textcolor{darkgreen}{(j) Sind Sie \textbf{Deutscher}? Nein, ich bin Inder.}\\


\vspace{1cm}

% Question 6
\section*{6. Ergänzen Sie Adjektivendungen (nur 5 Sätze)!}

(a) Alle jung-- Arbeiter haben ein neu -- Auto.\\
(b) Die klein-- Kinder sollen den schrecklich-- Film sehen.\\
(c) Die ausländisch -- Studentin fährt mit einem schön-- Auto.\\
(d) Die alt-- Dame wohnt in einem groß-- Haus in Varanasi.\\
(e) Viele neu -- Bücher liegen in dem alt -- Schrank.\\
(f) Eine indisch-- Frau schenkt den klein-- Kindern Schokolade.

\subsection*{Answer:}
\textcolor{darkgreen}{(a) Alle \textbf{jungen} Arbeiter haben ein \textbf{neues} Auto.}\\
\textcolor{darkgreen}{(b) Die \textbf{kleinen} Kinder sollen den \textbf{schrecklichen} Film sehen.}\\
\textcolor{darkgreen}{(c) Die \textbf{ausländische} Studentin fährt mit einem \textbf{schönen} Auto.}\\
\textcolor{darkgreen}{(d) Die \textbf{alte} Dame wohnt in einem \textbf{großen} Haus in Varanasi.}\\
\textcolor{darkgreen}{(e) Viele \textbf{neue} Bücher liegen in dem \textbf{alten} Schrank.}\\
\textcolor{darkgreen}{(f) Eine \textbf{indische} Frau schenkt den \textbf{kleinen} Kindern Schokolade.}\\




\vspace{1cm}

% Question 7
\section*{7. Übersetzen Sie ins Deutsche (nur 10 Sätze)!}

(a) The teacher is very late today; he is not well.\\
(b) Some foreigners speak Hindi very good.\\
(c) Have you already visited your friends?\\
(d) Can you please explain this word to me once again?\\
(e) I write a letter to my friend in the weekend.\\
(f) I am sorry, I cannot come with you.\\
(g) Some students read a German newspaper in the evening.\\
(h) She has purchased new German books in Mannheim.\\
(i) Our examination will start from Monday.\\
(j) Children don't want to play football in the morning.\\
(k) Professor has explained the theory to students many times.\\
(l) Today we are tired. We shall start our journey tomorrow.

\subsection*{Answer:}
\textcolor{darkgreen}{(a) Der Lehrer ist heute sehr spät; er ist nicht wohl.}\\
\textcolor{darkgreen}{(b) Einige Ausländer sprechen Hindi sehr gut.}\\
\textcolor{darkgreen}{(c) Hast du schon deine Freunde besucht?}\\
\textcolor{darkgreen}{(d) Kannst du mir bitte dieses Wort noch einmal erklären?}\\
\textcolor{darkgreen}{(e) Ich schreibe meinen Freund am Wochenende einen Brief.}\\
\textcolor{darkgreen}{(f) Es tut mir leid, ich kann nicht mit dir kommen.}\\
\textcolor{darkgreen}{(g) Einige Studenten lesen am Abend eine deutsche Zeitung.}\\
\textcolor{darkgreen}{(h) Sie hat neue deutsche Bücher in Mannheim gekauft.}\\
\textcolor{darkgreen}{(i) Unsere Prüfung wird am Montag beginnen.}\\
\textcolor{darkgreen}{(j) Kinder wollen am Morgen nicht Fußball spielen.}\\
\textcolor{darkgreen}{(k) Professor hat die Theorie den Studenten Mehrmals erklärt.}\\
\textcolor{darkgreen}{(l) Heute sind wir müde. Wir werden unsere Reise morgen beginnen.}\\
\

\vspace{1cm}

% Question 8
\section*{8. Schreiben Sie 10 Sätze über Ihre Familie oder über Ihr Institut!}

This section requires you to write ten sentences in German about your family or your institute. Here is an example of how you might structure this:

- Meine Familie wohnt in Kharagpur.
- Mein Vater arbeitet an der IIT Kharagpur.
- Meine Mutter ist Lehrerin.
- Ich habe zwei Geschwister.
- Wir feiern immer Diwali zusammen.
- Mein Institut ist sehr groß und schön.
- Es gibt viele Studenten aus verschiedenen Ländern hier.
- Die Professoren sind sehr freundlich und hilfsbereit.
- Wir haben viele Clubs und Aktivitäten im Institut.
- Ich liebe es, hier zu studieren.

\end{document}
