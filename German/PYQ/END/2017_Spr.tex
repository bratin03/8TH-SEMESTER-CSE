\documentclass[12pt]{article}
\usepackage[utf8]{inputenc}
\usepackage[T1]{fontenc}
\usepackage[ngerman,english]{babel}
\usepackage{amsmath}
\usepackage{geometry}
\usepackage{graphicx}
\usepackage{xcolor}
\definecolor{darkgreen}{rgb}{0.0, 0.5, 0.0}
\geometry{a4paper, margin=1in}

\title{
INDIAN INSTITUTE OF TECHNOLOGY KHARAGPUR
}

\begin{document}

\maketitle

\section*{End-Spring Semester 2016-17}

\begin{tabular}{ll}
Date of Examination: \underline{\hspace{4cm}} & Session (FN/AN): \underline{\hspace{2cm}} \\
Duration: 3 hrs & Subject No.: HS30048 \\
Subject Name: German & Full mark: 50 \\
Department/Center/School: Humanities \& Social Sciences &
\end{tabular}

\vspace{1em}

\section*{1. Übersetzen Sie ins Englische!}

Berlin: Die deutsche Hauptstadt will jeder sehen. Diese unglaubliche facettenreiche \textbf{red}{diverse} Stadt hat 3,4-Millionen Bewohner. Große deutsche Denkmäler wie das Brandenburger Tor und der Berliner Dom, Bauwerke wie der Alexanderplatz und die East Side Gallery, Kunstschätze wie der Pergamonaltar, aber auch Subkultur und Szene sind in der Stadt zu besichtigen.

Dresden: Diese Elbstadt hat mehr als Frauenkirche und Semperoper zu bieten. Zwar wurden im große Teile der Stadt zerstört, in ihrem historischen Kern präsentiert sie sich aber noch heute mit historischer Architektur aus der Renaissance oder dem Barock. Up to date sind hier einige der besten deutschen Restaurants und Hotels.

Frankfurt am Main: Das liegt nahe in der Stadt mit dem größten Flughafen Deutschlands. Fakt ist, dass „Mainhattan" sehr viel für seine Gäste zu bieten hat. Die Bankenstadt ist mit ihren vielen Museen international renommierte Kulturstadt. Das Nachtleben reicht vom gemütlichen Apfelweinschoppen bis zum alternativen Punkschuppen.

Ramsau: Dass es Touristen nicht nur in die großen deutschen Städte lockt, zeigt der achte Platz im Tripadvisor-Ranking der beliebtesten Reiseziele in Deutschland. Es bietet Deutschlandimpressionen wie aus dem Bilderbuch: eine spektakuläre Berglandschaft, unergründliche Seen, tiefe Wälder und blühende Wiesen. Der Freizeitwert ist unerreicht für Skifahrer und Snowboarder, Wanderer, Mountainbiker und Bergsteiger.
 
\textbf{VOKABULAR:} \\
unglaubliche = unbelievable; Denkmäler = monuments; Bauwerke = buildings; Kunstschätze = art treasures; besichtigen = to visit; bieten = to offer; zwar = indeed; Teile = parts; zerstört = destroyed; Kern = centre; Flughafen = airport; gemütlich = comfortable; lockt = lures; unergründliche = inexplicable; blühende = blooming; Wiesen = meadows; Freizeitwert = recreational value

\subsection*{Answers:}

Berlin: Everyone wants to see the German capital. This incredibly diverse city has 3.4 million inhabitants. Major German landmarks like the Brandenburg Gate and the Berlin Cathedral, architectural sights like Alexanderplatz and the East Side Gallery, artistic treasures like the Pergamon Altar, but also subculture and trendy scenes can be visited in the city.

Dresden: This city on the Elbe offers more than just the Church of Our Lady and the Semper Opera House. Although large parts of the city were destroyed, its historic core still presents itself today with architectural styles from the Renaissance or the Baroque period. Some of the best German restaurants and hotels are up to date here.

Frankfurt am Main: That’s no surprise in the city with Germany’s largest airport. The fact is, "Mainhattan" has a lot to offer its guests. The banking city is an internationally renowned cultural hub thanks to its many museums. The nightlife ranges from cozy apple wine pubs to alternative punk clubs.

Ramsau: The fact that tourists are not only drawn to large German cities is shown by its eighth place in the Tripadvisor ranking of the most popular travel destinations in Germany. It offers picture-perfect impressions of Germany: a spectacular mountain landscape, mysterious lakes, deep forests, and blooming meadows. Its recreational value is unmatched—for skiers and snowboarders, hikers, mountain bikers, and mountaineers.


\vspace{1em}

\section*{2. Bilden Sie das Präteritum und das Perfekt (nur 10)!}

\begin{enumerate}
    \item Der Ausländer spricht Hindi langsam.
    \item Robert schreibt seiner Mutter einen Brief.
    \item Diese Studentinnen kommen aus Berlin.
    \item Der Professor fährt nach Köln mit seinen Studenten.
    \item Der Amerikaner bleibt in Kharagpur ungefähr zwei Jahre.
    \item Die indischen Studenten trinken Tee mit Milch und Zucker.
    \item Das Mädchen bringt dem Gast die Zeitung.
    \item Mein Bruder studiert Chemie in München.
    \item Der Schüler schenkt der Lehrerin eine schöne Uhr.
    \item Jeder Arbeiter bei Siemens arbeitet 8 Stunden täglich.
    \item Der Professor erklärt den Studenten die Wörter.
\end{enumerate}

\subsection*{Answers:}
\begin{enumerate}
    \item \textcolor{darkgreen}{Der Ausländer sprach Hindi langsam. / Der Ausländer hat Hindi langsam gesprochen.} \textcolor{red}{Sprechen - Sprach - Gesprochen}
    \item \textcolor{darkgreen}{Robert schrieb seiner Mutter einen Brief. / Robert hat seiner Mutter einen Brief geschrieben.} \textcolor{red}{Schreiben - Schrieb - Geschrieben}
    \item \textcolor{darkgreen}{Diese Studentinnen kamen aus Berlin. / Diese Studentinnen sind aus Berlin gekommen.}\textcolor{red}{Kommen - Kam - Gekommen (sind)}
    \item \textcolor{darkgreen}{Der Professor fuhr nach Köln mit seinen Studenten. / Der Professor ist nach Köln mit seinen Studenten gefahren.} \textcolor{red}{Fahren - Fuhr - Gefahren (ist)}
    \item \textcolor{darkgreen}{Der Amerikaner blieb in Kharagpur ungefähr zwei Jahre. / Der Amerikaner ist in Kharagpur ungefähr zwei Jahre geblieben.} \textcolor{red}{Bleiben - Blieb - Geblieben (ist)}
    \item \textcolor{darkgreen}{Die indischen Studenten tranken Tee mit Milch und Zucker. / Die indischen Studenten haben Tee mit Milch und Zucker getrunken.} \textcolor{red}{Trinken - Trinkte - Getrunken}
    \item \textcolor{darkgreen}{Das Mädchen brachte dem Gast die Zeitung. / Das Mädchen hat dem Gast die Zeitung gebracht.} \textcolor{red}{Bringen - Brachte - Gebracht}
    \item \textcolor{darkgreen}{Mein Bruder studierte Chemie in München. / Mein Bruder hat Chemie in München studiert.} \textcolor{red}{Studieren - Studierte - Studiert}
    \item \textcolor{darkgreen}{Der Schüler schenkte der Lehrerin eine schöne Uhr. / Der Schüler hat der Lehrerin eine schöne Uhr geschenkt.} \textcolor{red}{Schenken - Schenkte - Geschenkt}
    \item \textcolor{darkgreen}{Jeder Arbeiter bei Siemens arbeitete 8 Stunden täglich. / Jeder Arbeiter bei Siemens hat 8 Stunden täglich gearbeitet.} \textcolor{red}{Arbeiten - Arbeitete - Gearbeitet}
    \item \textcolor{darkgreen}{Der Professor erklärte den Studenten die Wörter. / Der Professor hat den Studenten die Wörter erklärt.} \textcolor{red}{Erklären - Erklärte - Erklärt}
\end{enumerate}
\vspace{1em}

\section*{3. Ergänzen Sie Adjektivendungen (nur 5 Sätze)!}

\begin{enumerate}
    \item Der ausländisch\underline{\hspace{1cm}} Student hat viele gut\underline{\hspace{1cm}} Freunde.
    \item Ein reich\underline{\hspace{1cm}} Mann schenkt den arm\underline{\hspace{1cm}} Kindern die Bücher.
    \item Alle klein\underline{\hspace{1cm}} Kinder besuchen das deutsch\underline{\hspace{1cm}} Museum.
    \item Der amerikanisch\underline{\hspace{1cm}} Arbeiter hat ein groß\underline{\hspace{1cm}} Haus.
    \item Ein jung\underline{\hspace{1cm}} Mann kauft den klein\underline{\hspace{1cm}} Schülern Schokolade.
    \item Die alt\underline{\hspace{1cm}} Dame wohnt in einer schön\underline{\hspace{1cm}} Stadt.
\end{enumerate}

\subsection*{Answers:}
\begin{enumerate}
\item \textcolor{darkgreen}{Der \textbf{ausländische} Student hat viele \textbf{gute} Freunde.} \textcolor{red}{ausländisch - ausländische; gut - gute}
\item \textcolor{darkgreen}{Ein \textbf{reicher} Mann schenkt den \textbf{armen} Kindern die Bücher.} \textcolor{red}{reich - reicher; arm - armen}
\item \textcolor{darkgreen}{Alle \textbf{kleinen} Kinder besuchen das \textbf{deutsche} Museum.} \textcolor{red}{klein - kleinen; deutsch - deutsche}
\item \textcolor{darkgreen}{Der \textbf{amerikanische} Arbeiter hat ein \textbf{großes} Haus.} \textcolor{red}{amerikanisch - amerikanische; groß - großes}
\item \textcolor{darkgreen}{Ein \textbf{junger} Mann kauft den \textbf{kleinen} Schülern Schokolade.} \textcolor{red}{jung - jungen; klein - kleinen}
\item \textcolor{darkgreen}{Die \textbf{alte} Dame wohnt in einer \textbf{schönen} Stadt.} \textcolor{red}{alt - alte; schön - schönen}
\end{enumerate}

\vspace{1em}

\section*{4. Gebrauchen Sie die Modalverben!}

\begin{enumerate}
    \item Die Leute trinken nur Wein hier.
    \item Peter fährt nach Haus mit dem Zug.
    \item Spricht Maria schon gut Deutsch?
    \item Die Kinder spielen nur mit ihren Eltern.
    \item Viele Studenten verstehen die Sätze nicht.
\end{enumerate}
\subsection*{Answers:}
\begin{enumerate}
    \item \textcolor{darkgreen}{Die Leute \textbf{dürfen} nur Wein hier trinken.} \textcolor{red}{dürfen}
    \item \textcolor{darkgreen}{Peter \textbf{muss} nach Haus mit dem Zug fahren.} \textcolor{red}{müssen}
    \item \textcolor{darkgreen}{Kann Maria schon gut Deutsch sprechen?} \textcolor{red}{kann}
    \item \textcolor{darkgreen}{Die Kinder \textbf{sollen} nur mit ihren Eltern spielen.} \textcolor{red}{sollen}
    \item \textcolor{darkgreen}{Viele Studenten \textbf{können} die Sätze nicht verstehen.} \textcolor{red}{können}
\end{enumerate}


\vspace{1em}

\section*{5. Gebrauchen Sie passende Wörter!}

\begin{enumerate}
    \item Ich verstehe das Wort nicht. Können Sie bitte noch \underline{\hspace{2cm}} erklären?
    \item \underline{\hspace{2cm}} fahren wir nach Paris?
    \item Der Schüler versteht das Wort\underline{\hspace{2cm}}.
    \item \underline{\hspace{2cm}} haben wir nicht mehr Zeit.
    \item Ohne \underline{\hspace{2cm}} kann man nicht leben.
    \item \underline{\hspace{2cm}} ist verboten hier.
    \item Sind Sie \underline{\hspace{2cm}}? Nein, ich bin noch nicht.
    \item Es\underline{\hspace{2cm}} noch eine Stunde. Bitte warten Sie!
    \item Wie\underline{\hspace{2cm}} dauert die Fahrt von hier bis Kalkutta?
    \item Wo ist der Professor? Er ist \underline{\hspace{2cm}} Haus.
\end{enumerate}

\subsection*{Answers:}
\begin{enumerate}
    \item \textcolor{darkgreen}{Ich verstehe das Wort nicht. Können Sie bitte noch \textbf{einmal} erklären?} \textcolor{red}{einmal}
    \item \textcolor{darkgreen}{\textbf{Wann} fahren wir nach Paris?} \textcolor{red}{Wann}
    \item \textcolor{darkgreen}{Der Schüler versteht das Wort\textbf{ nicht}.} \textcolor{red}{nicht}
    \item \textcolor{darkgreen}{\textbf{Leider} haben wir nicht mehr Zeit.} \textcolor{red}{Leider}
    \item \textcolor{darkgreen}{Ohne \textbf{Wasser} kann man nicht leben.} \textcolor{red}{Wasser}
    \item \textcolor{darkgreen}{\textbf{Parken} ist verboten hier.} \textcolor{red}{Rauchen}
    \item \textcolor{darkgreen}{Sind Sie \textbf{da}? Nein, ich bin noch nicht.} \textcolor{red}{da}
    \item \textcolor{darkgreen}{Es\textbf{ dauert} noch eine Stunde. Bitte warten Sie!} \textcolor{red}{dauert}
    \item \textcolor{darkgreen}{Wie\textbf{ lange} dauert die Fahrt von hier bis Kalkutta?} \textcolor{red}{lange}
    \item \textcolor{darkgreen}{Wo ist der Professor? Er ist \textbf{im} Haus.} \textcolor{red}{im}
\end{enumerate}


\vspace{1em}

\section*{6. Setzen Sie richtige Präpositionen ein (nur 10)!}

\begin{enumerate}
    \item Eine Dame steht \underline{\hspace{2cm}} der Tür und fragt Herrn Brekle.
    \item Normalerweise legt sie die Bücher \underline{\hspace{2cm}} das Bett.
    \item Er arbeitet bis 20.00 Uhr und dann fährt \underline{\hspace{2cm}} Haus am Abend.
    \item Die Studenten gehen jetzt \underline{\hspace{2cm}} das Klassenzimmer.
    \item Die Katze sitzt \underline{\hspace{2cm}} dem Tisch in dem Wohnzimmer.
    \item \underline{\hspace{2cm}} 11.00 Uhr gehen wir zu dem Seminar.
    \item Diese Bücher sind nur \underline{\hspace{2cm}} die kleinen Kinder.
    \item Viele Studenten kommen \underline{\hspace{2cm}} Thailand und studieren in Indien.
    \item \underline{\hspace{2cm}} vier Jahren arbeitet mein Freund in Amerika.
    \item Das Krankenhaus liegt dem Schwimmbad \underline{\hspace{2cm}}.
    \item \underline{\hspace{2cm}} dem Essen liest er nur die deutschen Zeitungen.
\end{enumerate}

\subsection*{Answers:}
\begin{enumerate}
    \item \textcolor{darkgreen}{Eine Dame steht \textbf{vor} der Tür und fragt Herrn Brekle.} \textcolor{red}{an}
    \item \textcolor{darkgreen}{Normalerweise legt sie die Bücher \textbf{auf} das Bett.} \textcolor{red}{auf}
    \item \textcolor{darkgreen}{Er arbeitet bis 20.00 Uhr und dann fährt \textbf{nach} Haus am Abend.} \textcolor{red}{nach}
    \item \textcolor{darkgreen}{Die Studenten gehen jetzt \textbf{in} das Klassenzimmer.} \textcolor{red}{in}
    \item \textcolor{darkgreen}{Die Katze sitzt \textbf{unter} dem Tisch in dem Wohnzimmer.} \textcolor{red}{unter}
    \item \textcolor{darkgreen}{\textbf{Um} 11.00 Uhr gehen wir zu dem Seminar.} \textcolor{red}{Um}
    \item \textcolor{darkgreen}{Diese Bücher sind nur \textbf{für} die kleinen Kinder.} \textcolor{red}{für}
    \item \textcolor{darkgreen}{Viele Studenten kommen \textbf{aus} Thailand und studieren in Indien.} \textcolor{red}{aus}
    \item \textcolor{darkgreen}{\textbf{Vor} vier Jahren arbeitet mein Freund in Amerika.} \textcolor{red}{Vor}
    \item \textcolor{darkgreen}{Das Krankenhaus liegt dem Schwimmbad\textbf{} gegenüber.} \textcolor{red}{gegenüber}
    \item \textcolor{darkgreen}{\textbf{Nach} dem Essen liest er nur die deutschen Zeitungen.}  \textcolor{red}{Nach}
\end{enumerate}

\vspace{1em}

\section*{7. Übersetzen Sie ins Deutsche (nur 10 Sätze)!}

\begin{enumerate}
    \item Can you please explain this word to me?
    \item She writes a letter every day to her mother.
    \item I am sorry, I cannot come with you to watch the film.
    \item The book is costly, but very interesting.
    \item Professor explains the theory to students several times.
    \item My student has purchased many books from Calcutta.
    \item He drives the car very fast.
    \item We have got a lot of work, but unfortunately we don't have enough time.
    \item Please close the door before you go away.
    \item Sometimes students do not understand the sentences.
    \item Today many students want to study further in Germany.
    \item Have you already visited Paris?
\end{enumerate}

\subsection*{Answers:}
\begin{enumerate}
    \item \textcolor{darkgreen}{Können Sie mir  bitte dieses Wort erklären?}
    \item \textcolor{darkgreen}{Sie schreibt jeden Tag einen Brief an ihre Mutter.}
    \item \textcolor{darkgreen}{Es tut mir leid, ich kann nicht mit dir kommen, um den Film zu sehen.}
    \item \textcolor{darkgreen}{Das Buch ist teuer, aber sehr interessant.}
    \item \textcolor{darkgreen}{Der Professor erklärt den Studenten die Theorie mehrmals.}
    \item \textcolor{darkgreen}{Mein Student hat viele Bücher aus Kalkutta gekauft.}
    \item \textcolor{darkgreen}{Er fährt das Auto sehr schnell.}
    \item \textcolor{darkgreen}{Wir haben viel Arbeit, aber leider nicht genug Zeit.}
    \item \textcolor{darkgreen}{Bitte schließen Sie die Tür, bevor Sie weggehen.}
    \item \textcolor{darkgreen}{Manchmal verstehen die Studenten die Sätze nicht.}
    \item \textcolor{darkgreen}{Heute wollen viele Studenten weiter in Deutschland studieren.}
    \item \textcolor{darkgreen}{Hast du Paris schon besucht?}
\end{enumerate}


\end{document}
