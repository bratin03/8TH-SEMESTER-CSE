\documentclass[12pt]{article}
\usepackage[utf8]{inputenc}
\usepackage[T1]{fontenc}
\usepackage{amsmath}
\usepackage{geometry}
\geometry{a4paper, margin=1in}
\usepackage{graphicx}
\usepackage{xcolor} 
\definecolor{darkgreen}{rgb}{0.0, 0.5, 0.0}

\title{
INDIAN INSTITUTE OF TECHNOLOGY KHARAGPUR \\ Department of Humanities \& Social Sciences
}
\date{}

\begin{document}

\maketitle

\begin{tabular}{lcc} 
Date...............FN/AN & Time: \textbf{3} Hrs. Full Mark: 50 & No. of Students: \textbf{51} \\
End-Spring Semester, 2015-2016 & Dept.- HSS & Sub. No. HS30048 \\
$2^{\text{nd}}, 3^{\text{rd}} \& 4^{\text{th}}$ yr. B.Tech (H), M.Sc & & Sub Name: German \\
\end{tabular}

\vspace{1em}

\section*{1. Übersetzen Sie ins Englische!}

Als Herr Schmid das Haus verlassen \textcolor{red}{Leave} wollte, um den Frühzug zu erreichen, brachte ihm seine Frau einen Brief. „Vergiß \textcolor{red}{forget} nicht, diesen Brief einzuwerfen \textcolor{red}{drop}, bevor du ins Büro \textcolor{red}{office} gehst, damit Tante Ida ihn morgen noch bekommt \textcolor{red}{receives}! Dieser Brief ist sehr wichtig \textcolor{red}{important}!"

Aber Herr Schmid vergaß den Brief doch. Als er in der Stadt aus dem Zug stieg und sich beeilte, um pünktlich ins Büro zu kommen, hatte er den Brief noch in der Tasche. Er wollte gerade den Bahnhof verlassen, da klopfte ihm ein Herr auf die Schulter. „Denken Sie an den Brief!" sagte der Unbekannte. Während Herr Schmid zum nächsten Briefkasten ging, um den Brief einzuwerfen, rief schon wieder ein Fremder hinter ihm her: ,,Vergessen Sie nicht, Ihren Brief einzuwerfen!"

Nachdem er den Brief eingeworfen hatte, verließ er rasch \textcolor{red}{quickly} den Bahnhof. „Haben Sie schon an Ihren Brief gedacht?" rief ihm nach einigen Minuten eine freundliche Dame lächelnd nach. Herr Schmid wunderte sich darüber, dass ihn alle Leute an den Brief erinnerten, und fragte die Dame: ,Mein Gott \textcolor{red}{goodness}, woher wissen denn alle Leute, dass ich einen Brief einwerfen soll? Ich habe ihn doch schon längst eingeworfen." Da lachte die Dame und sagte: „Dann kann ich Ihnen ja auch den Zettel abmachen, der an Ihrem Mantel steckt." - Auf dem Zettel war geschrieben: ,„Bitte sagen Sie meinem Mann, dass er einen Brief einwerfen soll!"

\textbf{VOKABULAR:} \\
verlassen = to leave; erreichen = to catch; vergessen = to forget; einwerfen = to drop; steigen = to get down; sich beeilen = to rush; Tasche = pocket; klopfen = to tap; Schulter = shoulder; Unbekannte = stranger; Briefkasten = letter box; rasch = quickly; denken = to think; lächelnd = smiling; sich wundern = to be surprised; längst = long back; erinnern = to remind; lachen = to laugh; Zettel = piece of paper; abmachen = to take off; Mantel = coat; stecken = to pin; dass = that

\section*{1. Answer:}

As Mr. Schmid was about to leave the house to catch the early train, his wife brought him a letter.  
“Don’t forget to drop this letter in the mailbox before you go to the office, so that Aunt Ida still receives it tomorrow! This letter is very important!”

But Mr. Schmid forgot the letter after all. When he got off the train in the city and hurried to get to the office on time, he still had the letter in his pocket. Just as he was about to leave the station, a man tapped him on the shoulder.  
“Don’t forget the letter!” said the stranger.  

While Mr. Schmid walked to the nearest letter box to drop in the letter, another stranger called out behind him:  
“Don’t forget to drop in your letter!”

After he had posted the letter, he quickly left the station.  
“Have you already remembered your letter?” a friendly lady called after him with a smile a few minutes later.

Mr. Schmid was surprised that everyone was reminding him about the letter, and he asked the lady:  
“My goodness, how do all these people know that I have to post a letter? I already posted it a long time ago.”

Then the lady laughed and said:  
“Well, then I can take off the note that’s pinned to your coat.”  

— On the note it said:  
“Please tell my husband to post a letter!”


\vspace{1em}

\section*{2. Bilden Sie das Präteritum und das Perfekt (nur 10)!}

\begin{enumerate}
\item Er schreibt seiner Mutter einen Brief am Wochenende.
\item Dieser Mann braucht kein Auto heute.
\item Die Ausländer bleiben in Delhi ungefähr drei Monate.
\item Viele Inder fahren nach Deutschland im Winter.
\item Einige Studenten verstehen den Lehrer nicht.
\item Die Sekretärin kauft ein neues Auto heute.
\item Die Mutter schenkt ihrem Sohn eine schöne Uhr.
\item Die Studenten arbeiten in dem Institut und auch zu Haus.
\item Herr Martin trinkt Tee ohne Milch und Zucker.
\item Dieser Herr sucht einen Zigarettenautomaten hier in der Universität.
\item Die Dame kommt aus Japan mit ihren vier Kindern.
\end{enumerate}

\subsection*{Answer:}
\begin{enumerate}
\item \textcolor{darkgreen}{Er \textbf{schrieb} seiner Mutter einen Brief am Wochenende. Er hat seiner Mutter einen Brief am Wochenende \textbf{geschrieben}.} \textcolor{red}{Schreiben - Schrieb - Geschrieben}
\item \textcolor{darkgreen}{Dieser Mann \textbf{brauchte} kein Auto heute. Dieser Mann hat kein Auto heute \textbf{gebraucht}.} \textcolor{red}{Brauchen - Brauchte - Gebraucht}
\item \textcolor{darkgreen}{Die Ausländer \textbf{blieben} in Delhi ungefähr drei Monate. Die Ausländer sind in Delhi ungefähr drei Monate \textbf{geblieben}.} \textcolor{red}{Bleiben - Blieb - Geblieben} {sind is used}
\item \textcolor{darkgreen}{Viele Inder \textbf{fuhren} nach Deutschland im Winter. Viele Inder sind nach Deutschland im Winter \textbf{gefahren}.} \textcolor{red}{Fahren - Fuhr - Gefahren} {sind is used}
\item \textcolor{darkgreen}{Einige Studenten \textbf{verstanden} den Lehrer nicht. Einige Studenten haben den Lehrer nicht \textbf{verstanden}.} \textcolor{red}{Verstehen - Verstand - Verstanden}
\item \textcolor{darkgreen}{Die Sekretärin \textbf{kaufte} ein neues Auto heute. Die Sekretärin hat ein neues Auto heute \textbf{gekauft}.} \textcolor{red}{Kaufen - Kaufte - Gekauft}
\item \textcolor{darkgreen}{Die Mutter \textbf{schenkte} ihrem Sohn eine schöne Uhr. Die Mutter hat ihrem Sohn eine schöne Uhr \textbf{geschenkt}.} \textcolor{red}{Schenken - Schenkte - Geschenkt}
\item \textcolor{darkgreen}{Die Studenten \textbf{arbeiteten} in dem Institut und auch zu Haus. Die Studenten haben in dem Institut und auch zu Haus \textbf{gearbeitet}.} \textcolor{red}{Arbeiten - Arbeitete - Gearbeitet}
\item \textcolor{darkgreen}{Herr Martin \textbf{trank} Tee ohne Milch und Zucker. Herr Martin hat Tee ohne Milch und Zucker \textbf{getrunken}.} \textcolor{red}{Trinken - Trank - Getrunken}
\item \textcolor{darkgreen}{Dieser Herr \textbf{sucht} einen Zigarettenautomaten hier in der Universität. Dieser Herr hat einen Zigarettenautomaten hier in der Universität \textbf{gesucht}.} \textcolor{red}{Suchen - Suchte - Gesucht}
\item \textcolor{darkgreen}{Die Dame \textbf{kam} aus Japan mit ihren vier Kindern. Die Dame ist aus Japan mit ihren vier Kindern \textbf{gekommen}.} \textcolor{red}{Kommen - Kam - Gekommen} {ist is used}
\end{enumerate}

\section*{3. Setzen Sie richtige Präpositionen ein (nur 10)!}

\begin{enumerate}
\item Das Krankenhaus liegt der Schule \underline{\hspace{2cm}}
\item \underline{\hspace{2cm}} dem Essen geht er mit seinen Freunden spazieren.
\item Eine alte Dame steht \underline{\hspace{2cm}} der Tür und fragt Herrn Kühn.
\item 
\item \underline{\hspace{2cm}} einen guten Freund fährt er nie mit dem Zug.
\item Morgen fahren wir \underline{\hspace{2cm}} Haus mit dem Gast.
\item Er stellt das Buch \underline{\hspace{2cm}} dem Tisch in dem Wohnzimmer.
\item \underline{\hspace{2cm}} 13.00 Uhr gehen wir zum Theater.
\item Diese Bücher sind nur \underline{\hspace{2cm}} eine arme Schülerin.
\item Diese Studenten kommen \underline{\hspace{2cm}} Japan und studieren in Indien.
\item \underline{\hspace{2cm}} 2000 arbeitet er in Heidelberg bei Siemens.
\end{enumerate}

\subsection*{Answer:}
\begin{enumerate}
    \item \textcolor{darkgreen}{Das Krankenhaus liegt der Schule \textbf{gegenüber}.} \textcolor{red}{Gegenüber}
    \item \textcolor{darkgreen}{\textbf{Nach} dem Essen geht er mit seinen Freunden spazieren.} \textcolor{red}{Nach}
    \item \textcolor{darkgreen}{Eine alte Dame steht \textbf{vor} der Tür und fragt Herrn Kühn.} \textcolor{red}{An}
    \item \textcolor{darkgreen}{Die Studenten gehen jetzt \textbf{in} das Klassenzimmer.} \textcolor{red}{In}
    \item \textcolor{darkgreen}{\textbf{Ohne} einen guten Freund fährt er nie mit dem Zug.} \textcolor{red}{Ohne}
    \item \textcolor{darkgreen}{Morgen fahren wir \textbf{zu} Haus mit dem Gast.} \textcolor{red}{Zu}
    \item \textcolor{darkgreen}{Er stellt das Buch \textbf{bei} dem Tisch in dem Wohnzimmer.} \textcolor{red}{Bei}
    \item \textcolor{darkgreen}{\textbf{Um} 13.00 Uhr gehen wir zum Theater.} \textcolor{red}{Um}
    \item \textcolor{darkgreen}{Diese Bücher sind nur \textbf{für} eine arme Schülerin.} \textcolor{red}{Für}
    \item \textcolor{darkgreen}{Diese Studenten kommen \textbf{aus} Japan und studieren in Indien.} \textcolor{red}{Aus}
    \item \textcolor{darkgreen}{\textbf{Seit} 2000 arbeitet er in Heidelberg bei Siemens.} \textcolor{red}{Seit}
\end{enumerate}



\vspace{1em}

\section*{4. Ergänzen Sie Adjektivendungen (nur 5 Sätze)!}

\begin{enumerate}
\item Der jung\underline{\hspace{1cm}} Arbeiter hat ein neu\underline{\hspace{1cm}} Auto.
\item Eine indisch\underline{\hspace{1cm}} Frau schenkt den klein\underline{\hspace{1cm}} Kindern neue Bücher.
\item Keine klein\underline{\hspace{1cm}} Kinder sollen diesen schrecklich\underline{\hspace{1cm}} Film sehen.
\item Die alt\underline{\hspace{1cm}} Dame wohnt in einem groß\underline{\hspace{1cm}} Haus in Bangalore.
\item Die ausländisch\underline{\hspace{1cm}} Studentin fährt mit einem schön\underline{\hspace{1cm}} Auto.
\item Viele wichtig\underline{\hspace{1cm}} Bücher liegen in dem alt\underline{\hspace{1cm}} Schrank.
\end{enumerate}

\subsection*{Answer:}
\begin{enumerate}
    \item \textcolor{darkgreen}{Der \textbf{junge} Arbeiter hat ein \textbf{neues} Auto.} \textcolor{red}{Jung - Junge - Neu - Neues}
    \item \textcolor{darkgreen}{Eine \textbf{indische} Frau schenkt den \textbf{kleinen} Kindern neue Bücher.} \textcolor{red}{Indisch - Indische - Klein - Kleinen}
    \item \textcolor{darkgreen}{Keine \textbf{kleinen} Kinder sollen diesen \textbf{schrecklichen} Film sehen.} \textcolor{red}{Klein - Kleinen - Schrecklich - Schrecklichen}
    \item \textcolor{darkgreen}{Die \textbf{alte} Dame wohnt in einem \textbf{großen} Haus in Bangalore.} \textcolor{red}{Alt - Alte - Groß - Großen}
    \item \textcolor{darkgreen}{Die \textbf{ausländische} Studentin fährt mit einem \textbf{schönen} Auto.} \textcolor{red}{Ausländisch - Ausländische - Schön - Schönen}
    \item \textcolor{darkgreen}{Viele \textbf{wichtige} Bücher liegen in dem \textbf{alten} Schrank.} \textcolor{red}{Wichtig - Wichtige - Alt - Alten}
\end{enumerate}

\vspace{1em}

\section*{5. Gebrauchen Sie die Modalverben}

\begin{enumerate}
\item Die Kinder sehen nur diesen Film.
\item Viele Studenten verstehen diesen Satz nicht.
\item Die Leute trinken hier gern Tee und Kaffee.
\item Spricht diese Studentin schon gut Deutsch?
\item Die Studenten lesen die Zeitung abends.
\end{enumerate}

\subsection*{Answer:}
\begin{enumerate}
    \item \textcolor{darkgreen}{Die Kinder \textbf{dürfen} nur diesen Film sehen.} \textcolor{red}{Dürfen}
    \item \textcolor{darkgreen}{Viele Studenten \textbf{können} diesen Satz nicht verstehen.} \textcolor{red}{Können}
    \item \textcolor{darkgreen}{Die Leute \textbf{mögen} hier gern Tee und Kaffee.} \textcolor{red}{Mögen}
    \item \textcolor{darkgreen}{\textbf{Kann} diese Studentin schon gut Deutsch sprechen?} \textcolor{red}{Kann}
    \item \textcolor{darkgreen}{Die Studenten \textbf{sollen} die Zeitung abends lesen.} \textcolor{red}{Sollen}
    \end{enumerate}
\vspace{1em}

\section*{6. Gebrauchen Sie passende Wörter!}

\begin{enumerate}
\item Wo ist der Professor? Er fährt heute abend \underline{\hspace{2cm}} Haus.
\item \underline{\hspace{2cm}} haben wir nicht mehr Zeit.
\item Sind Sie \underline{\hspace{2cm}}. Nein, ich bin noch nicht.
\item Er versteht nicht. Können Sie bitte noch \underline{\hspace{2cm}} erklären?
\item Sind Sie \underline{\hspace{2cm}}. Nein, ich bin Deutscher.
\item Es \underline{\hspace{2cm}} zehn Studenten in dem Bus.
\item Wie \underline{\hspace{2cm}} dauert die Fahrt von Howrah bis Tatanagar?
\item \underline{\hspace{2cm}} fahren Sie nach Berlin?
\item Ohne \underline{\hspace{2cm}} kann man nicht leben.
\item Viele Ausländer sprechen \underline{\hspace{2cm}} gut.
\end{enumerate}
\subsection*{Answer:}
\begin{enumerate}
\item \textcolor{darkgreen}{Wo ist der Professor? Er fährt heute abend \textbf{nach} Haus.} \textcolor{red}{Nach}
\item \textcolor{darkgreen}{\textbf{Leider} haben wir nicht mehr Zeit.} \textcolor{red}{Leider}
\item \textcolor{darkgreen}{Sind Sie \textbf{hier}? Nein, ich bin noch nicht.} \textcolor{red}{Hier}
\item \textcolor{darkgreen}{Er versteht nicht. Können Sie bitte noch \textbf{einmal} erklären?} \textcolor{red}{Einmal}
\item \textcolor{darkgreen}{Sind Sie \textbf{Inder}? Nein, ich bin Deutscher.} \textcolor{red}{Inder} | \textcolor{darkgreen}{Sind Sie \textbf{Ausländer}? Nein, ich bin Deutscher.} \textcolor{red}{Ausländer}
\item \textcolor{darkgreen}{Es \textbf{sind} zehn Studenten in dem Bus.} \textcolor{red}{Sind}
\item \textcolor{darkgreen}{Wie \textbf{lange} dauert die Fahrt von Howrah bis Tatanagar?} \textcolor{red}{Lange}
\item \textcolor{darkgreen}{\textbf{Wann} fahren Sie nach Berlin?} \textcolor{red}{Wann}
\item \textcolor{darkgreen}{Ohne \textbf{Wasser} kann man nicht leben.} \textcolor{red}{Wasser}
\item \textcolor{darkgreen}{Viele Ausländer sprechen \textbf{Deutsch} gut.} \textcolor{red}{Deutsch}
\end{enumerate}


\vspace{1em}

\section*{7. Übersetzen Sie ins Deutsche (nur 10 Sätze)!}

\begin{enumerate}
\item Some students read a German newspaper after lunch.
\item She has brought a new dictionary from Delhi.
\item The examination will start from Monday.
\item We have got a lot of work, but we don't have enough time.
\item They have purchased a new car.
\item Have you already visited your mother?
\item In winter vacation, many students will visit German universities.
\item Professor explains the theory to students several times.
\item Please close the door before you go away.
\item He writes a letter every day to his brother.
\item I am sorry, I cannot help you.
\item Can you please explain these words to me once again?
\item The foreigner would like tea without milk and sugar.
\end{enumerate}

\subsection*{Answer:}
\begin{enumerate}
    \item \textcolor{darkgreen}{Einige Studenten lesen eine deutsche Zeitung nach dem Mittagessen.} \textcolor{red}{Einige - lesen - Zeitung - nach - dem - Mittagessen}
    \item \textcolor{darkgreen}{Sie hat ein neues Wörterbuch aus Delhi gebracht.} \textcolor{red}{neues - Wörterbuch - aus - Delhi - gebracht}
    \item \textcolor{darkgreen}{Die Prüfung beginnt ab Montag.} \textcolor{red}{Prüfung - beginnt - ab - Montag}
    \item \textcolor{darkgreen}{Wir haben viel Arbeit, aber wir haben nicht genug Zeit.} \textcolor{red}{viel - Arbeit - aber - nicht - genug - Zeit}
    \item \textcolor{darkgreen}{Sie haben ein neues Auto gekauft.} \textcolor{red}{gekauft - neues - Auto}
    \item \textcolor{darkgreen}{Hast du schon deine Mutter besucht?} \textcolor{red}{schon - besucht}
    \item \textcolor{darkgreen}{In den Winterferien werden viele Studenten deutsche Universitäten besuchen.} \textcolor{red}{In - den - Winterferien - werden - viele - deutsche - Universitäten - besuchen}
    \item \textcolor{darkgreen}{Der Professor erklärt den Studenten die Theorie mehrmals.} \textcolor{red}{erklärt - den - Studenten - die - Theorie - mehrmals}
    \item \textcolor{darkgreen}{Bitte schließe die Tür, bevor du gehst.} \textcolor{red}{Bitte - schließe - die - Tür - bevor - du - gehst}
    \item \textcolor{darkgreen}{Er schreibt jeden Tag einen Brief an seinen Bruder.} \textcolor{red}{schreibt - jeden - Tag - einen - Brief - an - seinen - Bruder}
    \item \textcolor{darkgreen}{Es tut mir leid, ich kann Ihnen nicht helfen.} \textcolor{red}{Es - tut - mir - leid - ich - kann - Ihnen - nicht - helfen}
    \item \textcolor{darkgreen}{Kannst du mir bitte diese Wörter noch einmal erklären?} \textcolor{red}{Kannst - du - mir - bitte - diese - Wörter - noch - einmal - erklären}
    \item \textcolor{darkgreen}{Der Ausländer mag Tee ohne Milch und Zucker.} \textcolor{red}{Der - Ausländer - mag - Tee - ohne - Milch - und - Zucker}
\end{enumerate}
\vspace{2em}

\noindent\textit{Signature of the Paper-Setter}

\end{document}
