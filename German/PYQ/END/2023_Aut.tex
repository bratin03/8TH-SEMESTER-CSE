\documentclass[a4paper,12pt]{article}
\usepackage[utf8]{inputenc}
\usepackage{geometry}
\geometry{margin=1in}
\usepackage{amsmath}
\usepackage{amssymb}
\usepackage{titlesec}
\usepackage{xcolor}
\definecolor{darkgreen}{rgb}{0.0, 0.5, 0.0}

% Formatting for sections
\titleformat{\section}{\normalfont\Large\bfseries}{\thesection}{1em}{}
\titleformat{\subsection}{\normalfont\large\bfseries}{\thesubsection}{1em}{}

\begin{document}

% Title
\begin{center}
    \textbf{INDIAN INSTITUTE OF TECHNOLOGY KHARAGPUR}\\
    \textbf{End-Autumn Semester Examination 2023-24}\\
    \vspace{0.5cm}
    \begin{tabular}{l l}
        Date of Examination: & \hspace{2cm} \\
        Session: & FN/AN \\
        Duration: & 3 hrs \\
        Subject No.: & HS51604 \\
        Subject Name: & German \\
        Department/Center/School: & Humanities \& Social Sciences \\
        Full Marks: & 50 \\
    \end{tabular}
\end{center}

\vspace{1cm}

% Question 1
\section*{Q.1 Translate the following passages into English (10 marks)}

A. Von Montag bis Freitag besuche ich vormittags die Schule. Der Unterricht dauert meistens bis um 13 Uhr. Manchmal habe ich nach der Mittagspause (lunch break) noch einmal Unterricht. In der Mittagspause können wir essen oder uns ausruhen (to take rest). Viele machen auch ihre Hausaufgaben \textcolor{red}{homework} in der Mittagspause. Nach der Schule muss ich die Hausaufgaben machen. Dafür brauche ich meistens nicht sehr viel Zeit. Oft nur eine Stunde. Danach habe ich Freizeit und kann machen, was ich möchte. Montags gehe ich am Nachmittag zum Sport. Ich spiele Tennis. Viele meiner Freunde machen Sport nach der Schule. Manche haben auch Musikunterricht.

Dienstag und Donnerstag gehe ich außerdem zum Fußballtraining. Ich spiele mit meinem Bruder und vielen Freunden in einer Mannschaft (team). Am Samstag oder am Sonntag sind oft Spiele gegen die anderen Fußballvereine, das macht am meisten Spaß. Abends esse ich gemeinsam mit meiner Familie. Anschließend kann ich noch ein wenig am Computer spielen oder mir einen Film ansehen. Während der Woche gehe ich selten nach 22 Uhr schlafen, weil ich früh am Morgen aufstehen muss. Denn die Schule beginnt bei mir schon um 7:45 Uhr. Mittwoch nachmittags habe ich kein besonderes Programm, meistens treffe ich Freunde oder mache Erledigungen (accomplishment) mit meiner Mutter.

Samstag und Sonntag ist keine Schule. Aber oft muss ich für Schularbeiten oder Tests lernen. So habe ich meistens auch am Wochenende etwas für die Schule zu erledigen. Aber es bleibt doch Zeit für einen Besuch bei meiner Großmutter. Manchmal sind wir auch bei meinem Onkel zum Essen eingeladen oder wir machen einen Ausflug aufs Land.

B. Ich wohne in Frankfurt am Main. Die Stadt hat über 700000 Einwohner, sie ist die fünftgrößte Stadt Deutschlands. Mir gefallt die Stadt, weil sie so international ist. Hier leben Menschen aus vielen Kulturen. Um den Hauptbahnhof herum gibt es viele internationale Lebensmittelgeschäfte und Restaurants. Frankfurt ist eine moderne Stadt mit vielen Hochhäusern, aber es gibt auch eine schöne Altstadt mit gemütlichen Kneipen \textcolor{red}{pubs}. Dort kann man Apfelwein trinken und Grüne Soße (sauce) essen. Das Frankfurter Nationalgericht besteht aus Kräutern \textcolor{red}{herbs}, Joghurt und anderen Zutaten. In Frankfurt steht auch das Goethehaus, das Geburtshaus des berühmten deutschen Dichters Johann Wolfgang von Goethe. Eine Schifffahrt auf dem Main macht viel Spaß. Man kann aber auch gut am Fluss spazieren gehen. Im Sommer finden hier viele Feste statt. Jedes Jahr kommen mehrere Millionen Menschen zum Museumsuferfest. Es dauert drei Tage und es gibt ein interessantes Programm mit viel Live-Musik.

\subsection*{Answer:}

\textbf{(A)} From Monday to Friday, I go to school in the mornings. Classes usually last until 1 p.m. Sometimes, I have lessons again after the lunch break. During the lunch break, we can eat or take a rest. Many students also do their homework during the lunch break. After school, I have to do my homework. That usually doesn't take very long—often just one hour. After that, I have free time and can do whatever I want. On Monday afternoons, I go to sports practice. I play tennis. Many of my friends also do sports after school. Some also have music lessons.

On Tuesdays and Thursdays, I also go to football training. I play in a team with my brother and many friends. On Saturday or Sunday, there are often matches against other football clubs—that’s the most fun. In the evenings, I eat together with my family. Afterward, I can play on the computer for a bit or watch a movie. During the week, I rarely go to bed after 10 p.m., because I have to get up early in the morning. School already starts at 7:45 a.m. on weekdays. On Wednesday afternoons, I don’t have any special program—usually, I meet friends or run errands with my mother.

On Saturdays and Sundays, there is no school. But I often have to study for school assignments or tests. So I usually have school-related things to do on the weekend too. But there’s still time for a visit to my grandmother. Sometimes we are invited to my uncle’s for lunch, or we go on a trip to the countryside.

\vspace{1em}

\textbf{(B)} I live in Frankfurt am Main. The city has over 700,000 inhabitants and is the fifth largest city in Germany. I like the city because it is so international. People from many different cultures live here. Around the main train station, there are many international grocery stores and restaurants. Frankfurt is a modern city with many skyscrapers, but there is also a beautiful old town with cozy pubs. There, you can drink apple wine and eat green sauce. Frankfurt’s national dish is made of herbs, yogurt, and other ingredients. Frankfurt is also home to the Goethe House, the birthplace of the famous German poet Johann Wolfgang von Goethe. A boat trip on the Main is a lot of fun. But you can also enjoy walking along the river. In summer, many festivals take place here. Every year, several million people come to the Museum Embankment Festival. It lasts three days and offers an interesting program with lots of live music.


\vspace{0.5cm}

Vocabulary: 
besuchen = to visit; Mittagspause = lunch break; ausruhen = to rest; Freizeit = leisure time; Mannschaft = team; Erledigungen = errands; international = international; Kulturen = cultures; gemütlich = cozy; Nationalgericht = national dish; Schifffahrt = boat trip; Museumsuferfest = museum riverbank festival.

\vspace{1cm}

% Question 2
\section*{Q.2 Fill in the blanks with appropriate prepositions (any ten) (10 marks)}

(a) Mein Freund stellt das Buch --- dem Tisch.\\
(b) --- 13.00 Uhr gehen die Studenten zum Theater.\\
(c) Diese Bücher sind nur --- die kleinen Kinder.\\
(d) Diese Studenten kommen --- Japan und studieren in Neu Delhi.\\
(e) --- 2022 arbeitet er in London bei Siemens.\\
(f) Das Krankenhaus liegt --- der Schule.\\
(g) --- dem Essen gehen sie mit ihren Freunden spazieren.\\
(h) Eine schöne Dame steht --- der Tür und fragt Herrn Robert.\\
(i) Die Studenten fahren jetzt --- Heidelberg.\\
(j) --- einen guten Freund fährt sie nie mit dem Zug.\\
(k) Morgen besucht Familie Kühn --- Kindern das Museum.

\subsection*{Answer:}
\textcolor{darkgreen}{(a) Mein Freund stellt das Buch \textbf{bei} dem Tisch.\\}
\textcolor{darkgreen}{(b) \textbf{Um} 13.00 Uhr gehen die Studenten zum Theater.\\}
\textcolor{darkgreen}{(c) Diese Bücher sind nur \textbf{für} die kleinen Kinder.\\}
\textcolor{darkgreen}{(d) Diese Studenten kommen \textbf{aus} Japan und studieren in Neu Delhi.\\}
\textcolor{darkgreen}{(e) \textbf{Seit} 2022 arbeitet er in London bei Siemens.\\}
\textcolor{darkgreen}{(f) Das Krankenhaus liegt der Schule \textbf{gegenüber}.\\}
\textcolor{darkgreen}{(g) \textbf{Nach} dem Essen gehen sie mit ihren Freunden spazieren.\\}
\textcolor{darkgreen}{(h) Eine schöne Dame steht \textbf{vor} der Tür und fragt Herrn Robert.\\}
\textcolor{darkgreen}{(i) Die Studenten fahren jetzt \textbf{nach} Heidelberg.\\}
\textcolor{darkgreen}{(j) \textbf{Ohne} einen guten Freund fährt sie nie mit dem Zug.\\}
\textcolor{darkgreen}{(k) Morgen besucht Familie Kühn \textbf{mit} Kindern das Museum.\\}


\vspace{1cm}

% Question 3
\section*{Q.3 Form sentences with the following groups of words (any five)! (10 marks)}

(a) Mein Bruder arbeitet bei Technogerma in Dubai.\\
(b) Robert schenkt seinem Freund einen Füller.\\
(c) Wie lange liest der Professor die Zeitung?\\
(d) Die Lehrerin arbeitet täglich zehn Stunden.\\
(e) Sie brauchen das Auto heute nicht.\\
(f) Die Lehrerin kauft dem Kind ein Buch heute.

\subsection*{Answer:}
\textcolor{darkgreen}{(a) Mein Bruder \textbf{arbeitet} bei Technogerma in Dubai.\\}
\textcolor{darkgreen}{(b) Robert \textbf{schenkt} seinem Freund einen Füller.\\}
\textcolor{darkgreen}{(c) Wie lange \textbf{liest} der Professor die Zeitung?\\}
\textcolor{darkgreen}{(d) Die Lehrerin \textbf{arbeitet} täglich zehn Stunden.\\}
\textcolor{darkgreen}{(e) Sie \textbf{brauchen} das Auto heute nicht.\\}
\textcolor{darkgreen}{(f) Die Lehrerin \textbf{kauft} dem Kind ein Buch heute.\\}

\vspace{1cm}

% Question 4
\section*{Q.4 Add appropriate adjective endings (any five sentences)! (10 marks)}

(a) Der jung--- Arbeiter hat ein neu--- Auto.\\
(b) Eine indisch--- Frau schenkt den klein--- Studenten neue Bücher.\\
(c) Die klein--- Kinder sollen diesen schrecklich--- Film nicht sehen.\\
(d) Die alt--- Dame wohnt in einem groß--- Haus in Bangalore.\\
(e) Die ausländisch--- Studentin fährt mit einem schön--- Auto.\\
(f) Der reich--- Mann hat ein teuer--- Haus.

\subsection*{Answer:}
\textcolor{darkgreen}{(a) Der \textbf{junge} Arbeiter hat ein \textbf{neues} Auto.\\}
\textcolor{darkgreen}{(b) Eine \textbf{indische} Frau schenkt den \textbf{kleinen} Studenten neue Bücher.\\}
\textcolor{darkgreen}{(c) Die \textbf{kleinen} Kinder sollen diesen \textbf{schrecklichen} Film nicht sehen.\\}
\textcolor{darkgreen}{(d) Die \textbf{alte} Dame wohnt in einem \textbf{großen} Haus in Bangalore.\\}
\textcolor{darkgreen}{(e) Die \textbf{ausländische} Studentin fährt mit einem \textbf{schönen} Auto.\\}
\textcolor{darkgreen}{(f) Der \textbf{reiche} Mann hat ein \textbf{teures} Haus.\\}

\vspace{1cm}

% Question 5
\section*{Q.5 Fill in the blanks with appropriate words (5 marks)}

(a) Sind Sie Student? Nein, ich bin ---.\\
(b) Viele Studenten --- nur  Fußball.\\
(c) Wie--- dauert die Fahrt von Berlin bis Magdeburg?\\
(d) Wo ist der ---? Er ist zu Haus.\\
(e) Ich verstehe den Satz nicht. Können Sie bitte noch --- erklären?\\
(f) --- fahren Sie nach Deutschland?\\
(g) Der Student versteht den Satz ---.\\
(h) Wir haben nicht --- Zeit.\\
(i) Ohne --- kann man nicht leben.\\
(j) In Indien trinken die Leute --- gern.

\subsection*{Answer:}
\textcolor{darkgreen}{(a) Sind Sie Student? Nein, ich bin \textbf{Lehrer}.\\}
\textcolor{darkgreen}{(b) Viele Studenten spielen nur \textbf{Fußball}.\\}
\textcolor{darkgreen}{(c) Wie \textbf{lange} dauert die Fahrt von Berlin bis Magdeburg?\\}
\textcolor{darkgreen}{(d) Wo ist der \textbf{Lehrer}? Er ist zu Haus.\\}
\textcolor{darkgreen}{(e) Ich verstehe den Satz nicht. Können Sie bitte noch \textbf{einmal} erklären?\\}
\textcolor{darkgreen}{(f) \textbf{Wann} fahren Sie nach Deutschland?\\}
\textcolor{darkgreen}{(g) Der Student versteht den Satz \textbf{nicht}.\\}
\textcolor{darkgreen}{(h) Wir haben nicht \textbf{mehr} Zeit.\\}
\textcolor{darkgreen}{(i) Ohne \textbf{Wasser} kann man nicht leben.\\}
\textcolor{darkgreen}{(j) In Indien trinken die Leute \textbf{Tee} gern.\\}
\

\vspace{1cm}

% Question 6
\section*{Q.6 Use modal verbs (any five sentences)! (5 marks)}

(a) Die Leute trinken viel Bier hier.\\
(b) Robert fährt nach Haus mit dem Bus.\\
(c) Der Gast trinkt Tee mit Milch und Zucker.\\
(d) Die Lehrerin bringt ein neues Buch heute.\\
(e) Die Kinder spielen nur mit ihren Freunden.\\
(f) Die Frau bringt dem Gast ein Glas Wasser.

\subsection*{Answer:}
\textcolor{darkgreen}{(a) Die Leute \textbf{mögen} viel Bier hier trinken.\\}
\textcolor{darkgreen}{(b) Robert \textbf{will} nach Haus mit dem Bus fahren.\\}
\textcolor{darkgreen}{(c) Der Gast \textbf{mag} Tee mit Milch und Zucker trinken.\\}
\textcolor{darkgreen}{(d) Die Lehrerin \textbf{kann} ein neues Buch heute bringen.\\}
\textcolor{darkgreen}{(e) Die Kinder \textbf{dürfen} nur mit ihren Freunden spielen.\\}
\textcolor{darkgreen}{(f) Die Frau \textbf{soll} dem Gast ein Glas Wasser bringen.\\}

\vspace{1cm}

% Question 7
\section*{Q.7 Change the following sentences into past tense (any five)! (5 marks)}

(a) Die Sekretärin kauft ein neues Auto heute.\\
(b) Die Mutter schenkt ihrem Sohn eine schöne Uhr.\\
(c) Die Studenten arbeiten in dem Institut und auch zu Haus.\\
(d) Herr Martin trinkt Tee ohne Milch und Zucker.\\
(e) Er schreibt seiner Mutter einen Brief am Wochenende.\\
(f) Dieser Mann braucht kein Auto heute.\\
(g) Viele Inder fahren nach Deutschland im Winter.

\subsection*{Answer:}
\textcolor{darkgreen}{(a) Die Sekretärin \textbf{kaufte} ein neues Auto heute.\\} \textcolor{red}{(Kaufen - kaufte - gekauft)}\\
\textcolor{darkgreen}{(b) Die Mutter \textbf{schenkte} ihrem Sohn eine schöne Uhr.\\} \textcolor{red}{(Schenken - schenkte - geschenkt)}\\
\textcolor{darkgreen}{(c) Die Studenten \textbf{arbeiteten} in dem Institut und auch zu Haus.\\} \textcolor{red}{(Arbeiten - arbeitete - gearbeitet)}\\
\textcolor{darkgreen}{(d) Herr Martin \textbf{trank} Tee ohne Milch und Zucker.\\} \textcolor{red}{(Trinken - trank - getrunken)}\\
\textcolor{darkgreen}{(e) Er \textbf{schrieb} seiner Mutter einen Brief am Wochenende.\\} \textcolor{red}{(Schreiben - schrieb - geschrieben)}\\
\textcolor{darkgreen}{(f) Dieser Mann \textbf{brauchte} kein Auto heute.\\} \textcolor{red}{(Brauchen - brauchte - gebraucht)}\\
\textcolor{darkgreen}{(g) Viele Inder \textbf{fuhren} nach Deutschland im Winter.\\} \textcolor{red}{(Fahren - fuhr - gefahren)}\\

\vspace{1cm}

% Question 8
\section*{Q.8 Translate the following sentences into German (any five)! (5 marks)}

(a) They purchased a new car yesterday.\\
(b) Have you already visited your mother?\\
(c) Many students will visit Germany in the weekend.\\
(d) He reads a German newspaper after lunch.\\
(e) She has brought a new book from Stuttgart.\\
(f) The examination will start from Tuesday.

\subsection*{Answer:}
\textcolor{darkgreen}{(a) Sie \textbf{kauften} ein neues Auto gestern.\\} \textcolor{darkgreen}{(b) Haben Sie schon Ihre Mutter besucht?\\}
\textcolor{darkgreen}{(c) Viele Studenten \textbf{werden} Deutschland am Wochenende besuchen.\\}
\textcolor{darkgreen}{(d) Er \textbf{liest} eine deutsche Zeitung nach dem Mittagessen.\\}
\textcolor{darkgreen}{(e) Sie \textbf{hat} ein neues Buch aus Stuttgart gebracht.\\}
\textcolor{darkgreen}{(f) Die Prüfung \textbf{wird} am Dienstag beginnen.\\}

\end{document}
