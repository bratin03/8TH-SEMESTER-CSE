\documentclass[a4paper,12pt]{article}
\usepackage[utf8]{inputenc}
\usepackage{geometry}
\geometry{margin=1in}
\usepackage{amsmath}
\usepackage{amssymb}
\usepackage{titlesec}
\usepackage{xcolor}
\definecolor{darkgreen}{rgb}{0.0, 0.5, 0.0}



% Formatting for sections
\titleformat{\section}{\normalfont\Large\bfseries}{\thesection}{1em}{}
\titleformat{\subsection}{\normalfont\large\bfseries}{\thesubsection}{1em}{}

\begin{document}

% Title
\begin{center}
    \textbf{INDIAN INSTITUTE OF TECHNOLOGY KHARAGPUR}\\
    \textbf{End-Autumn Semester Examination 2024-25}\\
    \vspace{0.5cm}
    \begin{tabular}{l l}
        Date of Examination: & \hspace{2cm} Session: \\
        Duration: & 3 hrs. \\
        Full Marks: & 50 \\
        Subject No.: & HS51604 \\
        Subject: & German \\
    \end{tabular}\\
    \vspace{0.5cm}
    Department of Humanities \& Social Sciences
\end{center}

\vspace{1cm}

% Question 1
\section*{Q.1 Translate the following passages into English (10 marks)}

Mein Name ist Andrea Müller und meine Familie lebt nicht gemeinsam \textcolor{red}{together} an einem Ort \textcolor{red}{Place}, sondern \textcolor{red}{rather} ist über mehrere Bundesländer \textcolor{red}{federal stehen} innerhalb Deutschlands verstreut. Ursprünglich \textcolor{red}{Originally} komme ich aus Nordrhein-Westfalen \textcolor{red}{North Rhine-Westphalia
} und habe in Köln studiert. Nach Abschluss des Studiums fand ich jedoch nicht gleich eine Arbeit, die mir zusagte und so entschied ich mich, zunächst einmal ins Ausland zu gehen und Erfahrungen zu sammeln.

Ich lebte zwei Jahre lang in den Niederlanden, wo es mir sehr gut gefiel und ich sowohl meine Englischkenntnisse verbessern, als auch die niederländische Sprache als neue Fremdsprache \textcolor{red}{foreign language} hinzulernen konnte. Mit dieser internationalen Berufserfahrung \textcolor{red}{Work experience} und den erweiterten Sprachkenntnissen fand ich eine Anstellung \textcolor{red}{Employment} in Hessen.

Dort lernte ich auch meinen Mann kennen, der ursprünglich aus Bayern \textcolor{red}{Bavaria} stammt. Wir heirateten und bekamen zwei Söhne. In Hessen haben wir uns inzwischen einen größeren Kreis an Freunden und Bekannten \textcolor{red}{acquaintances} aufgebaut, unsere Familien leben jedoch noch immer größtenteils in Nordrhein-Westfalen und Bayern. Hinzu kommt, dass meine fünf Geschwister ebenfalls nicht in Nordrhein-Westfalen sesshaft geworden sind, sondern über die gesamte Bundesrepublik Deutschland verstreut leben.

Nur bei größeren Familienfesten und Geburtstagen sehen wir uns alle. Ich würde sehr gern in der Nähe meiner Eltern leben, da diese mittlerweile auch ziemlich alt sind und sicherlich bald Unterstützung benötigen. Auch unsere Kinder vermissen die Großeltern und Verwandten oft. Unsere mittelfristige Perspektive ist es daher, für meinen Mann und mich in der nächsten Zeit Arbeitsstellen und ein Haus in Nordrhein-Westfalen zu finden.

\vspace{0.5cm}

Vocabulary: 
versteuern = to scatter; Ursprünglich = originally; Abschluss = completion; zusagen = to promise; entscheiden = to decide; Erfahrungung = experience; sammeln = to gain; Anstellung = job; heiraten = to marry; Kreis = circle; Bekannten = acquaintances; aufbauen = build up; sesshaften = to settle; ziemlich = quite; Unterstützung = support; benötigen = to need; Verwandten = relatives

\subsection*{Answer:}

My name is Andrea Müller, and my family does not live together in one place, but is spread across several federal states within Germany. I originally come from North Rhine-Westphalia and studied in Cologne. After completing my studies, I did not immediately find a job that suited me, so I decided to go abroad for a while and gain experience.

I lived in the Netherlands for two years, where I really enjoyed myself. There, I was able to improve my English skills and also learn Dutch as a new foreign language. With this international work experience and my improved language skills, I found a job in Hesse.

It was there that I also met my husband, who originally comes from Bavaria. We got married and had two sons. In Hesse, we have now built up a larger circle of friends and acquaintances, but our families still mostly live in North Rhine-Westphalia and Bavaria. In addition, my five siblings also did not settle in North Rhine-Westphalia, but live scattered throughout the whole of Germany.

We only all see each other at larger family celebrations and birthdays. I would really like to live near my parents, as they are now quite old and will surely need support soon. Our children also often miss their grandparents and relatives. Therefore, our medium-term goal is to find jobs and a house for my husband and me in North Rhine-Westphalia in the near future.


\vspace{0.5cm}

% Question 2
\section*{Q.2 Fill in the blanks with appropriate prepositions (any ten) (10 marks)}

(a) Das Schwimmbad liegt dem Krankenhaus --------.\\
(b) -------- dem Essen tanzen die Leute mit den Freunden.\\
(c) Die Studenten stehten -------- der Tür und fragen Professor Schmid.\\
(d) Die Studenten gehen jetzt -------- das Klassenzimmer.\\
(e) Dieser Bücher liegen -------- dem Tisch in dem Schlafzimmer.\\
(f) -------- 18.00 Uhr fahren die Freunde nach Regensburg.\\
(g) Diese Bleistifte sind nur -------- die armen Kinder.\\
(h) Diese Studenten kommen -------- Japan und studieren in Indien.\\
(i) -------- 1988 arbeiteten viele Ausländer in Deutschland.\\
(j) Er trinkt Tee -------- Zucker und Milch.\\
(k) Am Morgen geht er immer -------- den Park spazieren.

\subsection*{Answer:}
\textcolor{darkgreen}{(a) Das Schwimmbad liegt dem Krankenhaus \textbf{gegenüber}.}\\
\textcolor{darkgreen}{(b) \textbf{Nach} dem Essen tanzen die Leute mit den Freunden.}\\
\textcolor{darkgreen}{(c) Die Studenten stehen \textbf{vor} der Tür und fragen Professor Schmid. | Die Studenten stehen \textbf{an} der Tür und fragen Professor Schmid.}\\
\textcolor{darkgreen}{(d) Die Studenten gehen jetzt \textbf{in} das Klassenzimmer.}\\
\textcolor{darkgreen}{(e) Diese Bücher liegen \textbf{auf} dem Tisch in dem Schlafzimmer.}\\
\textcolor{darkgreen}{(f) \textbf{Um} 18.00 Uhr fahren die Freunde nach Regensburg.}\\
\textcolor{darkgreen}{(g) Diese Bleistifte sind nur \textbf{für} die armen Kinder.}\\
\textcolor{darkgreen}{(h) Diese Studenten kommen \textbf{aus} Japan und studieren in Indien.}\\
\textcolor{darkgreen}{(i) \textbf{Seit} 1988 arbeiteten viele Ausländer in Deutschland.}\\
\textcolor{darkgreen}{(j) Er trinkt Tee \textbf{mit} Zucker und Milch. | Er trinkt Tee \textbf{ohne} Zucker und Milch.}\\
\textcolor{darkgreen}{(k) Am Morgen geht er immer \textbf{durch} den Park spazieren.}\\
\vspace{0.5cm}


% Question 3
\section*{Q.3 Use modal verbs (any five sentences) (5 marks)}

(a) Die Studenten lesen die Zeitung \textcolor{magenta}{(newspaper)} abends.\\
(b) Die Studenten spielen mit dem roten \textcolor{magenta}{(red)} Fußball.\\
(c) Viele Kinder verstehen \textcolor{magenta}{(understand)} dieses Wort nicht.\\
(d) Die Dame trinkt Tee mit Milch und Zucker.\\
(e) Spricht die Ausländerin \textcolor{magenta}{(Foreiger lady)} schon \textcolor{magenta}{already} gut Hindi?\\
(f) Die Lehrerin kauft ihrem Sohn ein neues Buch heute.

\subsection*{Answer:}
\textcolor{darkgreen}{(a) Die Studenten \textbf{sollen} abends die Zeitung lesen.}\\
\textcolor{darkgreen}{(b) Die Studenten \textbf{wollen} mit dem roten Fußball spielen.}\\
\textcolor{darkgreen}{(c) Viele Kinder \textbf{können} dieses Wort nicht verstehen.}\\
\textcolor{darkgreen}{(d) Die Dame \textbf{will} Tee mit Milch und Zucker trinken.}\\
\textcolor{darkgreen}{(e) \textbf{Kann} die Ausländerin schon gut Hindi sprechen?}\\
\textcolor{darkgreen}{(f) Die Lehrerin \textbf{will} ihrem Sohn ein neues Buch kaufen.}



\vspace{0.5cm}

% Question 4
\section*{Q.4 Add appropriate adjective endings (any five sentences) (5 marks)}

(a) Die reich-- Dame wohnt in einem groß-- Haus in Neu Delhi.\\
(b) Viele neu Bücher liegen in dem rot -- Schrank.\\
(c) Die klein-- Kinder sollen den schrecklich -- Film nicht sehen.\\
(d) Die ausländisch -- Studentin fährt mit einem schön-- Auto.\\
(e) Alle jung-- Arbeiter haben ein alt -- Auto.\\
(f) Eine deutsch-- Frau schenkt den arm-- Leuten Schokolade.
\subsection*{Answer:}
\textcolor{darkgreen}{(a) Die \textbf{reiche} Dame wohnt in einem \textbf{großen} Haus in Neu Delhi.}\\
\textcolor{darkgreen}{(b) Viele \textbf{neue} Bücher liegen in dem \textbf{roten} Schrank.}\\
\textcolor{darkgreen}{(c) Die \textbf{kleinen} Kinder sollen den \textbf{schrecklichen} Film nicht sehen.}\\
\textcolor{darkgreen}{(d) Die \textbf{ausländische} Studentin fährt mit einem \textbf{schönen} Auto.}\\
\textcolor{darkgreen}{(e) Alle \textbf{jungen} Arbeiter haben ein \textbf{altes} Auto.}\\
\textcolor{darkgreen}{(f) Eine \textbf{deutsche} Frau schenkt den \textbf{armen} Leuten Schokolade.}\\

\vspace{0.5cm}

% Question 5
\section*{Q.5 Write ten sentences in German on ‘Ihr Lehrer’ or ‘Ihre Familie’ (5 marks)}

\vspace{0.5cm}

% Question 6
\section*{Q.6 Change the following sentences into present perfect tense (any five) (5 marks)}

(a) Peter Krein schreibt seinem Vater einen Brief.\\
(b) Der Professor fährt nach Deutschland mit seinen Studenten.\\
(c) Viele Ausländer trinken Tee ohne Milch und Zucker.\\
(d) Der Lehrer erklärt den Studenten die Wörter.\\
(e) Der Student spricht Deutsch sehr langsam aber richtig.\\
(f) Das Mädchen bringt dem Gast eine Tasse Kaffee.\\
(g) Die Studenten finden das Buch in Neu Delhi.

\subsection*{Answer:}
\textcolor{darkgreen}{(a) Peter Krein \textbf{hat} seinem Vater einen Brief \textbf{geschrieben}.}\\ \textcolor{red}{(Schreiben - Schrieb - Geschrieben)}\\
\textcolor{darkgreen}{(b) Der Professor \textbf{ist} nach Deutschland mit seinen Studenten \textbf{gefahren}.}\\ \textcolor{red}{(Fahren - Fuhr - Gefahren) (bin is used for "fahren")}\\
\textcolor{darkgreen}{(c) Viele Ausländer \textbf{haben} Tee ohne Milch und Zucker \textbf{getrunken}.}\\ \textcolor{red}{(Trinken - Trank - Getrunken)}\\
\textcolor{darkgreen}{(d) Der Lehrer \textbf{hat} den Studenten die Wörter \textbf{erklärt}.}\\ \textcolor{red}{(Erklären - Erklärte - Erklärt)}\\
\textcolor{darkgreen}{(e) Der Student \textbf{hat} Deutsch sehr langsam aber richtig \textbf{gesprochen}.}\\ \textcolor{red}{(Sprechen - Sprach - Gesprochen)}\\
\textcolor{darkgreen}{(f) Das Mädchen \textbf{hat} dem Gast eine Tasse Kaffee \textbf{gebracht}.}\\ \textcolor{red}{(Bringen - Brachte - Gebracht)}\\
\textcolor{darkgreen}{(g) Die Studenten \textbf{haben} das Buch in Neu Delhi \textbf{gefunden}.}\\ \textcolor{red}{(Finden - Fand - Gefunden)}\\

\vspace{0.5cm}

% Question 7
\section*{Q.7 Fill in the blanks by using appropriate words (5 marks)}

(a) Dieser Schüler versteht das Wort------ .\\
(b) ------- habe ich nicht mehr Zeit .\\
(c) Es gibt zehn -------- in dem Zimmer .\\
(d) Wie---- dauert die Fahrt von Howrah bis Tatanagar? .\\
(e) Wo ist der Professor ? Er ist --- seinem Zimmer .\\
(f) Ich verstehe nicht . Können Sie bitte noch ------- erklären ? .\\
(g) -------- fahren Sie nach Hamburg? .\\
(h) Ohne ----- kann man nicht leben .\\
(i) Viele Ausländer sprechen -------- gut .\\ 
(j) Sind Sie ---------- . Nein , ich bin Inder .

\subsection*{Answer:}
\textcolor{darkgreen}{(a) Dieser Schüler versteht das Wort \textbf{nicht}.}\\
\textcolor{darkgreen}{(b) \textbf{Jetzt} habe ich nicht mehr Zeit.}\\
\textcolor{darkgreen}{(c) Es gibt zehn \textbf{Studenten} in dem Zimmer.}\\
\textcolor{darkgreen}{(d) Wie \textbf{lange} dauert die Fahrt von Howrah bis Tatanagar?}\\
\textcolor{darkgreen}{(e) Wo ist der Professor? Er ist \textbf{in} seinem Zimmer.}\\
\textcolor{darkgreen}{(f) Ich verstehe nicht. Können Sie bitte noch \textbf{einmal} erklären?}\\
\textcolor{darkgreen}{(g) \textbf{Wann} fahren Sie nach Hamburg?}\\
\textcolor{darkgreen}{(h) Ohne \textbf{Wasser} kann man nicht leben.}\\
\textcolor{darkgreen}{(i) Viele Ausländer sprechen \textbf{Deutsch} gut.}\\
\textcolor{darkgreen}{(j) Sind Sie \textbf{Deutscher}? Nein, ich bin Inder.}\\

\vspace{0.5cm}

% Question 8 
\section*{Q8 Translate the following sentences into German (any five): (5 marks)}

(a) Some students read a German newspaper in the morning . \\
(b ) Our examination will start on Friday . \\
(c ) Children want to play football with their parents . \\
(d ) The teacher comes very late today ; he is not well . \\
(e ) Some foreigners speak Hindi very well . \\
(f ) Have you already visited your sister at the weekend ? \\
(g ) Can you please explain this sentence to me once again ?

\subsection*{Answer:}
\textcolor{darkgreen}{(a) Einige Studenten lesen am Morgen eine deutsche Zeitung.}\\
\textcolor{darkgreen}{(b) Unsere Prüfung wird am Freitag beginnen.}\\
\textcolor{darkgreen}{(c) Kinder wollen mit ihren Eltern Fußball spielen.}\\
\textcolor{darkgreen}{(d) Der Lehrer kommt heute sehr spät; er ist nicht wohl.}\\
\textcolor{darkgreen}{(e) Einige Ausländer sprechen Hindi sehr gut.}\\
\textcolor{darkgreen}{(f) Hast du am Wochenende deine Schwester schon besucht?}\\
\textcolor{darkgreen}{(g) Können Sie mir diesen Satz bitte noch einmal erklären?}\\

\end{document}
