\documentclass[12pt,a4paper]{article}
\usepackage[utf8]{inputenc}
\usepackage{amsmath}
\usepackage{geometry}
\geometry{margin=1in}
\setlength{\parindent}{0pt}
\setlength{\parskip}{1em}

\begin{document}

\begin{center}
\Large \textbf{INDIAN INSTITUTE OF TECHNOLOGY KHARAGPUR}\\
\large End-Autumn Semester 2018-19\\
\end{center}

\textbf{Date of Examination:} \\
\textbf{Subject No.:} HS51604 \\
\textbf{Session (FN/AN)} \hspace{1cm} \textbf{Duration:} 3 hrs \\
\textbf{Department:} Humanities \& Social Sciences \\
\textbf{Full Marks:} 70 \\
\textbf{Subject:} German

\vspace{1em}

\section*{Q.1. Translate the following passages into English:}

Rosa Echevarzu ist Sekretärin. Sie lernt Deutsch im Goethe Institut in La Paz in Bolivien. Sie kommt aus Santa Cruz. Rosa hat zwei Kinder, Juan und Lisa. Sie gehen in die Schule, Santra Barbara in La Paz. Sie lernen Englisch. Rosa möchte Deutsch sprechen. Sie sagt: ``Die Deutschkurse im Goethe-Institut sind interessant und gut für meine Arbeit.''

Boris Naumenkow kommt aus Kasachstan. Er lernt Deutsch in der Volkshochschule in Frankfurt am Main. Boris ist verheiratet mit Sina. Sie haben zwei Kinder, Lara und Natascha. Boris hat im Moment keine Arbeit. Die Naumenkows leben seit 2001 in Sprendlingen. Sie sprechen Russisch und Deutsch. Lara und Natascha lernen Englisch in der Schule. ``Deutschland ist für uns Sprache, Kultur, Heimat.''

Zhao Yafen ist Studentin. Sie lebt in Schanghai und studiert an der Tonji Universität. Sie ist 21 und möchte in Deutschland Biologie oder Chemie studieren. Ihre Hobbys sind Musik und Sport. Sie spielt Gitarre. Ihre Freundin Jin studiert Englisch. Sie möchte nach Kanada fahren. Deutsch ist für Yafen Musik. Sie sagt: ``Ich liebe Beethoven und Schubert.''

Heidi Klum kommt aus Bergisch Gladbach. Sie ist Model und präsentiert Mode (fashion) von internationalen Designern. Sie hat eine Mode-Kollektion und macht Werbung (advertisement) für H\&M und McDonalds. Heidi Klum arbeitet international, in Paris, York, Mailand und Düsseldorf. Sie spricht Deutsch in Manhattan und in Bergisch Gladbach. Sie hat eine Tochter, Leni. Designer-Mode ist ihr Job, zu Hause mag sie aber Jeans und T-Shirts. Sie macht viel Sport: Ballett und Jazz-Dance.

\section*{Q.2. Fill in the blanks!}

\begin{enumerate}
    \item Wie\underline{\hspace{2cm}} dauert die Fahrt von Howrah nach Kharagpur?
    \item Wo ist der Professor? Er fährt heute abend \underline{\hspace{2cm}} Deutschland.
    \item \underline{\hspace{2cm}} haben wir nicht mehr Zeit.
    \item Ohne \underline{\hspace{2cm}} kann man nicht leben.
    \item Viele Ausländer sprechen \underline{\hspace{2cm}} gut.
    \item Sind Sie \underline{\hspace{2cm}}. Nein, ich arbeite.
    \item Zehn Studenten aus Kharagpur \underline{\hspace{2cm}} nach Deutschland.
    \item Der Student schenkt der\underline{\hspace{2cm}} einen neuen Füller.
    \item Er versteht nicht. Können Sie bitte noch \underline{\hspace{2cm}} erklären?
    \item \underline{\hspace{2cm}} fahren Sie nach Berlin?
\end{enumerate}

\section*{Q.3. Use modal verbs (any ten)!}

\begin{enumerate}
    \item Die Studenten lesen die Zeitung abends.
    \item Die Kinder sehen nur diesen Film.
    \item Die Leute trinken hier gern Tee und Kaffee.
    \item Spricht diese Studentin schon gut Englisch?
    \item Viele Studenten verstehen diesen Satz nicht.
    \item Mein Freund trinkt gern Tee.
    \item Die Leute trinken nur Bier hier.
    \item Robert fährt nach Haus mit dem Bus.
    \item Sie versteht schon Hindi?
    \item Die Kinder spielen nur mit ihren Eltern.
    \item Meine Schwester spielt Tischtennis.
\end{enumerate}

\section*{Q.4. Write ten sentences in German about \emph{your friend} or about \emph{your teacher}!}

\begin{enumerate}
    \item die Mutter, der Sohn, kaufen, ein Fahrrad
    \item trinken, noch, eine Tasse, die Leute, Kaffee
    \item viele, fahren, Leute, nach, heute, Kalkutta
    \item er, Frau Brekle, eine Uhr, schenken, morgen
    \item er, sein Freund, schreiben, ein Brief, immer
    \item der Ausländer, gehen, zum Markt, täglich
\end{enumerate}

\section*{Q.6. Fill in the blanks with appropriate prepositions (any 10)!}

\begin{enumerate}
    \item Die Katze sitzt \underline{\hspace{2cm}} dem Tisch in dem Wohnzimmer.
    \item \underline{\hspace{2cm}} 11.00 Uhr gehen wir zum Seminar.
    \item Diese Bücher sind nur \underline{\hspace{2cm}} die kleinen Kinder.
    \item Eine Dame steht \underline{\hspace{2cm}} der Tür und fragt Herrn Brekle.
    \item Normalerweise legt sie die Bücher \underline{\hspace{2cm}} das Bett.
    \item Er arbeitet bis 20.00 Uhr und dann fährt \underline{\hspace{2cm}} Haus am Abend.
    \item Die Studenten fahren jetzt \underline{\hspace{2cm}} Japan.
    \item Viele Studenten kommen \underline{\hspace{2cm}} Thailand und studieren in Indien.
    \item \underline{\hspace{2cm}} vier Jahren arbeitet mein Freund in Amerika.
    \item Das Krankenhaus liegt \underline{\hspace{2cm}} dem Schwimmbad.
    \item \underline{\hspace{2cm}} dem Essen liest er die deutschen Zeitungen.
\end{enumerate}

\end{document}
